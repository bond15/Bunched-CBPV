\documentclass{article}
\usepackage{graphicx} % Required for inserting images
\usepackage{simplebnf}
\usepackage{bussproofs}
%\usepackage{hyperref}
\usepackage[llbracket,rrbracket]{stmaryrd}
\usepackage{amsmath}
\usepackage{amssymb}
\usepackage[dvipsnames]{xcolor}
\usepackage{cite}
\usepackage{scalerel}
\usepackage{tikz-cd}
\usepackage{oz}
\def\sq{\mathbin{\scalerel*{\strut\rule[-.5ex]{2ex}{2ex}}{\cdot}}}
\usepackage{kpfonts}
\usepackage{esint}
\usepackage{comment}
\usepackage{float}
\usepackage{appendix}



\DeclareFontFamily{U}{dmjhira}{}
\DeclareFontShape{U}{dmjhira}{m}{n}{ <-> dmjhira }{}

\DeclareRobustCommand{\yo}{\text{\usefont{U}{dmjhira}{m}{n}\symbol{"48}}}

% for eding diagrams https://tikzcd.yichuanshen.de/

\newcommand{\den}[1]{\llbracket #1 \rrbracket}
\newcommand{\blue}[1]{\textcolor{blue}{#1}}
\newcommand{\red}[1]{\textcolor{red}{#1}}
\newcommand{\sep}{\mathrel{-\mkern-6mu*}}
\newcommand{\thunk}[1]{\textrm{thunk }#1}
\newcommand{\injj}[2]{\textrm{inj}_{#1}#2}
\newcommand{\err}{\mho}
\newcommand{\print}[1]{\textrm{print }#1}
\newcommand{\force}[1]{\textrm{force }#1}
\newcommand{\ret}[1]{\textrm{ret }#1}
\newcommand{\bind}[3]{#1 \leftarrow #2 ; #3}
\newcommand{\newcase}[3]{\textrm{newcase}_{#1} \; #2 ; #3}
\newcommand{\match}[5]{\textrm{match }#1 \textrm{ with }#2 \;\{#3 . #4 | #5\}}
\newcommand{\unpack}[4]{\textrm{unpack }(#1,#2) = #3 ; #4}
\newcommand{\lett}[4]{\textrm{let }(#1,#2) = #3 ; #4}
\newcommand{\lets}[4]{\textrm{let }(#1*#2) = #3 ; #4}
\newcommand{\ite}[3]{\textrm{if }#1 \textrm{ then }#2 \textrm{ else }#3}
\newcommand{\at}{\textrm{@}}
\newcommand{\ttt}{\textrm{tt}}
\newcommand{\pworld}{\widehat{\mathbf{World}}}
\newcommand{\world}{{\mathbf{World}}}
\newcommand{\calculus}{\operatorname{-calculus}}




\begin{document}
\section{Untyped Syntax}
\begin{bnf}
$A$ : value types ::= $\textrm{Unit}$
| $\textrm{Bool}$
| $\textrm{Case A}$ 
| $\textrm{OSum}$
| $A\; \times \; A$  
| $A\; * \; A$
| $U\underline{B}$
;;
$B$ : computation types ::= $A \rightarrow \underline{B}$ 
| $A \sep \underline{B}$
| $F\;A$
;;
$V$ : values ::= $\ttt$
|$\true$
| $\false$
| $x$ : variable
| $\injj{V}{V}$ : injection into the open sum
| $(V,V)$
| $\pi_1 V$
| $\pi_2 V$
| $(V*V)$ : separating product
| $\rho_1 V$
| $\rho_2 V$
| $\thunk{M}$
;;
$M$ : computations ::= $\err$ : \red{need monad for this}
| $\force{V}$
| $\ret{V}$
| $\bind{x}{M}{N}$
| $M \; N$
| $M \at N$
| $\lambda x \colon A . M$
| $\alpha x \colon A . M$ : separating function
| $\newcase{A}{x}{M}$ : "allocate" a new case in the open sum
| $\match{V}{V}{\injj{}{x}}{M}{N}$ : match on open sum
| $\ite{V}{M}{N}$
;;
$\Gamma$ : value typing context ::= $\emptyset$ 
| $ x \; \colon A$
| $\Gamma ; \Gamma'$ : cartesian product
| $\Gamma \fatsemi \Gamma'$ : monoidal product
;;
$\Theta$ : stoup ::= $\cdot$ 
| $\bullet \; \colon B$
;;
\end{bnf}
\red{stoup forms?} could add computation products
\begin{comment}
    | $\lett{x}{y}{V}{M}$
| $\lets{x}{y}{V}{M}$
\end{comment}

\newpage
\section{Typed Syntax}
This version of the syntax is a combination of the $\nu\calculus$\cite{stark:namerp-tlca}
\cite{starkCategoricalModelsLocal1996} of Pitts and Stark and the affine $\alpha\lambda\calculus$ of O'Hearn 
\cite{ohearnResourceInterpretationsBunched1999} expressed in CBPV and with the addition of an extensible sum type.
\blue{for (rule airity $>$ 1) non separating connectives, contexts don't need to be appended($\Gamma ; \Delta$)?}
\subsection{Structural Rules}
; admits weakening, contraction, and exchange (cartesian closed structure)\\
$\fatsemi$ admits weakening and exchange (semicartisian symmetric monoidal structure)
\subsection{Value Typing}
\subsubsection{Unit}



\begin{prooftree}
\AxiomC{}
\RightLabel{Intro}
\UnaryInfC{$\Gamma \vdash_v \ttt : \textrm{Unit}$}
\end{prooftree}

\subsubsection{Bool}
\begin{prooftree}
\AxiomC{}
\RightLabel{$\textrm{Intro}_1$}
\UnaryInfC{$\Gamma \vdash_v \true : \textrm{Bool}$}
\end{prooftree}

\begin{prooftree}
\AxiomC{}
\RightLabel{$\textrm{Intro}_2$}
\UnaryInfC{$\Gamma \vdash_v \false  : \textrm{Bool}$}
\end{prooftree}

\begin{prooftree}
\AxiomC{$\Gamma \vdash_v V : \textrm{Bool}$}
\AxiomC{$\Gamma | \cdot \vdash_c M_1 : B$}
\AxiomC{$\Gamma | \cdot \vdash_C M_2 : B$}
\RightLabel{Elim}
\TrinaryInfC{$\Gamma | \cdot \vdash_c \ite{V}{M_1}{M_2} : B$}
\end{prooftree}

\subsubsection{Var}
\begin{prooftree}
\AxiomC{}
\RightLabel{Intro}
\UnaryInfC{$x : \textrm{A}  \vdash_v x : \textrm{A}$}
\end{prooftree}

\subsubsection{OSum}
\begin{prooftree}
\AxiomC{$\Gamma \vdash_v V_c : \textrm{Case A}$}
\AxiomC{$\Gamma \vdash_v V : A$}
\RightLabel{Intro}
\BinaryInfC{$\Gamma \vdash_v \injj{V_c}{V} : \textrm{OSum}$}
\end{prooftree}

\begin{prooftree}
\AxiomC{$\Gamma \vdash_v V_T : \textrm{Case A}$}
\AxiomC{$\Gamma \vdash_v V : \textrm{OSum}$}
\AxiomC{$\Gamma ; (x : A)| \cdot \vdash_c M : B$}
\AxiomC{$\Gamma | \cdot \vdash_c N : B$}
\RightLabel{Elim}
\QuaternaryInfC{$\Gamma | \cdot \vdash_c \match{V_T}{V}{\textrm{inj}\;x}{M}{N}$}
\end{prooftree}

\subsubsection{Value Product}
\begin{prooftree}
\AxiomC{$\Gamma \vdash_v N : A_1$}
\AxiomC{$\Gamma \vdash_v M : A_2$}
\RightLabel{Intro}
\BinaryInfC{$\Gamma \vdash_v (N , M) : A_1 \times A_2$}
\end{prooftree}

\begin{prooftree}
\AxiomC{$\Gamma \vdash_v P : A_1 \times A_2$}
\RightLabel{$\textrm{Elim}_1$}
\UnaryInfC{$\Gamma \vdash_v \pi_1 P : A_1$}
\end{prooftree}

\begin{prooftree}
\AxiomC{$\Gamma \vdash_v P : A_1 \times A_2$}
\RightLabel{$\textrm{Elim}_2$}
\UnaryInfC{$\Gamma \vdash_v \pi_2 P : A_2$}
\end{prooftree}

\subsubsection{Separating Product}
\blue{we have these because the monoidal product is semicartesian}
\begin{prooftree}
\AxiomC{$\Gamma \vdash_v M : A_1$}
\AxiomC{$\Delta \vdash_v N : A_2$}
\RightLabel{Intro}
\BinaryInfC{$\Gamma \fatsemi \Delta \vdash_v (M * N) : A_1 * A_2$}
\end{prooftree}

\begin{prooftree}
\AxiomC{$\Gamma \fatsemi \Delta \vdash_v P : A_1 * A_2$}
\RightLabel{$\textrm{Elim}_1$}
\UnaryInfC{$\Gamma \fatsemi \Delta \vdash_v \rho_1 P : A_1$}
\end{prooftree}

\begin{prooftree}
\AxiomC{$\Gamma \fatsemi \Delta \vdash_v P : A_1 * A_2$}
\RightLabel{$\textrm{Elim}_2$}
\UnaryInfC{$\Gamma \fatsemi \Delta \vdash_v \rho_2 P : A_2$}
\end{prooftree}

\subsubsection{Thunk}
\begin{prooftree}
\AxiomC{$\Gamma | \cdot \vdash_c M : B$}
\RightLabel{Intro}
\UnaryInfC{$\Gamma \vdash_v  \thunk{M} : U B$}
\end{prooftree}

\begin{prooftree}
\AxiomC{$\Gamma \vdash_v V : U B$}
\RightLabel{Elim}
\UnaryInfC{$\Gamma | \cdot \vdash_c \force{V} : B$}
\end{prooftree}

\subsection{Computation Typing}
\subsubsection{Function}
\begin{prooftree}
\AxiomC{$\Gamma ; (x : A) | \cdot \vdash_c V : B$}
\RightLabel{Intro}
\UnaryInfC{$\Gamma | \cdot \vdash_c (\lambda x : A . V): A \rightarrow B$}
\end{prooftree}

\begin{prooftree}
\AxiomC{$\Gamma | \Theta \vdash_c M : A \rightarrow B $}
\AxiomC{$\Delta \vdash_v N : A $}
\RightLabel{Elim}
\BinaryInfC{$\Gamma ; \Delta | \Theta \vdash_c M N : B$}
\end{prooftree}

\subsubsection{Separating Function}
\begin{prooftree}
\AxiomC{$\Gamma \fatsemi (x : A) | \cdot \vdash_c V : B$}
\RightLabel{Intro}
\UnaryInfC{$\Gamma | \cdot \vdash_c (\alpha x : A . V): A \sep B$}
\end{prooftree}

\begin{prooftree}
\AxiomC{$\Gamma | \Theta \vdash_c M : A \sep B $}
\AxiomC{$\Delta \vdash_v N : A $}
\RightLabel{Elim}
\BinaryInfC{$\Gamma \fatsemi \Delta | \Theta \vdash_c M N : B$}
\end{prooftree}


\subsubsection{Return}
\begin{prooftree}
\AxiomC{$\Gamma \vdash_v V : A$}
\RightLabel{Intro}
\UnaryInfC{$\Gamma | \cdot \vdash_c \ret{V} : FA$}
\end{prooftree}

\begin{prooftree}
\AxiomC{$\Gamma | \Theta \vdash_c M : F A $}
\AxiomC{$\Gamma ; (x : A) | \cdot \vdash_c N : B $}
\RightLabel{Elim}
\BinaryInfC{$\Gamma | \Theta \vdash_c x \leftarrow M ; N : B$}
\end{prooftree}

\subsubsection{New Case}
\begin{prooftree}
\AxiomC{$\Gamma \fatsemi (x : Case A) | \Theta \vdash_c M : B $}
\RightLabel{new}
\UnaryInfC{$\Gamma | \Theta \vdash_c \newcase{A}{x}{M}$}
\end{prooftree}

\subsubsection{Error}
\begin{prooftree}\textsl{}
\AxiomC{}
\RightLabel{Intro}
\UnaryInfC{$\Gamma | \cdot \vdash_c \err : B$}
\end{prooftree}

\subsection{Equations}
\blue{for effects (but also computation rules)}
\subsubsection{Allocation}

The following are from Ian Stark's Categorial Models for Local Names \cite{starkCategoricalModelsLocal1996}
\begin{enumerate}
    \item swap - allocation swap
    \item drop - (\red{our monad might not support this)}
    \item Fresh - (a kind of congruence)
\end{enumerate}
\subsubsection{Dynamic Type}
\blue{the prism laws}
\href{https://hackage.haskell.org/package/lens-5.3.2/docs/Control-Lens-Prism.html}{here}
\begin{enumerate}
    \item 
\end{enumerate}

\begin{comment}

\begin{prooftree}
\AxiomC{$\Gamma \vdash $}
\RightLabel{}
\UnaryInfC{$\Gamma \vdash$}
\end{prooftree}

\begin{prooftree}
\AxiomC{$\Gamma \vdash $}
\AxiomC{$\Gamma \vdash $}
\RightLabel{}
\BinaryInfC{$\Gamma \vdash$}
\end{prooftree}

\begin{prooftree}
\AxiomC{$\Gamma \vdash $}
\AxiomC{$\Gamma \vdash $}
\AxiomC{$\Gamma \vdash $}
\RightLabel{}
\TrinaryInfC{$\Gamma \vdash$}
\end{prooftree}

\end{comment}
\section{Semantics}
\subsection{Worlds}
Normally, the value types in a CBPV would be interpreted as Sets. However, in our case this is insufficient as 
the interpretation of the extensible sum type, OSum, is dependent on what Cases have been "allocated" via 
newcase. Thus we interpret value types in a presheaf category on a category of worlds where objects are a map 
from a finite set to syntactic types. Let SynTy be a set of syntactic types that can be used in the OSum Case store.

\begin{figure}[!ht]
\centering    
\begin{bnf}
$\textrm{SynTy}$ ::= $\textrm{Unit}$
| $\textrm{Bool}$
| $\textrm{SynTy} \times \textrm{SynTy}$
;;
\end{bnf}
\end{figure}

The denotation of these syntactic types is independent of what Cases have been allocated.
Limiting the extensible sum type to contain only base types and products of base types allows 
us to temporarily avoid two separate issues. 
\begin{itemize}
    \item The first issue has to do with circularity of definitions. If we say a world is a 
    mapping from a fininte set to semantic types and that semantic types are presheaves on 
    the category of worlds, then we have tied ourselves in a knot. This issue is pointed 
    out in, among other places, \cite{sterling_denotational_2023}.
    \begin{align*}
        \textrm{world} \cong X \rightarrow \textrm{SemTy} \;\;\;\; \textrm{SemTy} \cong [\world , \mathbf{Set}]\\
    \end{align*}
    
    \item The second issue is well-foundedness of the interpretation of OSum. Consider if we allow the extensible 
    sum type to store the type ($\textrm{OSum} \rightarrow \textrm{OSum}$) and try to recursively interpret $\textrm{OSum}$.

\end{itemize}


 Here we define the category $\world$\footnote{Similar categories: $HeapV$ \cite{SIMPSON-Independence}, 
 $IFam_X$ \cite{sterlingFreeTheoremsUnivalent}, polymorphic algebraic theories \cite{fioreMultiversalPolymorphicAlgebraic2013},
 \cite{sterlingSyntaxSemanticsAbstract2016}}
 \footnote{Formalized 
 \href{https://github.com/bond15/Bunched-CBPV/blob/68f8b4d006edc9df1d830d7f9cc63822ec77a379/src/Data/Worlds.agda#L33}{here} 
 as $\world := (Inc \downarrow \textrm{SynTy})^{op}$, where $Inc$ is the inclusion functor from the category of fininte sets 
 and injections into the category $\mathbf{Set}$ and SynTy is a constant functor.}. 
 A world $: X \rightarrow \testrm{SynTy}$ is a map from a finite set into the set of syntactic types. Morphisms $f : X \rightarrow Y$ 
 in $\world$ are injective functions $i : Y \hookrightarrow X$ such that the 
 following triangle commutes. The finite sets here represent a finite 
 collection of case symbols and the injective map $i$ represents \textit{renaming}. We may denote a world $X \xrightarrow{w} \textrm{SynTy}$ 
 by $w$.
 

\begin{figure}[!ht]
    \centering
    \begin{tikzcd}
X \arrow[rdd, "w"'] &       & Y \arrow[ldd, "w'"] \arrow[ll, "i"', hook'] \\
              &       &                                \\
              & \textrm{SynTy} &                               
\end{tikzcd}
\caption{Morphisms in $\world$}
\end{figure}
 
 
\begin{comment}
Previously, I had maps between worlds as injections where the source map was 
included in the domain map. I've flipped the ordering to make the DCC obviously 
affine(monoidal unit is isomorphic to cartesian unit).    
\end{comment}



  \subsubsection{Monoidal Structure}

  We need to introduce additional structure on $\world$ to interpret the 
  separating connectives $A * B$ and $A\sep B$. For this, we will pull 
  from existing literature on categorical semantics of bunched typing, 
  specifically \cite{ohearnResourceInterpretationsBunched1999} and section 3.1 of \cite{pym_semantics_2002}. 
  We use the Day convolution \cite{nlabDayConv} to get a monoidal structure on 
  $\pworld$\footnote{$\mathbf{Set}^{(\world^{op})}$; contravariant presheaves on $\world$} 
  from a monoidal structure on $\world$. First, we define an operation $w \otimes_w w'$ on $worlds$ by taking 
  the set coproduct of their domains\footnote{This is not the coproduct in FinSet\_mono \cite{SE-FinSetMonoCoprod}} 
  and using the recursor $X + Y \xrightarrow{rec \; w \; w'} SynTy$ 
  to define a map into syntactic types. The action on morphisms 
  is given by defining the top injection via $map : (A \rightarrow C) \rightarrow 
  (B \rightarrow D) \rightarrow A + B \rightarrow C + D$ and the commuting triangles still hold. 
  The unit for this monoid structure is the empty map $\lambda()$ 
  \footnote{The empty map is also the terminal object in $\mathbf{Worlds}$}.

\begin{figure}[!ht]
    \centering
    \begin{tikzcd}
        W + Z \arrow[rdd, "rec\; w_3 \; w_4"'] &       & X + Y \arrow[ldd, "rec\; w_1 \;w_2"] \arrow[ll, "map \; i \; j"', hook'] \\
                                               &       &                                                                         \\
                                               & SynTy &                                                                        
        \end{tikzcd}
    \caption{$\otimes_w$ action on morphisms}
\end{figure}


Using this operation, we define the monoidal product for $\pworld$. The action on objects being: 
\[
(A \otimes B) X = \int_{}^{Y,Z}\ A(Y) \times B(Z) \times \world[X,Y \otimes_w Z]
\]
The monoidal unit is given by: 
\[ 
    I(X) = \mathbf{World}[X , \lambda()]
\]
The separating function is defined by: \footnote{Z interpreted as a set}
\[
(A \sep B) X = \mathbf{Set}[ |Z| , B(X \otimes_w Z)^{A(Z)}] 
\]
\begin{comment}
Finally, we have that:
\[
\textrm{Hom}(A \otimes B , C) \cong \textrm{Hom}(A , B \sep C) 
\]
\end{comment}

These definitions are standard in the literature on bunched implication with the 
exception that our \textit{resources} are finite maps specifying typing of allocated
case symbols. The bicartesian stucture on $\pworld$ along with 
the monoidal structure ($\otimes$) and its right adjoint ($\sep$) bundled together 
gives us the structure of a bicartesian doubly closed category or BiDCC. 

\subsubsection{Affine BiDCC}
The concrete model that we are working with has the property 
that $\mathbf{1} \cong I$ \footnote{See section 3.3 of \cite{pym_semantics_2002}}. 
That is, the cartesian unit is isomorphic to the monoidal unit. 
This property validates weakening for $\fatsemi$. Note that:

\[ 
    \mathbf{1}(X) = \{*\} \\
    I(X) = \world[X , \lambda()]
\]

$\lambda()$ is terminal in $\world$ and the cardinality of $|\world[\_\;,\lambda()]| \;= 1$.
\\
\subsubsection{Concrete Separating Connectives}
It may be possible for us to have an alternative interpretation of the separating product if we work in a 
full subcategory of $\pworld$. O'Hearn has affine models\cite{ohearnResourceInterpretationsBunched1999}
\cite{ohearn_bunched_2003} for his $\alpha\lambda\calculus$ \footnote{And an extension with commands, 
memory cells, and assignment called he calls SCI+ which is based on Reynold's Syntactic Control of Interference} 
using a presheaf category on $\mathbf{FinSet}_{\textrm{mono}}$. He shows that(section 4 , \cite{ohearnModelSyntacticControl1993}}) 
the full subcategory of $\mathbf{Set}^{\mathbf{FinSet}_{\textrm{mono}}^{op}}$ which consists of pullback preserving 
functors lets you ask for the \textit{smallest} world that any $a \in A(X)$ comes form, aka $support(a)$(Section 9.2,
 \cite{ohearn_bunched_2003}). With this notion of support, you can define non-interference as:
\[ 
    a \# b \iff (support(a) \cap support(b) = \emptyset)
\]
With non-interference, $A * B$ can be directly interpreted as:
\[ 
    (A * B) X = \{ (a , b) \in A(X) \times B(X) | a \# b\}\\
    (A * B) f (a , b) = (A(f)(a), B(f)(b))
\]
It should be noted that the full subcategory mentioned above is equivalent to the Shanuel topos (Remark 6.9 \cite{Pitts_2013}).
It seems this would extend to our \textit{multisorted} scenario\footnote{The Schanuel topos models classical HOL, is this problematic?} 
\cite{sterlingSyntaxSemanticsAbstract2016}\cite{gabbayNewApproachAbstract2002}. See Biering's masters thesis 
\cite{bieringLogicBunchedImplications}(section 4.4) for a note on when the day tensor does have a right adjoint in the subcategory of sheaves.

\subsection{Allocation Monad}
To model allocation of case symbols, we can use a modified version of the indexed state monad \cite{plotkinNotionsComputationDetermine2001}
\cite{CBPV-Book}. This monad is a special case of a general construction\footnote{https://ncatlab.org/nlab/show/geometric+morphism#BetweenPresheafToposes}.
Given a functor $F : C \rightarrow D$ between two categories, it induces a functor on presheaf categories $F^* : [D^{op} , Set] 
\rightarrow [C^{op} , Set]$ via precomposition of $F$ which has both a left and right adjoint given by Kan extensions.

\begin{figure}[!ht]
    \centering
    \begin{tikzcd}
        {[D^{op},Set]} \arrow[rr, "F^*" description] &  & {[C^{op},Set]} 
        \arrow[ll, "\Sigma" description, bend right, shift right=3] 
        \arrow[ll, "\forall" description, bend left, shift left=3] 
        \arrow[ll, "\dashv" rotate=-90, phantom, bend left, shift right] 
        \arrow[ll, "\dashv" rotate=-90, phantom, bend right, shift left]
    \end{tikzcd}
    \caption{essential geometric morphism between presheaf topoi}
\end{figure}

We instantiate this construction twice
\footnote{code \href{https://github.com/bond15/Bunched-CBPV/blob/899a80968c055f086069b409b1f85ffb5a9d9aa5/src/Models/FuturePast.agda#L35}{here}}
, using the inclusion functor from the discrete category $Inc : \lvert\world\rvert \rightarrow \world$ 
and its opposite $Inc^{op}$. This yields a monad on our value category, contravariant presheaves on $\world$, by composing the top maps with
the bottom maps\footnote{The usual presentation of the indexed state monad has covariant presheaves for the value category. ours is flipped 
to accomodate the definition of $\world$}. 

\begin{figure}[!ht]
    \centering
    \begin{tikzcd}
        {[\world^{op},Set]} 
        \arrow[rr, "Inc^{*op}" description, bend left] &  & {[\lvert\world\rvert,Set]} 
        \arrow[ll, "\forall", bend left, shift left] 
        \arrow[ll, "\dashv" rotate=-90, phantom] 
        \arrow[rr, "\Sigma" description, bend left] &  & {[\world,Set]} 
        \arrow[ll, "Inc^*", bend left] 
        \arrow[ll, "\dashv" rotate=-90, phantom]
    \end{tikzcd}
    \caption{our allocation monad}
\end{figure}
The concrete interpretation for this monad's action on some object $X$ at some world $w_1$ is: 
forall future worlds $w_2$ reachable by $w_1$, there exists some yet future world $w_3$ reachable from $w_2$ with a value $X(w_3)$.
\[
    T(X)(w_1) = \forall_{w_2}\forall_{w_1 \rightarrow w_2}\Sigma_{w_3}\Sigma_{w_2 \rightarrow w_3}X(w_3)
\]

\subsubsection{Limitations}
There are plenty of limitations with our current approach to modeling the dynamic type. Leveraging the analogy with state
\footnote{Using Jon's breakdown \cite{sterlingFreeTheoremsUnivalent}}, our model supports \textit{dynamic} allocation of cases but only 
for \textit{ground} types and allocation is \textit{globally} observable. We can adapt Sterling's model\cite{sterling_denotational_2023}
to store semantic types instead of syntactic types. Kammar has a candidate monad for local store\cite{kammarMonadFullGround2017}. 
It is unclear to me if local store with rich semantic types exists yet.

\subsection{Interpreting Types}
Taking contravariant presheaves on $\world$ as our value category and covariant presheaves on $\world$ as our computation category,
we can give a denotation of the syntax. Interpretation of syntactic types $\llbracket\_\rrbracket_{el} : \textrm{SynTy} \rightarrow \pworld$ 
is not dependent on what cases have been allocated. $\mathbf{1}$ is the terminal presheaf in $\pworld$ and $+,\times$ come from the 
bicartesian structure of $\pworld$.
\begin{align*}
    \llbracket Unit \rrbracket_{el} &= \mathbf{1}\\
    \llbracket Bool \rrbracket_{el} &= \mathbf{1}  + \mathbf{1}\\
    \llbracket A \times B \rrbracket_{el} &= \llbracket A \rrbracket_{el} \times \llbracket B \rrbracket_{el} \\
\end{align*}
Next, we interpret the remaining value types in $\pworld$. The main definitions of interest are the interpretation of OSum and Case. 
OSum is an extensible sum type where the current world determines the number and type of cases in the extensible sum. 
Case $A$ is interpreted to be the set of allocated case symbols for type $A$ in the current world. Note that syntactic type equality
\footnote{Although, I currently use the 
\href{https://github.com/bond15/Bunched-CBPV/blob/1538ee3b3e1da806ec44bf635b35a5284ebb4924/src/Models/FuturePast.agda#L201}{usual path type} 
in the Agda development. We might want Sterling's univalent reference types \cite{sterlingFreeTheoremsUnivalent}} is used in the set comprehension. 

\begin{comment}
    

This means that $\sigma : \testrm{Case} (A \times (B \times C))$ and $\sigma' : \textrm{Case}((A \times B) \times C)$ are distinct elements 
of two distinct sets. 
It may be nice to have a more flexible definition 
$\{ \sigma | \sigma \in dom(\rho) \land \llbracket \rho(\sigma) \rrbracket_{el} \cong \llbracket A \rrbracket_{el}\}$. 
We would have to be more careful when constructing and destructing elements of $OSum$. 
Consider $\sigma : Case \;A$ and $\injj{\sigma}{(*,a)}: OSum$. 
Should this be allowed with $\llbracket A \rrbracket \cong \llbracket Unit \times A \rrbracket$?
\end{comment}

\begin{align*}
    \llbracket Unit \rrbracket_v &= \llbracket Unit \rrbracket_{el}\\
    \llbracket Bool \rrbracket_v &= \llbracket Bool \rrbracket_{el}\\
    \llbracket A \times B \rrbracket_v &= \llbracket A \times B \rrbracket_{el}\\
    \llbracket OSum \rrbracket_v (w)&= \sum_{\sigma \in dom(w)} \llbracket w(\sigma) \rrbracket_{el}(w)\\
    \llbracket Case \; A \rrbracket_v(w) &= \{ \sigma | \sigma \in dom(w) \land w(\sigma)= A \} \\
    \llbracket A * B \rrbracket_v &= \llbracket A \rrbracket_v \otimes \llbracket B \rrbracket_v \\
    \llbracket U  \underline{B} \rrbracket_v &= \; U \llbracket B \rrbracket_c 
\end{align*}
Lastly, we interpret the computation types\footnote{ The denotation for the regular function type is somewhat surprising. 
See section 7.1 of Levy's thesis (or 6.7.3 of the CBPV book). Levy's model for cell generation has:
\[
    \den{A \rightarrow B}_c(w)= \mathbf{Set} [ \den{A}_v(w) , \den{B}_c(w) ]
\]
That is, the denotation of the computational function type is given pointwise where $A$ is a covariant presheaf, and $B$ is a
contravariant presheaf and the result should be a contravariant presheaf(stuff is oped in our case). 
This functor does work in our situation. However, from O'Hearn's affine $\alpha\lambda\calculus$, we would expect this to also work:
\[
    \den{A \rightarrow B}(w)= \mathbf{Set^{\world}}[ \den{A}_v(w \otimes_w \_) , \den{B}_c(w \otimes_w \_)]
\]
The situation is now more complicated because those monoidal operations exist in two different categories!
(need to check this!) Additionally, this doesn't type check since the homset is taking objects from different categories.
For the separating function, we should expect
\[
    \den{A \sep B}(w)= \mathbf{Set^{\world}}[ \den{A}_v , \den{B}_c(w \otimes_w \_)]
\]
We need to bring another world into scope to fix the typing issues.
\[
    \den{A \sep B}(w)= \forall_{w'} \mathbf{Set}[ \den{A}_v(w') , \den{B}_c(w \otimes_w w') ]
\]
}

\begin{align*}
    \den{A \rightarrow B }_c(w) &= \mathbf{Set}[ \den{A}_v(w) , \den{B}_c(w) ]\\
    \den{A \sep B}_c(w) &= \forall_{w'} \mathbf{Set}[ \den{A}_v(w') , \den{B}_c(w \otimes_w w') ]\\
    \den{F A}_c &= F\den{A}_v\\
\end{align*}
\red{No, this isn't quite correct because of a variance issue. Consider the introduction rule for $\sep$ 
and its inverse. 
\[
    intro : \den{\Gamma \otimes A \vdash_c B} \rightarrow \den{\Gamma \vdash_c A \sep B}\\
    intro (M)(w_1)(\gamma : \den{\Gamma}_0)(w_1)(w_2)(a : \den{A}_0(w_2)) = M(w_1 \otimes w_2)(id_{w_1\otimes w_2},\gamma , a)\\
    \\
    inv : \den{\Gamma \vdash_c A \sep B} \rightarrow \den{\Gamma \otimes A \vdash_c B} \\
    inv (M)(w_1)(w_2,w_3,f : w_1 \rightarrow w_2 \otimes w_3 , \gamma : \den{\Gamma}_0(w_2), a : \den{A}_0(w_3)) = \den{B}_1(f)(M(w_2)(\gamma)(w_3)(a))
\]}

\subsection{Interpreting Contexts}
In \textit{regular} type theory, contexts have the structure of a list and are interpreted as an n-ary product in the category used 
to interpret types. In the bunched type theory we are working with here, there are two connectives for constructing contexts 
(";" and "$\fatsemi$") and the structure of contexts are \textit{tree-like}. We use $;$ to denote the usual context construction operation 
that is interpreted as a cartesian product and $\fatsemi$ to denote the new context construction operation that is interpreted as the 
tensor product.

\begin{align*}
    \llbracket x : A \rrbracket_{ctx} &= \llbracket A \rrbracket_v \\
    \llbracket \emptyset \rrbracket_{ctx} &= \mathbf{1} \\
    \llbracket \Gamma \;; \Gamma \rrbracket_{ctx} &=  \llbracket \Gamma \rrbracket_{ctx} \times \llbracket \Gamma \rrbracket_{ctx}\\
    \llbracket \Gamma \;\fatsemi \Gamma \rrbracket_{ctx} &=  \llbracket \Gamma \rrbracket_{ctx} \otimes \llbracket \Gamma \rrbracket_{ctx} \\
\end{align*}

The cartesian fragment of the context supports weakening and contraction. The monoidal fragment supports weakening
\footnote{consequence of $\mathbf{1} \cong I$ in our model}.


\subsection{Interpreting Terms}
Value terms $\Gamma \vdash M : A$ are interpreted as morphisms 
$\llbracket \Gamma \rrbracket_{ctx} \xrightarrow{M} \llbracket A \rrbracket_{v}$ in $\pworld$ which are natural transformations 
from $\mathbf{World^{op}}$ to $\mathbf{Set}$. So for any term $\Gamma \vdash M : A$, we need to provide a family of maps 
$\forall w ,\; \alpha_w : \llbracket \Gamma \rrbracket(w) \rightarrow \llbracket A \rrbracket(w)$ such that given any morphism 
$f : w \rightarrow w'$ in $\mathbf{World^{op}}$ we have:
\begin{figure}[!ht]
    \centering
    \begin{tikzcd}
        \llbracket \Gamma \rrbracket(w) 
        \arrow[dd, "\alpha_w"] 
        \arrow[rr, "\llbracket \Gamma \rrbracket(f)"'] &  & 
        \llbracket \Gamma \rrbracket(w') 
        \arrow[dd, "\alpha_{w'}"] \\ &  & \\
        \llbracket A \rrbracket(w) 
        \arrow[rr, "\llbracket A \rrbracket(f)"'] &  & 
        \llbracket A \rrbracket(w')                               
        \end{tikzcd}

\end{figure}\\
\subsubsection{Value Terms}
Here we define the terms as natural transformations by giving the components. $w$ is the current world and 
$\gamma : \llbracket \Gamma \rrbracket(w)$. 
\begin{align*}
    \llbracket  \ttt \rrbracket_{vtm}(w, \gamma) &= *\\
    \llbracket  \true \rrbracket_{vtm}(w,\gamma) &= inl \;*\\
    \llbracket  \false \rrbracket_{vtm}(w,\gamma) &= inr \;*\\
\end{align*}

Extracting a variable from the context is a matter of using the correct cartesian and monoidal projections
\footnote{monoidal projections given by a bilinear map $! \otimes id$ composed with the monoidal identity laws 
\cite{JACOBS199473}(Definition 2.1.ii)}. We can use this process (call it $lookup$) to construct the proper 
denotation for variables.

\begin{align*}
    \llbracket  x \rrbracket_{vtm} &= lookup(x) \\
\end{align*}

For injection of values into the open sum type, we can assume we already have 
$\den{\Gamma} \xrightarrow{\sigma} \den{\textrm{Case} \; A}$ and $\den{\Gamma} \xrightarrow{V} \den{A}$.
 We then want to construct $\den{\injj{\sigma}{V}} : \den{\Gamma} \rightarrow \den{\textrm{OSum}}$. 
It suffices to construct $\den{\textrm{Case} A \times A} \rightarrow \den{\textrm{OSum}}$
\footnote{It is morally $\textrm{transport} \; p \; a$ in the 
\href{https://github.com/bond15/Bunched-CBPV/blob/a2da10ec10f7bedcce8ded4aea6646b3a184d0b4/src/Models/FuturePast.agda#L264}
{formalization} }.

\begin{align*}
    \llbracket  \injj{\sigma}{V}\rrbracket_{vtm}(w, ((x , p : w(x)=A), a)) &= (x , a )\\
\end{align*}

The denotation of product and separating product are both akin to contructing bilinear maps.
\begin{align*}
    \llbracket  (V_1, V_2)\rrbracket_{vtm} &= \den{V_1} , \den{V_2}\\
    \llbracket  (V_1 * V_2)\rrbracket_{vtm} &= \den{V_1} * \den{V_2}\\
\end{align*}
Projection maps are just postcomposing with the correct projections
\begin{align*}
    \den{\pi_1 V} &= \den{\Gamma}_v \xrightarrow{\den{V}_v} \den{A_1 \times A_2} \xrightarrow{\pi_1} \den{A_1}\\
    \den{\pi_2 V} &= \den{\Gamma}_v \xrightarrow{\den{V}_v} \den{A_1 \times A_2} \xrightarrow{\pi_2} \den{A_2}\\
    \den{\rho_1 V} &= \den{\Gamma \fatsemi \Delta}_v \xrightarrow{\den{V}_v} \den{A_1 * A_2} \xrightarrow{\rho_1} \den{A_1}\\
    \den{\rho_1 V} &= \den{\Gamma \fatsemi \Delta}_v \xrightarrow{\den{V}_v} \den{A_1 * A_2} \xrightarrow{\rho_2} \den{A_2}\\
\end{align*}

Before we give the interpretation of $\thunk{M}$, we need to discuss the interpretation of computation judgments.

\subsubsection{Computation Terms}
Following Levy's model
\footnote{The computation judgments in Levy's model are not required to satisfy naturality. The problem is the store type, 
which we don't have (section 6.7.2 paragraph 2 \cite{CBPV-Book}), that is not covariant or 
contravariant in world morphisms.}
, the denotation of computation judgments are not morphisms in the computation category. 
Instead, they are denoted as a family of set maps:
\begin{align*}
    \den{\Gamma | \Theta \vdash_c M : B} &= \forall_w \mathbf{Set}[ \den{\Gamma}(w) , \den{\textit{B}}(w)]
\end{align*}
\blue{I'm not satisfied with this. One alternative that typechecks is $\mathbf{Set}^{\world}[ F \den{\Gamma}_v , \den{B}_c]$,
but this is unnatural.}


And finally, the interpretation of $\thunk{M}$ given $\den{M}(w)= \den{\Gamma}(w) \rightarrow \den{B}(w)$.
\[
    \den{\thunk{M}}_{vtm}(w)(\gamma : \den{\Gamma}(w)) = \lambda(w')(f : w' \rightarrow w). \den{M}(w)(\den{\Gamma}_1(f)(\gamma))\\
    \den{\force{V}}_{ctm}(w)(\gamma : \den{\Gamma}(w)) = \den{V}_{vtm}(w)(\gamma)(w)(id_w) \\
    \den{\ret{v}}_{ctm}(w)(\gamma : \den{\Gamma}(w)) = (w , id_w , \den{V}_{vtm}(w)(\gamma))\\
    \den{\lambda x : A . M }_{ctm}(w)(\gamma : \den{\Gamma}(w))(a : \den{A}_{vtm}(w)) = \den{M}_{ctm}(w)(\gamma,a)\\
    \den{M \; N}_{ctm}(w)(\gamma : \den{\Gamma}(w) , \delta : \den{\Delta}(w)) = \den{M}_{ctm}(w)(\gamma)(\den{N}_{vtm}(w)(\delta)) 
\]

\begin{align*}
    \llbracket \err \rrbracket_{ctm} &= \red{\textrm{add error to alloc monad}}\\
    \llbracket \force{V} \rrbracket_{ctm} &= \\
    \llbracket \ret{V} \rrbracket_{ctm} &= \\
    \llbracket \bind{x}{M}{N} \rrbracket_{ctm} &= \\
    \llbracket M \; N \rrbracket_{ctm} &= \\
    \llbracket M \at N \rrbracket_{ctm} &= \\
    \llbracket \lambda x : A . M \rrbracket_{ctm} &=  \\
    \llbracket \alpha x : A . M \rrbracket_{ctm} &= \\
    \llbracket \newcase{A}{\sigma}{M} \rrbracket_{ctm} &= \\
    \llbracket \match{V_1}{V_2}{x}{M}{N}\rrbracket_{ctm} &= \\
    \llbracket \ite{V}{M}{N} \rrbracket_{ctm} &= \\
\end{align*}
\red{TODO: notation consistency (ex subscripts on $\den{}$)}

\bibliographystyle{acm}
\bibliography{ref}
\end{document}
