\documentclass{article}
    % General document formatting
    \usepackage[margin=0.7in]{geometry}
    \usepackage[parfill]{parskip}
    \usepackage[utf8]{inputenc}
    \usepackage{xcolor}
    \usepackage{hyperref}
    
    % Related to math
    \usepackage{stmaryrd}
    \usepackage{amsmath,amssymb,amsfonts,amsthm}
    \newcommand{\den}[1]{\llbracket #1 \rrbracket}
    \newcommand{\sep}{\mathrel{-\mkern-6mu*}}
    \newcommand{\blue}[1]{\textcolor{blue}{#1}}
    \newcommand{\red}[1]{\textcolor{red}{#1}}
    \newcommand{\pshC}{\widehat{\mathcal{C}}}
    \newcommand{\pshCC}{\widehat{\mathcal{C \times C}}}
    \newcommand{\dayprod}{\otimes^{Day}}


\begin{document}
\section{Bicartesian Doubly Closed Category}
Given a category $\mathcal{C}$, its presheaf category ($\pshC := [\mathcal{C}^{op}, Set]$)
is bicartesian closed. Given a monoidal category ($\mathcal{C}, \otimes_C , I_C$), 
its presheaf category is bicartesian closed and monoidal closed via the Day convolution. 
The monoidal product is given by:
\[
    (P \dayprod Q)(x) = \int_{}^{y,z} \mathcal{C}[ x , y \otimes_C z ] 
    \times P(y) \times Q(z)
\]

The Day monoidal product has the universal property that any maps out of it are in bijective
correspondence with a family of maps natural in $x$ and $y$ 
\href{https://github.com/bond15/Bunched-CBPV/blob/82136dc6f4f9e4034391877f0d959a6ff1b62dfc/src/Data/BiDCC.agda#L222}
{(Agda)}: 
\footnote{here $\overline{\times}$ is the \textit{external} product}


\[
    \pshC[P \dayprod Q , R] \;\cong \; \pshCC [ P \overline{\times} Q , R \circ \otimes_C ]\; \cong \; \Pi_{x, y : \;ob \;C}\; Set[P(x) \times Q(y) , R(x \otimes_C y)]
\]

The monoidal closed structure is given by: 
\begin{align*}
    (P \sep Q)(X) = \pshC[P , Q(X , -)]
\end{align*}
With the universal property that the closed structure is right adjoint to the tensor
\href{https://github.com/bond15/Bunched-CBPV/blob/82136dc6f4f9e4034391877f0d959a6ff1b62dfc/src/Data/BiDCC.agda#L354}{(Agda)}: 
\begin{align}
    \pshC [ A \otimes_C B , C] \cong \pshC [ A , B \sep C]
\end{align}

Bicartesian doubly closed categories have been used in the denotational semantics of bunched type theories 
\cite{pym_semantics_2002}\cite{bieringLogicBunchedImplications}\cite{ohearn_bunched_2003}.

\section{Towards Bunched Call By Push Value with Dynamic Store}
Categorical models of dynamic store use presheaf categories to model the dependence of the heap structure on a current \textit{world}
\cite{CBPVbook}\cite{sterling_denotational_2023}\cite{kammarMonadFullGround2017}. Seemingly none of these existing models 
attempt to combine a call by push value language with the separating type connectives, $\otimes$ and $\sep$, used in bunched type theories.
Our investigation into possible models of such a language have run into some potential issues. To illustrate this, we will start with the 
model for a call by push value language with dynamic store presented in chapter 7 of Levy's thesis.

\subsection{Definitions}
Let ($C, \otimes_C , I_C$) be a monoidal category, the value category be $\mathcal{V} := [C , Set]$, computation category $\mathcal{C} := [C^{op}, Set]$, and use the \textit{standard}
monad for ground dynamic store with $F : \mathcal{V} \rightarrow \mathcal{C}$ as:
\[
    F(A)(x) := \Sigma_{y : ob \;C}\Sigma_{f : C[ x , y ]}A(y)    
\]
and $U : \mathcal{C} \rightarrow \mathcal{V}$ as :
\[
    U(\underline{B})(x) := \Pi_{y : ob \; C}\Pi_{f : C [ x , y]}\underline{B}(y)   
\]

The oblique morphisms in this model are given by families of maps:
\[
    \mathcal{O}[A , \underline{B}] := \Pi_{x : ob \; C} Set[A(x) , \underline{B}(x)]   
\]

we have the following isomorphims:
\[
    \mathcal{V}[A , U(\underline{B})] \cong \mathcal{O}[A , \underline{B}] \cong \mathcal{C}[F(A) , \underline{B}]   
\]
And we can attempt to define a computation separating function by:
\[
    (A \sep \underline{B})(x) := \Pi_{y : ob \;C} Set[A(y), \underline{B}(x \otimes_C y)]
\]
\subsection{Problems with an Abstract Monoidal Category}
Before committing to the category 
of worlds used in Levy's model, we will work with an arbitrary monoidal category ($C, \otimes_C , I_C$). 
\subsubsection{Issue 1: Universal Property of Tensor for Oblique Morphisms}
Let's attempt to show the following:
\[
  \mathcal{O}[P \otimes Q , \underline{R}] \cong \mathcal{O\times}[P \overline{\times} Q , \underline{R} \circ \otimes_C]  
\]
where 
\[
    \mathcal{O\times}[P \overline{\times} Q , \underline{R} \circ \otimes_C] 
    := \Pi_{x , y : ob \;C}Set[P(x)\times Q(y) , \underline{R}(x \otimes_C y)]
\]

A problem arises when trying to define the backwards map of this isomorphims. 
Given $m : \mathcal{O\times}[P \overline{\times} Q , \underline{R} \circ \otimes_C]$ and $x : ob \; C$, we need to define a map
$Set[(P \otimes Q)(x), \underline{R}(x)]$. This is a map out of a coequalizer 
\footnote{since coends in $Set$ can be encoded as coequalizers}
which we can attempt to give as a map induced from:
\[
    (f : y\otimes_C z \rightarrow x , p : P(y) , q : Q(z)) \mapsto \;? : \underline{R}(x)
\]
However, using the data we currently have, we can only construct
\[
    m(y)(z)(p,q) : \underline{R} (y \otimes_C z)    
\]
and since $\underline{R}$ is contravariant, we can't use $\underline{R}(f) : \underline{R}(x) \rightarrow \underline{R}(y \otimes_C z)$.
This is not surprising since the proof of this universal property in the value category 
$\mathcal{V}[P \otimes Q , R] \cong \mathcal{V\times}[P \overline{\times} Q , R \circ \otimes_C]$
uses the functorial action of $R$ on $f$ 
(see \href{https://github.com/bond15/Bunched-CBPV/blob/d4de5ebe3a2a42499b24c13a8d2da7f3a2cc1b36/src/Data/BiDCC.agda#L120}{here})
\footnote{note the difference in variance is due to the fact this proof is for presheaves and not covariant presheaves}
So by swapping the variance of $R$ (now $\underline{R}$ since it is from the computation category) this proof should break.
Seemingly, this proof won't go through when we assume a generic monoidal category $C$. Perhaps we can recover this property if 
we work with a specific concrete category?

\subsubsection{Issue 2: Universal Property of the Separating Function Type}



\section{Issue}
There is a variance issue when trying to add a \textbf{computational} separating 
function type to Levy's dynamic store model\cite{CBPVbook}
\footnote{Chapter 6}\footnote{The following issue exists in our setup too.}. Take the category 
of worlds to be $\mathcal{W} := FinSet_{mono}$, the value category to be $\mathcal{V} :=  [\mathcal{W}, Set]$ 
and the computation category to be $\mathcal{C} := [\mathcal{W}^{op} , Set]$. 
Value judgments $\Gamma \vdash_v M : A$ are denoted as morphisms in $\mathcal{V}$.
Computation judgments $\Gamma \vdash_c M : B$ are denoted as families of maps $\forall(w : ob \;W) 
\rightarrow Set[ \den{\Gamma}(w) , \den{B}(w)]$. Note that we are dropping the storage
part (S) of Levy's monad. The monoidal structure on $\mathcal{W}$ given by disjoint union yields a monoidal
structure on $\mathcal{V}$ via the Day convolution\footnote{\textit{covariant}
Day convolution given by taking the monoidal structure on $\mathcal{W}^{op}$ and
then applying the day convolution}. 

\[
    (A \otimes_D B)_0(w_1) = \int_{}^{w_2,w_3} \mathcal{W}[ w_2 \otimes w_3 , w_1 ] 
    \times A(w_2) \times B(w_3)
\]

The separating function in the \textbf{value category}($A , B : ob \;\mathcal{V}$) is given by:
\[
    (A \sep B)_0(w) = \mathcal{V}[ \den{A} , \den{B}(w \otimes \_)]
\]
And we have that:
\begin{align}
    \mathcal{V} [ A \otimes_D B , C] \cong \mathcal{V} [ A , B \sep C]
\end{align}
The \textbf{computational} function type ($A : ob \;\mathcal{V} , B : ob \;\mathcal{C}$) is given by:
\[ 
    (A \rightarrow B)_0(w) = Set[ \den{A}(w) , \den{B}(w)]
\]
We can try to define the \textbf{computational} separating function ($A : ob \;\mathcal{V} , B : ob \;\mathcal{C}$) as :
\[
    (A \sep B)_0(w) = \forall(w' : ob \; W) \rightarrow Set[ \den{A}(w'), \den{B}(w \otimes w')] 
\]
which is a contravariant functor. We should expect the following isomorpism of types(in Set?):
\[
    (A \otimes_D B) \rightarrow C \cong A \rightarrow B \sep C  
\]
given by:
\begin{align*}
    &fun : ((A \otimes_D B) \rightarrow C) \rightarrow (A \rightarrow B \sep C )\\
    &fun \; M \; w_1 \;(a : \den{A}(w_1)) \; w_2 \;(b : \den{B}(w_2)) = M (w_1 \otimes w_2)(id_{w_1 \otimes w_2} , a , b)\\
    \\
    &inv :(A \rightarrow B \sep C ) \rightarrow ((A \otimes_D B) \rightarrow C)\\
    &inv \; M \; w_1 \; (w_2,w_3,f : w_2 \otimes w_3 \rightarrow w_1, a : \den{A}(w_2), b : \den{B}(w_3)) = \red{\den{B}_1(f)}(M \; w_2 \; a \; w_3 \; b)
\end{align*}
However, the variance of $\den{B}$ gives us $\den{B}_1(f) : \den{B}(w_1) \rightarrow \den{B}(w_2 \otimes w_3)$ 
which is the opposite direction that \textit{we want}\footnote{Meaning this is how the isomorpism goes in (1)}.
\subsection{Our Model}
I was able to derive an \textit{inverse} (likely not able to show the isomorphism) in our model, 
but it felt like a hack and involves an \red{arbitrary choice}. Without reproducing all the details here, 
the gist is the following:

\begin{align*}
    &s2p : \mathcal{V}[ A \otimes_D B , A \times B]    \\
    &s2p (w_1)(w_2,w_3,f : w_2 \otimes w_3 \hookrightarrow w_1, a , b) = \den{A}_1(inl\; ; f )(a) , \den{B}_1(inr \; ; f)(b)\\
    \\
    &inv :(A \rightarrow B \sep C ) \rightarrow ((A \otimes_D B) \rightarrow C)\\
    &inv \; M \; w \; s = \den{B}_1(\red{inl \; or \; inr})(M \; w \; (\pi_1 \; p)\; w \; (\pi_2 \;p)) \\ 
    & \;\;\;\; where \\
    & \;\;\;\;\;\;\;\; p : \den{A \times B}(w)\\
    & \;\;\;\;\;\;\;\; p = s2p \; w \; s
\end{align*}

\section{A possible way forward}
I'm starting to look at a weaker version of the setup in section 2.4 of \cite{sterling_denotational_2023} 
which is a model of $\textrm{SystemF}_{\mu}^{ref}$. I think we had already worked out the 
computational separating function for an algebra model of CBPV. 

\bibliographystyle{acm}
\bibliography{ref}
\end{document}