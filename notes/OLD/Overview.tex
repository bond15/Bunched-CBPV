\documentclass{article}
\usepackage{graphicx} % Required for inserting images
\graphicspath{ {./images/} }
\usepackage{simplebnf}
\usepackage{bussproofs}
\usepackage{amsmath}
\usepackage{amssymb}
\usepackage[dvipsnames]{xcolor}
\usepackage{stmaryrd}
\usepackage{comment}
\usepackage{bm}
\usepackage{amsthm}

%used in colored inline comments
\usepackage[dvipsnames]{xcolor}

\usepackage[utf8]{inputenc}
\usepackage[english]{babel}
%using biblatex https://www.overleaf.com/learn/latex/Bibliography_management_with_biblatex
\usepackage[
    backend=biber,
    style=trad-abbrv,
  ]{biblatex}
\addbibresource{ref-overview.bib}


\newtheorem{thm}{Theorem}
\newtheorem{deff}{Definition}

\newtheorem{lem}[thm]{Lemma}
\newtheorem{cor}[thm]{Corollary}
\newtheorem{rem}[thm]{Remark}
\newtheorem*{remark}{Remark}
\newtheorem*{question}{\red{Question}}
\newtheorem{definition}[deff]{Definition}
\newtheorem*{construction}{Construction}
\newtheorem{conj}[thm]{Conjecture}

% why isn't this resoled by \includin{commands} ?
\newcommand{\blue}[1]{\textcolor{blue}{#1}}
\newcommand{\red}[1]{\textcolor{red}{#1}}



\begin{document}
%\usepackage{xcolor}

\newcommand{\den}[1]{\llbracket #1 \rrbracket}
\newcommand{\blue}[1]{\textcolor{blue}{#1}}
\newcommand{\red}[1]{\textcolor{red}{#1}}
\newcommand{\sep}{\mathrel{-\mkern-6mu*}}
\newcommand{\thunk}[1]{\textrm{thunk }#1}
\newcommand{\injj}[2]{\textrm{inj}_{#1}#2}
\newcommand{\err}{\mho}
\newcommand{\print}[1]{\textrm{print }#1}
\newcommand{\force}[1]{\textrm{force }#1}
\newcommand{\ret}[1]{\textrm{ret }#1}
\newcommand{\bind}[3]{#1 \leftarrow #2 ; #3}
\newcommand{\newcase}[3]{\textrm{newcase}_{#1} \; #2 ; #3}
\newcommand{\match}[5]{\textrm{match }#1 \textrm{ with }#2 \;\{#3 . #4 | #5\}}
\newcommand{\unpack}[4]{\textrm{unpack }(#1,#2) = #3 ; #4}
\newcommand{\lett}[4]{\textrm{let }(#1,#2) = #3 ; #4}
\newcommand{\lets}[4]{\textrm{let }(#1*#2) = #3 ; #4}
\newcommand{\ite}[3]{\textrm{if }#1 \textrm{ then }#2 \textrm{ else }#3}
\newcommand{\at}{\textrm{@}}
\newcommand{\ttt}{\textrm{tt}}
\newcommand{\tru}{\textrm{true}}
\newcommand{\fsl}{\textrm{false}}
\newcommand{\pworld}{\widehat{\mathbf{World}}}
\newcommand{\world}{{\mathbf{World}}}
\newcommand{\calculus}{\operatorname{-calculus}}

\title{Towards a Parametricity Logic for Gradual Type Theory}
\maketitle

\section{Introduction}
It is well known that binary parametricity is a useful principle for 
reasoning about programs in a polymorphic language\cite{Reynolds}\cite{wadler_theorems_1989}.
\red{don't just state, provide elaboration on why it is useful. 
examples of parametricity proofs in different domains (ex security/noninterference)}
Parametricity in the presence of a dynamic type has proven tricky to establish 
and existing definitions break full abstraction relative to System F\cite{TwoParamThreeUniv}.
Even so, we claim that the \textit{Type World Logical Relation} method used to establish parametricity 
in the presence of a dynamic type captures relevant relational reasoning principles\blue{like what?}.
To demonstrate this, we provide a parametricity logic\cite{PlotkinAbadi} to reason about data abstraction.
This logic is sound \blue{relative to our operational semantics?} and the model is constructed in Iris. 



\section{Languages}
\subsection{Poly G}
\cite{SemFoundGradType} page 274.

\subsection{Poly C}
\cite{SemFoundGradType} page 282

\subsection{Poly G to Poly C}
\cite{SemFoundGradType} page 283

\subsection{Poly C to OSum}
\cite{SemFoundGradType} page 292
\red{syntax for OSum}

\red{contract translation?}

\section{Parametricity}

Parametricity is formulated and proved via a family of proof methods titled \textit{logical relations}.
The typical structure of a binary logical relation proof method consists of an inductive assignment of 
types to binary relations on values. 
These relations are meant to capture when two values of a type should be considered related.
The prototypical example of parametricity is given by a Reynolds-style binary logical relation defined on the types of pure System F.

It is well known that parametricity changes in the presence of effects.
To accomidate effects such as higher order store and recursion, 
the logical relation is enhanced to a step indexed Kripke logical relation.
The logical relation used to establish parametricity for a language with a dynamic type
is based off of that of higher order store. In order establish parametricity in 
the presence of a dynamic type....

\subsection{Binary Logical Relation for OSum}
Untyped, using de Bruijn indicies, based on \cite{ALogicalApproachTypeSound}
\begin{align*}
\mathcal{V}\llbracket X\rrbracket\rho&= \rho(X)\\
\mathcal{V}\llbracket Unit\rrbracket\rho&=\{(*,*)\}\\
\mathcal{V}\llbracket Bool\rrbracket\rho&=\{(true,true),(false,false)\}\\
\mathcal{V}\llbracket OSum\rrbracket\rho&=\{(inj_{\sigma_1} v_1, inj_{\sigma_2} v_2) \;|\;
(\sigma_1 , \sigma_2) \mapsto R \;\land\; (v_1,v_2) \in \rhd R  \}\\
\mathcal{V}\llbracket Case \;X\rrbracket\rho&=\{(\sigma_1 , \sigma_2) \; | \; (\sigma_1 , \sigma_2) \mapsto \mathcal{V}\llbracket X \rrbracket \rho \}\\
\mathcal{V}\llbracket X \times Y \rrbracket\rho&= \{((v_1,v_2),(v_3,v_4)) 
\;|\; (v_1,v_3) \in \mathcal{V}\llbracket X \rrbracket \rho
\; \land \; (v_2,v_4) \in \mathcal{V}\llbracket Y \rrbracket \rho\}\\
\mathcal{V}\llbracket X \rightarrow Y \rrbracket\rho&= 
\{ (f,g) \; | \; \square (\forall x y, (x , y) \in \mathcal{V}\llbracket X \rrbracket \rho \implies (f x , g y) \in \mathcal{E}\llbracket Y \rrbracket \rho)\} \\
\mathcal{V}\llbracket \forall X. Y\rrbracket\rho&= 
\{(f , g) \;| \; \square (\forall R, (f[X],g[X]) \in \mathcal{E}\llbracket Y \rrbracket (\rho , X \mapsto R))\}\\
\mathcal{V}\llbracket \exists X. Y\rrbracket\rho&= 
\{ (pack \;v_1 , pack \; v_2) \; | \exists R, (v_1,v_2) \in \mathcal{E}\llbracket Y \rrbracket (\rho, X \mapsto R)\; \}\\
\end{align*}

\subsection{Binary Logical Relation for PolyG}

\section{Logic}

\section{Examples}
By switching from a RLR to a TWLR\cite{TwoParamThreeUniv}, we have lost full abstraction.
But do we care? Do we loose the ability to reason about contextual equivalences we'd like to show?

\subsection{Simple, Pure examples}
swap (see theorem 273 of \cite{SemFoundGradType})
\subsection{Data Abstraction}
\subsection{Universal Properties}
\subsection{Local Store}
\cite{NominalRelReason} \cite{ParamAndLocalVars} An example of relating programs which use different numbers of allocations.
\[
let\; \sigma = new\; \tau; M \cong_{ctx} M  
\]
where $\sigma$ is not used in $M$.

\begin{comment}
\subsubsection{Reynolds-Style Logical Relation}
\blue{Example binary logical relation on system F. 
Threading through a "map" of type variables to relations.}

\subsubsection{Type World Logical Relation}
\blue{Example binary logical relation on system F. 
Using a Kripke World to "hold" the mapping of type variables to relations }

\subsubsection{Reconciling the Difference}

\cite{TwoParamThreeUniv}
\red{Conjecture: The claim that the degeneracy of the universal type can't be shown in TWLR 
seems to suffer from lack of imagination. 
Consider a richer Kripke world in which type variables can be tagged as deallocated 
based on their lexical scope.}

\section{Parametricity Logics}
\subsection{Pure}
\cite{PlotkinAbadi}
\subsection{With Recursion}
\cite{birkedal2006linear}
\cite{LSILR}
\subsection{Effects}
\cite{mogelberg2009relational}
\subsection{Substructural?}
\end{comment}

\printbibliography
\end{document}