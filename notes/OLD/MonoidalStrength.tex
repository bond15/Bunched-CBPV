\documentclass{article}
    % General document formatting
    \usepackage[margin=0.7in]{geometry}
    \usepackage[parfill]{parskip}
    \usepackage[utf8]{inputenc}
    \usepackage{xcolor}
    \usepackage{hyperref}
    \usepackage{tikz-cd}

    % Related to math
    \usepackage{stmaryrd}
    \usepackage{amsmath,amssymb,amsfonts,amsthm}
    \newcommand{\den}[1]{\llbracket #1 \rrbracket}
    \newcommand{\sep}{\mathrel{-\mkern-6mu*}}
    \newcommand{\blue}[1]{\textcolor{blue}{#1}}
    \newcommand{\red}[1]{\textcolor{red}{#1}}
    \newcommand{\pshC}{\widehat{\mathcal{C}}}
    \newcommand{\pshCC}{\widehat{\mathcal{C \times C}}}
    \newcommand{\dayprod}{\otimes^{Day}}


\begin{document}
\section{Model Setup}
\subsection{Bicartesian Doubly Closed Category}
Given a category $\mathcal{C}$, its presheaf category ($\pshC := [\mathcal{C}^{op}, Set]$)
is bicartesian closed. Given a monoidal category ($\mathcal{C}, \otimes_C , I_C$), 
its presheaf category is bicartesian closed and monoidal closed via the Day convolution. 
The monoidal product is given by:
\[
    (P \dayprod Q)(x) = \int_{}^{y,z} \mathcal{C}[ x , y \otimes_C z ] 
    \times P(y) \times Q(z)
\]
With a useful fact:
\begin{align*}
    ^{inc}[h , A(f)(a'), B(g)(b')] = ^{inc}[h ; (f \otimes g)  , a' , b']
\end{align*} 
where $h : C[x , y \otimes z]$, $f : C[y , y']$, $g : C[z , z']$, $a' : A(y')$, and $b' : B(z')$.
The Day monoidal product has the universal property that any maps out of it are in bijective
correspondence with a family of maps natural in $x$ and $y$ 
\href{https://github.com/bond15/Bunched-CBPV/blob/82136dc6f4f9e4034391877f0d959a6ff1b62dfc/src/Data/BiDCC.agda#L222}
{(Agda)}: 
\footnote{here $\overline{\times}$ is the \textit{external} product}


\[
    \pshC[P \dayprod Q , R] \;\cong \; \pshCC [ P \overline{\times} Q , R \circ \otimes_C ]\; \cong \; \Pi_{x, y : \;ob \;C}\; Set[P(x) \times Q(y) , R(x \otimes_C y)]
\]

The monoidal closed structure is given by: 
\begin{align*}
    (P \sep Q)(X) = \pshC[P , Q(X , -)]
\end{align*}
With the universal property that the closed structure is right adjoint to the tensor
\href{https://github.com/bond15/Bunched-CBPV/blob/82136dc6f4f9e4034391877f0d959a6ff1b62dfc/src/Data/BiDCC.agda#L354}{(Agda)}: 
\begin{align}
    \pshC [ A \otimes_C B , C] \cong \pshC [ A , B \sep C]
\end{align}

\subsection{Towards Bunched CBPV with Dynamic Store}
Let ($C, \otimes_C , I_C$) be a monoidal category, the value category be $\mathcal{V} := [C^{op} , Set]$, 
computation category $\mathcal{C} := [C , Set]$, and use the \textit{standard}
monad for ground dynamic store with $F : \mathcal{V} \rightarrow \mathcal{C}$ as:
\[
    F(A)(x) := \Sigma_{y : ob \;C}\Sigma_{f : C^{op}[ x , y ]}A(y)    
\]
and $U : \mathcal{C} \rightarrow \mathcal{V}$ as :
\[
    U(\underline{B})(x) := \Pi_{y : ob \; C}\Pi_{f : C^{op}[ x , y]}\underline{B}(y)   
\]

The oblique morphisms in this model are given by families of maps:
\[
    \mathcal{O}[A , \underline{B}] := \Pi_{x : ob \; C} Set[A(x) , \underline{B}(x)]   
\]

we have the following isomorphims:
\[
    \mathcal{V}[A , U(\underline{B})] \cong \mathcal{O}[A , \underline{B}] \cong \mathcal{C}[F(A) , \underline{B}]   
\]
and a monad on $\mathcal{V}$ via the adjunction between $F$ and $U$.
\[
    T(A)(x) := \Pi_{y : ob \; C}\Pi_{f : C^{op}[ x , y]}\Sigma_{z : ob \;C}\Sigma_{g : C^{op}[ y , z ]}A(z) 
\]
The monadic Unit
\[
    \eta_A(x)(a:A(x)) = \lambda x' . \lambda f : FS[x , x'] . (x' , id_{x'}, A(f)(a))    
\]
\section{Issue with Abstract Model}
It seems we cannot define the strength map:
\[
    str\otimes_{P,Q} : \mathcal{V}[ P \otimes T(Q), T (P \otimes Q)]
\]
via the universal property of tensor, it suffices to construct components of the form:
\[
    Set[P(x) \times T(Q)(y) ,T(P \otimes Q)(x \otimes_C y) ]
\]
introducing terms and unfolding definitions, we have:
\begin{align*}
    &p : P(x)\\
    &q : T(Q)(y) = \Pi_{u : ob \; C}\Pi_{f : C^{op}[ y , u]}\Sigma_{v : ob \;C}\Sigma_{g : C^{op}[ u , v ]}Q(v)\\
    &h : C^{op}[x\otimes_C y , z]
\end{align*}
we get to choose some future world $w$ of $z$, $C^{op}[z , w]$, 
at which we must provide 
\[ 
    ? : (P \otimes Q)(w) = \int_{}^{r,s} \mathcal{C}[ w , r \otimes_C s ] 
\times P(r) \times Q(s)
\]
How to proceed? Notice that in order to \textit{extract} a value $Q(\_)$ from $q$,
we need to provide a morphism $C^{op}[y , u]$ for some $u$. 
Working abstractly with no additional assumptions about $C$, all we can provide is $id_y$ which yields:
\[
    q(y)(id_y) : \Sigma_{v : ob \;C}\Sigma_{g : C^{op}[ y , v ]}Q(v)  
\]
Then our context becomes:
\begin{align*}
    &p : P(x)\\
    &g : C^{op}[ y , v ]\\
    &q' : Q(v)\\
    &h : C^{op}[x\otimes_C y , z]
\end{align*}
We still need to choose some future world $w$ of $z$.
From the data available to us, it seems we are stuck using identity, $id_z$, once again.
This leaves us with the obligation:
\[
    ? : (P \otimes Q)(z) = \int_{}^{r,s} \mathcal{C}[ z , r \otimes_C s ] 
\times P(r) \times Q(s)
\]
The choice of $r$ seems forced to be $x$ for which we have $p: P(x)$.
The natural choices for $s$ would be $v$ or $y$, but it seems we've hit a variance issue yet again.
Observe that \red{if $g: C^{op}[ y , v ]$ was instead $g : C^{op}[v , y]$} we'd have
\[
    ((id \otimes_C g) ;h , p , q') = (h , p , Q(g)(q'))
\]
which are equal relative to the coend quotient.

\section{Concrete Model}
Now we consider substituting the monoidal category ($C, \otimes_C , I_C$) with ($FinSet_{mono}^{op}, \uplus , \emptyset $)
where $\uplus$ is disjoint union of sets. Take $\mathcal{FS}$ as shorthand for $FinSet_{mono}$. 

\subsection{Components}
Define as a bilinear map.


\begin{align*}
    & str_{A, B} : \Pi_{x,y :ob \mathcal{FS}}\; Set[A(x)\times T(B)(y) , T(A\otimes B)(x \uplus y)] \\
    & str_{A, B} (x , y)(a : A(x), tb :T(B)(y))=\lambda z . \lambda f: \mathcal{FS}[x \uplus y , z].\; \red{??} 
\end{align*}

where 
\[
    \red{??}: \Sigma_{w : ob \mathcal{FS}}\Sigma_{g : \mathcal{FS}[z , w]}(A\otimes B)(w)
\]

We recognize that since $f$ is injective and the domain is a disjoint union, $z$ is partitioned into three parts
\begin{align*}
    z_x :& \textrm{ the range of $f_x$, f restricted to x}\\
    z_y :& \textrm{ the range of $f_y$, f restricted to y}\\
    z_m :& \; z - (z_x \uplus z_y)\\
    \textrm{where } & z \cong z_x \uplus z_y \uplus z_m
\end{align*}


Using $f_y : y \rightarrow z_y$
\[
  (v , g : FS[z_y , v], b : B(v)) = tb(zy)(f_y)  
\]

We can construct future world $w := z_x \uplus v \uplus z_m$, and map $z \xrightarrow{z_x \;\uplus\; g \;\uplus\; z_m} w$.
Now we must provide 
\[
    (A \otimes B)(w) = \int_{}^{r,s} \mathcal{FS}[r \uplus s,w] 
    \times A(r) \times B(s)
\]
have $r := z_x$ and $s := v \uplus z_m$.
Then the element at the future world is 
\[
  ^{inc}[ id_w, A(f_x)(a), B(inl)(b)]
\]

\subsection{Naturality}
Given $f : FS[x , x']$, $g: FS[y , y']$, the following maps in $Set$ should commute.
\begin{figure}[!ht]
    \centering
\begin{tikzcd}
    A(x)\times T(B)(y) \arrow[rr, "A(f) \times T(B)(g)"] \arrow[dd, "{Str_{A , B}(x , y)}"] &  & A(x')\times T(B)(y') \arrow[dd, "{Str_{A,B}(x',y')}"] \\
                                                                                            &  &                                                       \\
    T(A \otimes B)(x \uplus y) \arrow[rr, "T(A \otimes B)(f \uplus g)"]                   &  & T(A \otimes B)(x' \uplus y')                        
    \end{tikzcd}
\end{figure}

For $a : A(x),\; tb : T(B)(y),\; z : ob\;FS ,\; h' : FS [ x' \uplus y' , z]$, the naturality condition is:
\[
 Str(a,tb)(z)(f \uplus g ; h') = Str(a',tb')(z)(h')   
\]
where $a' = A(f)(a)$ and $tb' = T(B)(g)(tb) = \lambda y''. \lambda f' : FS [ y' , y'']. tb(y'')(f ; f') = tb(\_)(f ; \_)$.
\paragraph*{Issue}
Note that we relied on the fact that we could split the given morphism in the strength map to construct a future world 
$zy = \textrm{Image}(inr ; x \uplus y \rightarrow z)$ of $y$.
However, it is not necessarily the case that $\textrm{Image}(inr ; f \uplus g ; h') = \textrm{Image}(inr ; h')$.
Thus, the application of $tb$ to two potentially different finite sets has no guarantee that the future world $v$ of $tb$ are comparable.
This implies that the future worlds $w$ of $z$ may not be comparable, which inhibits our ability to compare the results of the strength maps.  

\subsection{Laws}
\subsubsection{Strength with I is Irrelevant}
\subsubsection{Strength Respects Associators}
\subsubsection{Strength Commutes with Monad Unit}
\begin{figure}[!ht]
    \centering
    \begin{tikzcd}
        & A \otimes B \arrow[ld, "id_A \otimes \eta_B"'] \arrow[rd, "\eta_{A\otimes B}"] &                \\
A \otimes T(B) \arrow[rr, "{str_{A,B}}"'] &                                                                                & T(A \otimes B)
\end{tikzcd}
\end{figure}
$\eta_{A\otimes B}$ as a bilinear map computes to:
\begin{align*}
    & \eta' : \Pi_{x,y :ob \mathcal{FS}}\; Set[A(x)\times B(y) , T(A\otimes B)(x \uplus y)] \\
    & \eta' (x , y)(a : A(x), b :B(y)) = \lambda z . \lambda f: \mathcal{FS}[x \uplus y , z].^{inc}[f , a , b]
\end{align*}

The strength map precomposed with $id_A \otimes \eta_B$ yields:
\[
    (z_y , id_{z_y}, b' = B(f_y)(b)) = tb(z_y)(f_y)  
\]
thus, the future worlds are equal $z \cong z_x \uplus z_y \uplus z_m \cong w$ and the maps $z\rightarrow w$ are equal.
It remains to show that
\[
    ^{inc}[id, A(f_x)(a), B(inl)(b')] = ^{inc}[f , a , b]
\]
Recall $ ^{inc}[h , A(f)(a), B(g)(b)] = ^{inc}[(f \uplus g) ; h , a , b]  $

\begin{align*}
    ^{inc}[id_{z_x \uplus (z_y \uplus z_m)}, A(f_x)(a), B(inl)(b')] &= \\
    ^{inc}[id_{z_x \uplus (z_y \uplus z_m)}, A(f_x; id_{z_x})(a), B(f_y ; inl)(b)]  &= \\
    ^{inc}[(f_x \uplus f_y) ; (id_{z_x} \uplus inl), a , b] &= \\
    ^{inc}[f , a , b]\\
    \qed
\end{align*}
\subsubsection{Strength Commutes with Monad Multiplication}

\end{document}