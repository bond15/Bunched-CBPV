\documentclass{article}
\usepackage{graphicx} % Required for inserting images
\usepackage{simplebnf}
\usepackage{bussproofs}
\usepackage{amsmath}
\usepackage{amssymb}
\usepackage[dvipsnames]{xcolor}
\usepackage{stmaryrd}
\usepackage{comment}
\usepackage{bm}

\usepackage[english]{babel}
\usepackage{amsthm}

\newtheorem{thm}{Theorem}
\newtheorem{deff}{Definition}

\newtheorem{lem}[thm]{Lemma}
\newtheorem{cor}[thm]{Corollary}
\newtheorem{rem}[thm]{Remark}
\newtheorem*{remark}{Remark}
\newtheorem{definition}[deff]{Definition}
\newtheorem*{construction}{Construction}
\newtheorem{conj}[thm]{Conjecture}

\begin{document}
%\usepackage{xcolor}

\newcommand{\den}[1]{\llbracket #1 \rrbracket}
\newcommand{\blue}[1]{\textcolor{blue}{#1}}
\newcommand{\red}[1]{\textcolor{red}{#1}}
\newcommand{\sep}{\mathrel{-\mkern-6mu*}}
\newcommand{\thunk}[1]{\textrm{thunk }#1}
\newcommand{\injj}[2]{\textrm{inj}_{#1}#2}
\newcommand{\err}{\mho}
\newcommand{\print}[1]{\textrm{print }#1}
\newcommand{\force}[1]{\textrm{force }#1}
\newcommand{\ret}[1]{\textrm{ret }#1}
\newcommand{\bind}[3]{#1 \leftarrow #2 ; #3}
\newcommand{\newcase}[3]{\textrm{newcase}_{#1} \; #2 ; #3}
\newcommand{\match}[5]{\textrm{match }#1 \textrm{ with }#2 \;\{#3 . #4 | #5\}}
\newcommand{\unpack}[4]{\textrm{unpack }(#1,#2) = #3 ; #4}
\newcommand{\lett}[4]{\textrm{let }(#1,#2) = #3 ; #4}
\newcommand{\lets}[4]{\textrm{let }(#1*#2) = #3 ; #4}
\newcommand{\ite}[3]{\textrm{if }#1 \textrm{ then }#2 \textrm{ else }#3}
\newcommand{\at}{\textrm{@}}
\newcommand{\ttt}{\textrm{tt}}
\newcommand{\tru}{\textrm{true}}
\newcommand{\fsl}{\textrm{false}}
\newcommand{\pworld}{\widehat{\mathbf{World}}}
\newcommand{\world}{{\mathbf{World}}}
\newcommand{\calculus}{\operatorname{-calculus}}
\section{Presheaf Hyperdoctrine}

Goal: Define a higher order logic for a simply typed language with a syntactic extensible sum type.
Since the term model uses presheaves, our hyperdoctrine will be based on presheaves. 
Let our base category be denoted $\mathbf{C}$. 
\blue{NOTE: we are working with covariant presheaves.}
Our term model with be denoted as $\widehat{\mathbf{C}} = [\bf{C} , \bf{SET}]$

Following the definition TODO Cite (biering bi hyperdoc 2007) a First-Order hyperdoctrine,
consists of a functor $\mathcal{P}$
\[
 \mathcal{P} : \bf{C}^{op} \rightarrow \bf{Poset}   
\]
such that
\begin{align*}
    & \forall (X : ob \;\bf{C}), \mathcal{P}(X) \textrm{ is a heyting algebra}\\
    & \forall (f : \bf{C}[X , Y]), \mathcal{P} : \mathcal{P}(Y) \rightarrow \mathcal{P}(X) \textrm{ is a heyting homomorphism}\\
    & diagonal \\
    & adjoints
\end{align*}

A cannonical way to construct a first order hyperdocrine is to have an \textit{internal} heyting algebra $H : ob \; \widehat{\bf{C}}$
and take $\mathcal{P}= y(H) = \widehat{\bf{C}}[- , H]$ where $y$ is the yonedda embedding.
\begin{thm}
    There exists an internal Heyting algebra $Prop$ in $Set$
\end{thm}
\begin{proof}
    details pending
\end{proof}

\begin{thm}
    $Prop$ can be \textit{lifted} to $\widehat{\bf{C}}$, 
    denoted as $Prop_{\widehat{\bf{C}}}$, and it is an internal Heyting algebra in $\widehat{\bf{C}}$.
\end{thm}
\begin{proof}
    details pending
\end{proof}

\begin{thm}
    $\mathcal{P} = \widehat{\bf{C}}[- , Prop_{\widehat{\bf{C}}}]$ is a hyperdoctrine on $\widehat{\bf{C}}$.
\end{thm}
\begin{proof}
    details pending, Each $\mathcal{P}(X)$ is a heyting algebra \textit{pointwise}.
    We have an ordering \textit{pointwise} as well.. but what is the ordering of the 
    internal heyting algebra $Prop{\widehat{\bf{C}}}$? 
    Is it the \textit{default} ordering you can define from a heyting algebra (page 199 \cite{Sheaves})
    \[
      X \leq Y \iff X \land Y = X  
    \]
    where does this equaltiy "live"?
    \blue{Answer} For separation logic it is the sieve of downward closed subsets 
    where the downward closed subsets use an extension ordering on heaps. 
    The collection of subsets in the sieve is then ordered by inclusion. 
    This is the ordering for a bialgebra
\end{proof}

With a hyperdoctrine in hand, we can interpret a higher order logic over 
a simply typed language. 

\begin{bnf}
    $\alpha$ : Types ::= Unit | $\alpha \times \alpha$ | $\alpha \rightarrow \alpha$ ;;
    $M$ : Terms ::= $tt$ | $(M , N)$ | $\pi_1 M$ | $\pi_2 M$ | $\lambda (x \colon \alpha). M$ | $MN$
\end{bnf}

\begin{construction}
    Denotation of Types
\end{construction}
\begin{align*}
    \den{Unit} &:= \bf{1} \\
    \den{\alpha \times \beta} &= \den{\alpha} \times \den{\beta}\\
    \den{\alpha \rightarrow \beta} &= \den{\beta}^{\den{\alpha}}
\end{align*}

\begin{construction}
    Denotation of Terms
\end{construction}
\begin{align*}
    \den{\Gamma \vdash tt : Unit} &= \den{\Gamma} \xrightarrow{\den{tt}} \den{Unit}\\ 
    \den{\Gamma \vdash (M , N) : \alpha \times \beta} &= \den{\Gamma} \xrightarrow{\Delta} \den{\Gamma} \times \den{\Gamma} \xrightarrow{\den{\Gamma \vdash M : \alpha}\times \den{\Gamma \vdash N : \beta}} \den{\alpha} \times \den{\beta}\\
    etc..
\end{align*}

\begin{definition}
    A higher order logic over the simply typed language.
\end{definition}
\begin{bnf}
    $\phi$ ::= $\top$ | $\bot$ | $P(x)$| $x =_{\alpha} y$ | $\phi \land \psi$ | $\phi \lor \psi$ | $\phi \implies \psi$ | $\forall (x \colon \alpha), \phi$ | $\exists (x \colon \alpha), \phi$ ;;
\end{bnf}

\begin{definition}
    Logic formation rules.
\end{definition}
\begin{remark}
    $\Gamma$ is a context of terms, $\Phi$ is a context of propositions.
\end{remark} 

\begin{prooftree}
    \AxiomC{}
    \RightLabel{$\top$-Form}
    \UnaryInfC{$\Gamma ; \Phi \vdash \top \textrm{ Prop}$}
\end{prooftree}

\begin{remark}
    $\bf{prop}$ is an additional type added to the term language.
    See page 313 \cite{Jacobs}
\end{remark}
\begin{remark}
    $\den{\bf{prop}} = Prop_{\widehat{\bf{C}}}$
\end{remark}
\begin{remark}
    $\bf{Prop}$ is a judgement stating the term is a well formed proposition.
\end{remark}
\begin{prooftree}
    \AxiomC{$\Gamma \vdash P : \alpha \rightarrow \bf{prop}$}
    \AxiomC{$\Gamma \vdash x : \alpha$}
    \RightLabel{$P$-Form}
    \BinaryInfC{$\Gamma ; \Phi \vdash P(x) \textrm{ Prop}$}
\end{prooftree}

\begin{prooftree}
    \AxiomC{$\Gamma \vdash M : \alpha$}
    \AxiomC{$\Gamma \vdash N : \alpha$}
    \RightLabel{$=$-Form}
    \BinaryInfC{$\Gamma ; \Phi \vdash M =_{\alpha} N \textrm{ Prop}$}
\end{prooftree}

\begin{remark}
    Notice the term context in $\forall$-Form
\end{remark}
\begin{prooftree}
    \AxiomC{$\Gamma , x : \alpha ; \Phi \vdash \phi \textrm{ Prop}$}
    \RightLabel{$\forall$-Form}
    \UnaryInfC{$\Gamma ; \Phi \vdash \forall(x : \alpha), \phi \textrm{ Prop}$}
\end{prooftree}

\begin{prooftree}
    \AxiomC{$\Gamma ; \Phi \vdash \phi \textrm{ Prop}$}
    \AxiomC{$\Gamma ; \Phi \vdash \psi \textrm{ Prop}$}
    \RightLabel{$\land$-Form}
    \BinaryInfC{$\Gamma ; \Phi \vdash \phi \land \psi \textrm{ Prop}$}
\end{prooftree}
etc..

\begin{construction}
    Denotation of propositions.
\end{construction}
Generic form: Given $\Gamma ; \Phi \vdash \phi \textrm{ Prop}$,
we have $\mathcal{P}(\den{\Gamma}) : ob \; \bf{Poset}$ 
which has an order $\leq_{\mathcal{P}(\den{\Gamma})}$.
\[
  \den{\Gamma ; \Phi \vdash \phi } = \den{\Phi} \leq _{\mathcal{P}(\den{\Gamma})} \den{\phi}  
\]
The relation $\den{\Phi} \leq _{\mathcal{P}(\den{\Gamma})} \den{\phi}$ may or may not hold. 
Our logic is proof irrelevant.

Specific Form:
\begin{align*}
    \den{\Gamma ; \Phi \vdash \top} &= \den{\Phi} \leq_{\mathcal{P}(\den{\Gamma})} \top \\
    \den{\Gamma ; \Phi \vdash \bot} &= \den{\Phi} \leq_{\mathcal{P}(\den{\Gamma})} \bot \\
    \den{\Gamma, x : \alpha ; \Phi \vdash P(x)} &= ? \\
    \den{\Gamma, x : \alpha , y : \alpha ; \Phi \vdash x =_{\alpha} y} &= ? \\
    \den{\Gamma ; \Phi \vdash \phi \lor \psi } &= \den{\Phi} \leq_{\mathcal{P}(\den{\Gamma})} (\den{\phi} \lor \den{\psi}) \\
    \den{\Gamma ; \Phi \vdash \phi \lor \psi } &= \den{\Phi} \leq_{\mathcal{P}(\den{\Gamma})} (\den{\phi} \lor \den{\psi}) \\
    \den{\Gamma ; \Phi \vdash \phi \implies \psi } &= \den{\Phi} \leq_{\mathcal{P}(\den{\Gamma})} (\den{\phi} \implies \den{\psi}) \\
    \den{\Gamma ; \Phi \vdash \forall(x : \alpha), \phi } &= \den{\Phi} \leq_{\mathcal{P}(\den{\Gamma})} \forall_{(\den{\Gamma}, \den{\alpha})}(\den{\phi}) \\
    \den{\Gamma ; \Phi \vdash \exists(x : \alpha), \phi } &= \den{\Phi} \leq_{\mathcal{P}(\den{\Gamma})} \exists_{(\den{\Gamma}, \den{\alpha})}(\den{\phi})
\end{align*}

\begin{definition}
    A proposition $\Gamma ; \cdot \vdash \phi$ is said to be \textbf{valid} when $\top \leq_{\mathcal{P}(\den{\Gamma})} \den{\phi}$
\end{definition}

derivations

relations?

\bibliographystyle{acm}
\bibliography{ref}
\end{document}cite{}