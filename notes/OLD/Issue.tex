\documentclass{article}
    % General document formatting
    \usepackage[margin=0.7in]{geometry}
    \usepackage[parfill]{parskip}
    \usepackage[utf8]{inputenc}
    \usepackage{xcolor}
    \usepackage{hyperref}
    
    % Related to math
    \usepackage{stmaryrd}
    \usepackage{amsmath,amssymb,amsfonts,amsthm}
    \newcommand{\den}[1]{\llbracket #1 \rrbracket}
    \newcommand{\sep}{\mathrel{-\mkern-6mu*}}
    \newcommand{\blue}[1]{\textcolor{blue}{#1}}
    \newcommand{\red}[1]{\textcolor{red}{#1}}
    \newcommand{\pshC}{\widehat{\mathcal{C}}}
    \newcommand{\pshCC}{\widehat{\mathcal{C \times C}}}
    \newcommand{\dayprod}{\otimes^{Day}}


\begin{document}
\section{Bicartesian Doubly Closed Category}
Given a category $\mathcal{C}$, its presheaf category ($\pshC := [\mathcal{C}^{op}, Set]$)
is bicartesian closed. Given a monoidal category ($\mathcal{C}, \otimes_C , I_C$), 
its presheaf category is bicartesian closed and monoidal closed via the Day convolution. 
The monoidal product is given by:
\[
    (P \dayprod Q)(x) = \int_{}^{y,z} \mathcal{C}[ x , y \otimes_C z ] 
    \times P(y) \times Q(z)
\]

The Day monoidal product has the universal property that any maps out of it are in bijective
correspondence with a family of maps natural in $x$ and $y$ 
\href{https://github.com/bond15/Bunched-CBPV/blob/82136dc6f4f9e4034391877f0d959a6ff1b62dfc/src/Data/BiDCC.agda#L222}
{(Agda)}: 
\footnote{here $\overline{\times}$ is the \textit{external} product}


\[
    \pshC[P \dayprod Q , R] \;\cong \; \pshCC [ P \overline{\times} Q , R \circ \otimes_C ]\; \cong \; \Pi_{x, y : \;ob \;C}\; Set[P(x) \times Q(y) , R(x \otimes_C y)]
\]

The monoidal closed structure is given by: 
\begin{align*}
    (P \sep Q)(X) = \pshC[P , Q(X , -)]
\end{align*}
With the universal property that the closed structure is right adjoint to the tensor
\href{https://github.com/bond15/Bunched-CBPV/blob/82136dc6f4f9e4034391877f0d959a6ff1b62dfc/src/Data/BiDCC.agda#L354}{(Agda)}: 
\begin{align}
    \pshC [ A \otimes_C B , C] \cong \pshC [ A , B \sep C]
\end{align}

Bicartesian doubly closed categories have been used in the denotational semantics of bunched type theories 
\cite{pym_semantics_2002}\cite{bieringLogicBunchedImplications}\cite{ohearn_bunched_2003}.

\section{Towards Bunched Call By Push Value with Dynamic Store}
Categorical models of dynamic store use presheaf categories to model the dependence of the heap structure on a current \textit{world}
\cite{CBPVbook}\cite{sterling_denotational_2023}\cite{kammarMonadFullGround2017}. Seemingly none of these existing models 
attempt to combine a call by push value language with the separating type connectives, $\otimes$ and $\sep$, used in bunched type theories.
Our investigation into possible models of such a language have run into some potential issues when attempting to define the computational separating function type.
To illustrate this, we will start with the 
model for a call by push value language with dynamic store presented in chapter 7 of Levy's thesis.

\subsection{Definitions}
Let ($C, \otimes_C , I_C$) be a monoidal category, the value category be $\mathcal{V} := [C^{op} , Set]$, 
computation category $\mathcal{C} := [C , Set]$, and use the \textit{standard}
monad for ground dynamic store with $F : \mathcal{V} \rightarrow \mathcal{C}$ as:
\[
    F(A)(x) := \Sigma_{y : ob \;C}\Sigma_{f : C^{op}[ x , y ]}A(y)    
\]
and $U : \mathcal{C} \rightarrow \mathcal{V}$ as :
\[
    U(\underline{B})(x) := \Pi_{y : ob \; C}\Pi_{f : C^{op}[ x , y]}\underline{B}(y)   
\]

The oblique morphisms in this model are given by families of maps:
\[
    \mathcal{O}[A , \underline{B}] := \Pi_{x : ob \; C} Set[A(x) , \underline{B}(x)]   
\]

we have the following isomorphims:
\[
    \mathcal{V}[A , U(\underline{B})] \cong \mathcal{O}[A , \underline{B}] \cong \mathcal{C}[F(A) , \underline{B}]   
\]
And we can attempt to define a computation separating function by:
\[
    (A \sep \underline{B})(x) := \Pi_{y : ob \;C} Set[A(y), \underline{B}(x \otimes_C y)]
\]

\subsection{Issue with Monad Strength Relative to Day Tensor}
One consequence of this is that in an algebra model of bunched CBPV (using monad T and day convolution for tensor), 
we can't define the monoidal closed structure in a similar way to the exponential (using strength and eval).

We have a monad on $\mathcal{V}$ via the adjunction between $F$ and $U$.
\[
    T(A)(x) := \Pi_{y : ob \; C}\Pi_{f : C^{op}[ x , y]}\Sigma_{z : ob \;C}\Sigma_{g : C^{op}[ y , z ]}A(z) 
\]
\subsubsection{Abstract}
It seems we cannot define the strength map:
\[
    str\otimes_{P,Q} : \mathcal{V}[ P \otimes T(Q), T (P \otimes Q)]
\]
via the universal property of tensor, it suffices to construct components of the form:
\[
    Set[P(x) \times T(Q)(y) ,T(P \otimes Q)(x \otimes_C y) ]
\]
introducing terms and unfolding definitions, we have:
\begin{align*}
    &p : P(x)\\
    &q : T(Q)(y) = \Pi_{u : ob \; C}\Pi_{f : C^{op}[ y , u]}\Sigma_{v : ob \;C}\Sigma_{g : C^{op}[ u , v ]}Q(v)\\
    &h : C^{op}[x\otimes_C y , z]
\end{align*}
we get to choose some future world $w$ of $z$, $C^{op}[z , w]$, 
at which we must provide 
\[ 
    ? : (P \otimes Q)(w) = \int_{}^{r,s} \mathcal{C}[ w , r \otimes_C s ] 
\times P(r) \times Q(s)
\]
How to proceed? Notice that in order to \textit{extract} a value $Q(\_)$ from $q$,
we need to provide a morphism $C^{op}[y , u]$ for some $u$. 
Working abstractly with no additional assumptions about $C$, all we can provide is $id_y$ which yields:
\[
    q(y)(id_y) : \Sigma_{v : ob \;C}\Sigma_{g : C^{op}[ y , v ]}Q(v)  
\]
Then our context becomes:
\begin{align*}
    &p : P(x)\\
    &g : C^{op}[ y , v ]\\
    &q' : Q(v)\\
    &h : C^{op}[x\otimes_C y , z]
\end{align*}
We still need to choose some future world $w$ of $z$.
From the data available to us, it seems we are stuck using identity, $id_z$, once again.
This leaves us with the obligation:
\[
    ? : (P \otimes Q)(z) = \int_{}^{r,s} \mathcal{C}[ z , r \otimes_C s ] 
\times P(r) \times Q(s)
\]
The choice of $r$ seems forced to be $x$ for which we have $p: P(x)$.
The natural choices for $s$ would be $v$ or $y$, but it seems we've hit a variance issue yet again.
Observe that \red{if $g: C^{op}[ y , v ]$ was instead $g : C^{op}[v , y]$} we'd have
\[
    ((id \otimes_C g) ;h , p , q') = (h , p , Q(g)(q'))
\]
which are equal relative to the coend quotient.
\subsubsection{Concrete}
Now we consider substituting the monoidal category ($C, \otimes_C , I_C$) with ($FinSet_{mono}^{op}, \uplus , \emptyset $)
where $\uplus$ is disjoint union of sets.
This category is used to represent single sorted dynamic heap configurations. 
Take $\mathcal{FS}$ as shorthand for $FinSet_{mono}$. Start again with the assumptions.
\begin{align*}
    &p : P(x)\\
    &q : T(Q)(y) = \Pi_{u : ob \; \mathcal{FS}}\Pi_{f : \mathcal{FS}[ y , u]}\Sigma_{v : ob \;\mathcal{FS}}\Sigma_{g : \mathcal{FS}[ u , v ]}Q(v)\\
    &h : \mathcal{FS}[x\uplus y , z]
\end{align*}
By working with a concrete category, we have a few more paths available to us.
\paragraph{Partition Method}
We recognize that since $h$ is injective and the domain is a disjoint union, $z$ is partitioned into three parts
\begin{align*}
    z_x :& \textrm{ the range of h restricted to x}\\
    z_y :& \textrm{ the range of h restricted to y}\\
    z_m :& \; z - (z_x \uplus z_y)\\
    \textrm{where } & z \cong z_x \uplus z_y \uplus z_m
\end{align*}
Thus we have maps
\begin{align*}
    &h_x: \mathcal{FS}[x , z_x]\\
    &h_y: \mathcal{FS}[y , z_y]
\end{align*}
The map $h_y$ can be used for $q$
\begin{align*}
        %&q(y)(id_y) : \Sigma_{v : ob \;\mathcal{FS}}\Sigma_{g : \mathcal{FS}[ y , v ]}Q(v)\\
       %&\textrm{or}\\
        &q(z_y)(h_y) : \Sigma_{v : ob \;\mathcal{FS}}\Sigma_{g : \mathcal{FS}[ z_y , v ]}Q(v)
\end{align*}

Context:
\begin{align*}
    &p : P(x)\\
    &q' : Q(v)\\
    &g : \mathcal{FS}[ z_y , v ]\\
    %&q' : Q(v)\\
    %&g : \mathcal{FS}[ y , v ]\\
    &h : \mathcal{FS}[x\uplus y , z]\\
    &h_x: \mathcal{FS}[x , z_x]\\
    &h_y: \mathcal{FS}[y , z_y]
\end{align*}
We can choose future world $z_{xm} \uplus v$ where $z_{xm} = z_x \uplus z_m$.
Thus our obligation is:
\[
    ? : (P \otimes Q)(z_{xm} \uplus v) = \int_{}^{r,s} \mathcal{FS}[r \uplus s,z_{xm} \uplus v ] 
    \times P(r) \times Q(s)
\]
chosing $r = z_{xm}$ and $s = v$:
\[
    ? : \mathcal{FS}[z_{xm} \uplus v,z_{xm} \uplus v ] \times P(z_{xm}) \times Q(v)
\]
which we can satisfy by 
\[
  (id, P(h_x ; inl)(p), q')  
\]
The arbitrary choice here is that either $p$ or $q$ gets \textit{the extra space} in $z$.
\blue{This seems to be a messier version of the \textit{projection} method below. }

\paragraph{Projection Method}
\begin{align*}
    &p : P(x)\\
    &q : T(Q)(y) = \Pi_{u : ob \; \mathcal{FS}}\Pi_{f : \mathcal{FS}[ y , u]}\Sigma_{v : ob \;\mathcal{FS}}\Sigma_{g : \mathcal{FS}[ u , v ]}Q(v)\\
    &h : \mathcal{FS}[x\uplus y , z]
\end{align*}
we have morphisms
\begin{align*}
    & j : \mathcal{FS}[x , z]\\
    & k : \mathcal{FS}[y , z]
\end{align*}
via precomposing $h$ with $inl$ and $inr$. We can use $k$ with $q$ yielding
\begin{align*}
    &q' : Q(v)\\
    &g : \mathcal{FS}[ z , v ]\\
\end{align*}
This time we have another choice of future world from $z$, being $v$. 
$Q$ is covariant w.r.t $\mathcal{FS}$, so we have no way to transport $q' : Q(v)$ back to any
previous world of $v$.
\[
  (h ; g , p, \red{Q(k ; g)(q')})  
\]
We can perform an arbitrary \textit{hack}\footnote{see section 2.4.2 for similar reasoning}
by noting that we can lift $p: P(x)$ to $p' = P(j ; g)(p) : P(v)$ which we can use to construct an element of 
\[
    (P \otimes Q)(v \uplus_C v) = \int_{}^{r,s} \mathcal{FS}[r \uplus s, v \uplus v] 
    \times P(r) \times Q(s)
\]
via $(id, P(j ; g)(p), q')$. 
Then, using the contravariance of $F(P \otimes Q)$, 
we can go from $F(P \otimes Q)(v \uplus v)$ to $F(P \otimes Q)(v)$ via $g ; inl$ or $g ; inr$.
\red{TODO: I believe this definition of components fails naturality.}

\red{This definition fails to satisfy the following equation: }
\[
  A \otimes B \xrightarrow{A \otimes \eta_B} A \otimes T(B) \xrightarrow{str_{A,B}} T(A \otimes B) 
  = A \otimes B \xrightarrow{\eta_{A\otimes B}}T(A \otimes B)
\]
See \href{https://github.com/bond15/Bunched-CBPV/blob/b77d06a7207a6d42b86633f52a9fe93f6beaafd1/src/Data/ConcreteFin.agda#L197}{Agda}.
 Equality fails at the first component of the tuple, $z \uplus z \neq z$.


\subsection{Problems with an Abstract Monoidal Category}
Before committing to the category 
of worlds used in Levy's model, we will work with an arbitrary monoidal category ($C, \otimes_C , I_C$). 
\subsubsection{Issue 1: Universal Property of Tensor for Oblique Morphisms}
Let's attempt to show the following:
\[
  \mathcal{O}[P \otimes Q , \underline{R}] \cong \mathcal{O\times}[P \overline{\times} Q , \underline{R} \circ \otimes_C]  
\]
where 
\[
    \mathcal{O\times}[P \overline{\times} Q , \underline{R} \circ \otimes_C] 
    := \Pi_{x , y : ob \;C}Set[P(x)\times Q(y) , \underline{R}(x \otimes_C y)]
\]

A problem arises when trying to define the backwards map of this isomorphims. 
Given $m : \mathcal{O\times}[P \overline{\times} Q , \underline{R} \circ \otimes_C]$ and $x : ob \; C$, we need to define a map
$Set[(P \otimes Q)(x), \underline{R}(x)]$. This is a map out of a coequalizer 
\footnote{since coends in $Set$ can be encoded as coequalizers}
which we can attempt to give as a map induced from:
\[
    (f : C[x , y\otimes_C z] , p : P(y) , q : Q(z)) \mapsto \;? : \underline{R}(x)
\]
However, using the data we currently have, we can only construct
\[
    m(y)(z)(p,q) : \underline{R} (y \otimes_C z)    
\]
and since $\underline{R}$ is covariant in $C$, we can't use $\underline{R}(f) : \underline{R}(x) \rightarrow \underline{R}(y \otimes_C z)$.
This is not surprising since the proof of this universal property in the value category 
$\mathcal{V}[P \otimes Q , R] \cong \mathcal{V\times}[P \overline{\times} Q , R \circ \otimes_C]$
uses the functorial action of $R$ on $f$ 
(see \href{https://github.com/bond15/Bunched-CBPV/blob/d4de5ebe3a2a42499b24c13a8d2da7f3a2cc1b36/src/Data/BiDCC.agda#L120}{here})
\footnote{note the difference in variance is due to the fact this proof is for presheaves and not covariant presheaves}
So by swapping the variance of $R$ (now $\underline{R}$ since it is from the computation category) this proof should break.
Seemingly, this proof won't go through when we assume a generic monoidal category $C$. Perhaps we can recover this property if 
we work with a specific concrete category?

\subsubsection{Issue 2: Universal Property of the Separating Function Type}
This is just another perspective on the variance issue above. We'd like to show
\[
    \mathcal{O}[P \otimes Q , \underline{R}] \cong \mathcal{O}[P , Q \sep \underline{R}]    
\]
Since we don't have the universal property of tensor for oblique morphisms, we can try to get at this proof via the 
universal property of tensor in the value category. Note that we have
\[
    \mathcal{O}[P \otimes Q , \underline{R}] \cong \mathcal{V}[P \otimes Q , U(\underline{R})] \cong \mathcal{V\times}[P \overline{\times} Q , U(\underline{R}) \circ \otimes_C] 
\]
and 
\[
    \mathcal{O}[P , Q \sep \underline{R}] \cong \mathcal{V}[P , U(Q \sep \underline{R})]
\]
So we can try to show 
\[
    \mathcal{V\times}[P \overline{\times} Q , U(\underline{R}) \circ \otimes_C] \cong \mathcal{V}[P , U(Q \sep \underline{R})]
\]
Again, we fail to define the backwards direction of this isomorphim due to a variance issue with $\underline{R}$. 
Given $m : \mathcal{V}[P , U(Q \sep \underline{R})]$, it suffices to construct a map 
$eval : \mathcal{V\times}[ U(Q \sep \underline{R}) \overline{\times} Q , U(\underline{R}) \circ \otimes_C] $ with components 
\[
    (x , y)(f : U(Q \sep \underline{R})(x), q : Q(y)) \mapsto \;? \;: \;  (U(\underline{R}))(x \otimes_C y)    
\]
unfolding some of the definitions, we have 
\begin{align*}
    f &: \Pi_{z : ob \;C}\Pi_{g : C^{op}[x , z]}(\Pi_{w : ob \; C}Set[Q(w), \underline{R}(z \otimes_c w)])\\
    ? &: \Pi_{z : ob \;C}\Pi_{g : C^{op}[x \otimes_C y, z]}(R(z))
\end{align*}
Thus we have to define $? : \underline{R}(z)$ from the following data:
\begin{align*}
    &x,y,z : ob \;C \\
    &q : Q(y)\\
    &f : \Pi_{z : ob \;C}\Pi_{g : C^{op}[x , z]}(\Pi_{w : ob \; C}Set[Q(w), \underline{R}(z \otimes_c w)])\\
    &g : C^{op}[x\otimes_C y , z]
\end{align*}
The \textit{obvious} thing do to would be to use $f (x)(id_x)(y)(q) : \underline{R}(x \otimes_C y)$ and $\underline{R}(g)$, 
but the variance of $\underline{R}$ is working against us.

\subsubsection{Issue 3: Problem with Action Laws}
The following three isomorphims should hold:
\begin{align*}
    I_{D} \sep^{c} Q &\cong Q\\
    P \otimes_D P' \sep^{c} Q &\cong P \sep^{c} P' \sep^{c} Q\\
    U(P \sep^{c}Q) &\cong P \sep^{v} UQ
\end{align*}
where $P , P' : ob \;\mathcal{V}$, $Q : ob \; \mathcal{C}$, and $\otimes_D,I_D:= Yoneda(I_C) = C[- , I_C]$ are the day convolution product and its identity.
The top two isomorphims come from the requirement that CBPV function types should be an action 
of $\mathcal{V}^{op}$ on $\mathcal{C}$. Let's see how the first isomorphism fails. 
Defining the forward direction by its components:
\begin{align*}
    &(x : ob \; C)(f : \Pi_{y : ob \;C}Set[C[y , I_C], Q(x \otimes_C y)]) \mapsto \;?\; : Q(x)
\end{align*}
we have
\[
  f(I_C)(id_{I_C}) : Q(x \otimes_C I_C)  
\]
from which we can obtain
\[
  Q(x \otimes_C I_C \xrightarrow{idr} x)(f(I_C)(id_{I_C}) : Q(x \otimes_C I_C)) : Q(x)  
\]
In attempting to define the backwards direction, we run into issues.
\[
    (x : ob \; C)(q : Q(x))(y : ob \;C)(f : C[y , I_C]) \mapsto \;?\; : Q (x \otimes_C y)    
\]
We'd expect to use the functorial action of $Q$ on morphisms of $C$
\[
    Q(g)(q) : Q(x \otimes_C y)
\]
but we'd need a morphism $g : C [ x , x\otimes_C y]$. 
Note that \textbf{if the direction of $f$ were inverted}, we'd be able to define this by 
\[
  g := x \xrightarrow{idr^{-1}} x \otimes_C I_C \xrightarrow{id_x \otimes_C f} x \otimes_C y  
\]

\subsection{Problems with Concrete Models}
Now we consider substituting the monoidal category ($C, \otimes_C , I_C$) with ($FinSet_{mono}^{op}, \oplus , \emptyset $)
where $\oplus$ is given by disjoint union of sets. 
This category is used to represent single sorted dynamic heap configurations. 

\subsubsection{Action Laws with Concrete Category}
Again we attempt to show $I_{D} \sep^{c} Q \cong Q$. The forward direction holds following the abstract case.
It seems we can't define the backwards direction.
\[
    (x : ob \;C)(r : Q(x))(y : ob \;C)(! : C[ \emptyset, y] ) \mapsto \; ? \; : Q(x \uplus y)
\]
again, we'd expect to use the functorial action of $Q$
\[
    Q(g)(r) : Q (x \uplus y)    
\]
where $g : C[x \uplus y , x]$, but we have \red{no hope of defining this morphism}! \red{HARD STOP}.
Notice that even if the given map was inverted as we desired in the abstract case, that is, $C[y , \emptyset]$ 
instead of $C[\emptyset , y]$, then we could define this backwards isomorphim using \red{absurd}!

\subsubsection{Universal Property of Tensor for Oblique Morphisms}
See Agda file \href{https://github.com/bond15/Bunched-CBPV/blob/cc0cb155be26ca383343247485e462ec2377dfe6/src/Data/ConcreteFin.agda#L85}{here}.
Again, we will attempt to show the following:
\[
    \mathcal{O}[P \otimes Q , \underline{R}] \cong \mathcal{O\times}[P \overline{\times} Q , \underline{R} \circ \oplus]  
\]
We reconsider the backwards direction, given 
\[
    m : \Pi_{x , y : ob \;C} Set[P(x)\times Q(y), \underline{R}(x \uplus y)]
\]
we need to construct components of the form $?_z : Set[(P\otimes Q)(z),  \underline{R}(z)]$. 
To map out of the coequalizer, we can define a map
\[
    (f: x\uplus y \rightarrow z , p : P(x), q : Q(y)) \mapsto \; ? : \underline{R}(z)
\]
\textbf{Attempt 1}\\
We can promote $p$ and $q$ to the larger world $z$.
\begin{align*}
    g &: x \rightarrow z = f \circ inl \\
    h &: y \rightarrow z = f \circ inr \\
    p' &: P(z) = P(g)(p)\\
    q' &: Q(z) = Q(h)(q)
\end{align*}
We can then use $m$ at $(z , z)$
\[ 
    m(z)(z)(p' , q') : \underline{R}(z \uplus z)
\]
and \red{\textbf{arbitrarily}} restrict the resulting element of $\underline{R}(z \uplus z)$ to $\underline{R}(z)$ 
using $\underline{R}(inl)$ \red{or} $\underline{R}(inr)$. This does satisfy the coequalizer requirement. 
To define a map out of the day convolution product, we can define a map on the underlying \textit{diagram}
 and then prove a coequalizer condition. Given:
 \begin{align*}
    &y \;z \;y'\; z' : ob \;C \\
    &f : C[y' , y] \\ 
    &g : C[z' , z] \\
    &(h : C[x , y'\otimes_c z'], p : P(y), q: Q(z))
 \end{align*}
Any map, $m$, out of the diagram must satisfy
\[
    m (h ; (f \otimes_c g), p , q) = m (h , P(f)(p), Q(g)(q))
\]
Concretely, in this case:
\begin{align*}
    &R(inl)(m\;x \;x (P(inl ; f\otimes_c g ; h)(p) , Q(inr ; f\otimes_c g , h)(q)))  \\
    &=\\
    &R(inl)(m \;x \;x (P(f ; inl ; h)(p),Q(g ; inr ; h)(q)))
\end{align*}

However, the section and retraction of the isomorphim looks less promising. 
It seems we are missing some kind of naturality condition.

Section:
\begin{align*}
    &\textrm{Given:}\\
    &b : \mathcal{O\times}[P \overline{\times} Q , \underline{R} \circ \oplus]\\
    &x\;y : ob \;FinSet_{mono}\\
    &p : P(x)\\
    &q : Q(y)\\
    &\textrm{Show:}\\
    &R(inl)(b (x \uplus y)(x \uplus y)(P(inl)(p),Q(inr)(q))) = b\; x\; y\;(p ,q)
\end{align*}

Retraction:
\begin{align*}
    &\textrm{Given:}\\
    &b : \mathcal{O}[P \otimes Q , \underline{R}]\\
    &x \;y \;z : ob \;FinSet_{mono}\\
    &f : FinSet_{mono}[y \uplus z , x]\\
    &p : P(y)\\
    &q : Q(z)\\
    &\textrm{Show:}\\
    &R(inl)(b (x \uplus y)[(id , P(inl ; f)(p), Q(inr ; f)(q))]) = b \; x \; [(f , p , q)]
\end{align*}

\textbf{Attempt 2}\\
We reconsider the backwards direction, given 
\[
    m : \Pi_{x , y : ob \;C} Set[P(x)\times Q(y), \underline{R}(x \uplus y)]
\]
we need to construct components of the form $?_z : Set[(P\otimes Q)(z),  \underline{R}(z)]$. 
To map out of the coequalizer, we can define a map
\[
    (f: x\uplus y \rightarrow z , p : P(x), q : Q(y)) \mapsto \; ? : \underline{R}(z)
\]
We can recognize that since $f$ is injective and the domain is a disjoint union, $z$ is partitioned into three parts
\begin{align*}
    z_x :& \textrm{ the range of f restricted to x}\\
    z_y :& \textrm{ the range of f restricted to y}\\
    z_{miss} :& \; z - (z_x \uplus z_y)\\
    \textrm{where } & z \cong z_x \uplus z_y \uplus z_{miss}
\end{align*}
Thus we can \red{\textbf{arbitrarily}} promote $p$ to $p' : P(z_x \uplus z_{miss})$ \red{or} $q$ to $q' : Q(z_y \uplus z_{miss})$
\begin{align*}
    m (z_x \uplus z_{miss})(z_y)(p' , q)\\
    m (z_x)(z_y \uplus z_{miss})(p , q')
\end{align*}
Choose the first option, check the coequalizer condition.
\begin{align*}
    &p : P(y) , q : Q(z)\\
    &f : y \rightarrow y' \\
    &g : z \rightarrow z'\\
    &m (x_{y} \uplus x_{miss})(x_{z})(P(y\rightarrow(x_{y} \uplus x_{miss}))(p) , q) 
    = m (x_{y'} \uplus x_{miss})(x_{z'})(P(),Q(g)(q))
\end{align*}

\bibliographystyle{acm}
\bibliography{ref}
\end{document}