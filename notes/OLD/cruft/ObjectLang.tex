\documentclass{article}
\usepackage{graphicx} % Required for inserting images
\usepackage{simplebnf}
\usepackage{bussproofs}
\usepackage{amsmath}
\usepackage{amssymb}
\usepackage[dvipsnames]{xcolor}
\usepackage{stmaryrd}
\usepackage{comment}
\usepackage{bm}

\begin{document}
%\usepackage{xcolor}

\newcommand{\den}[1]{\llbracket #1 \rrbracket}
\newcommand{\blue}[1]{\textcolor{blue}{#1}}
\newcommand{\red}[1]{\textcolor{red}{#1}}
\newcommand{\sep}{\mathrel{-\mkern-6mu*}}
\newcommand{\thunk}[1]{\textrm{thunk }#1}
\newcommand{\injj}[2]{\textrm{inj}_{#1}#2}
\newcommand{\err}{\mho}
\newcommand{\print}[1]{\textrm{print }#1}
\newcommand{\force}[1]{\textrm{force }#1}
\newcommand{\ret}[1]{\textrm{ret }#1}
\newcommand{\bind}[3]{#1 \leftarrow #2 ; #3}
\newcommand{\newcase}[3]{\textrm{newcase}_{#1} \; #2 ; #3}
\newcommand{\match}[5]{\textrm{match }#1 \textrm{ with }#2 \;\{#3 . #4 | #5\}}
\newcommand{\unpack}[4]{\textrm{unpack }(#1,#2) = #3 ; #4}
\newcommand{\lett}[4]{\textrm{let }(#1,#2) = #3 ; #4}
\newcommand{\lets}[4]{\textrm{let }(#1*#2) = #3 ; #4}
\newcommand{\ite}[3]{\textrm{if }#1 \textrm{ then }#2 \textrm{ else }#3}
\newcommand{\at}{\textrm{@}}
\newcommand{\ttt}{\textrm{tt}}
\newcommand{\tru}{\textrm{true}}
\newcommand{\fsl}{\textrm{false}}
\newcommand{\pworld}{\widehat{\mathbf{World}}}
\newcommand{\world}{{\mathbf{World}}}
\newcommand{\calculus}{\operatorname{-calculus}}

\section{Object Language}
\subsection{Raw Terms}
The untyped syntax for CBPV OSum, add separating connectives, remove error.

%CBPV OSum%
\begin{bnf}
    $A$ : Value Types ::= $X$
    | Unit
    | Case $A$
    | OSum
    | $A \times A$
    | $A * A$
    | $\exists X . A$
    | $U \underline{B}$
    ;;
    $\underline{B}$ : Computation Types ::= $A \rightarrow \underline{B}$
    | $A \sep \underline{B}$
    | $\forall X . \underline{B}$
    | $F A$
    ;;
    $V$ : Values ::= $x$
    | $tt$
    | $\sigma$
    | $\textrm{inj}_V V$
    | $(V , V)$
    | $(V * V)$
    | pack $(A , V)$ as $\exists X .A$
    | thunk $M$
    ;;
    $M$ : Computations ::= 
     $\lambda x \colon A . M$
    | $M V$
    | $\alpha x \colon A . M$ 
    | $M @ V$
    | $\Lambda X . M$
    | $M[A]$
    | ret $V$
    | force $V$
    | newcase$_A x ; M$
    | match V with V \{ inj $x . M \| N$ \}
    | let ($x , x$) = $V ; M$
    | let ($x * x$) = $V ; M$
    | unpack $(X , x) = V ; M$
    ;;
    $\Gamma$ : Value Context ::= $\cdot$
    | $\Gamma , x \colon A$ 
    | $\Gamma * x \colon A$
    ;;
    $\Delta$ : Type Context ::= $\cdot$
    | $\Delta , X$
    ;;
\end{bnf}
\subsection{Typed Terms}

\begin{prooftree}
    \AxiomC{}
    \RightLabel{}
    \UnaryInfC{$\Delta ; \Gamma, x \colon A \vdash_v x : A$}
\end{prooftree}

\begin{prooftree}
    \AxiomC{}
    \RightLabel{}
    \UnaryInfC{$\Delta ; \Gamma * x \colon A  \vdash_v x : A$}
\end{prooftree}

\begin{prooftree}
    \AxiomC{}
    \RightLabel{}
    \UnaryInfC{$\Delta ; \Gamma \vdash_v \ttt : \textrm{Unit}$}
\end{prooftree}


\begin{prooftree}
    \AxiomC{$\Delta ; \Gamma \vdash_v \sigma : \textrm{Case} A$}
    \AxiomC{$\Delta ; \Gamma \vdash_v V : A$}
    \RightLabel{}
    \BinaryInfC{$\Delta ; \Gamma \vdash_v \injj{\sigma}{V}: \textrm{OSum}$}
\end{prooftree}

\begin{prooftree}
    \AxiomC{$\Delta ; \Gamma \vdash_v V_1 : A_1$}
    \AxiomC{$\Delta ; \Gamma \vdash_v V_2 : A_2$}
    \RightLabel{}
    \BinaryInfC{$\Delta ; \Gamma \vdash_v (V_1,V_2) : A_1 \times A_2$}
\end{prooftree}

\begin{prooftree}
    \AxiomC{$\Delta ; \Gamma_1 \vdash_v V_1 : A_1$}
    \AxiomC{$\Delta ; \Gamma_2 \vdash_v V_2 : A_2$}
    \RightLabel{}
    \BinaryInfC{$\Delta ; \Gamma_1 * \Gamma_2 \vdash_v (V_1 * V_2) : A_1 * A_2$}
\end{prooftree}

\begin{prooftree}
    \AxiomC{$\Delta ; \Gamma \vdash_v V : A[A'/X]$}
    \RightLabel{}
    \UnaryInfC{$\Delta ; \Gamma \vdash_v \textrm{pack}(A',V) \textrm{ as } \exists X .A : \exists X .A$}
\end{prooftree}

\begin{prooftree}
    \AxiomC{$\Delta ; \Gamma \vdash_c M : \underline{B} $}
    \RightLabel{}
    \UnaryInfC{$\Delta ; \Gamma \vdash_v \textrm{thunk } M : U \underline{B}$}
\end{prooftree}

\begin{prooftree}
    \AxiomC{$\Delta ; \Gamma , x : A \vdash_c M : \underline{B}$}
    \RightLabel{}
    \UnaryInfC{$\Delta ; \Gamma \vdash _c  \lambda x : A . M : A \rightarrow \underline{B}$}
\end{prooftree}

\begin{prooftree}
    \AxiomC{$\Delta ; \Gamma \vdash_c M : A \rightarrow \underline{B}$}
    \AxiomC{$\Delta ; \Gamma \vdash_v N : A$}
    \RightLabel{}
    \BinaryInfC{$\Delta ; \Gamma \vdash_c M N : \underline{B}$}
\end{prooftree}

\begin{prooftree}
    \AxiomC{$\Delta ; \Gamma * x : A \vdash_c M : \underline{B}$}
    \RightLabel{}
    \UnaryInfC{$\Delta ; \Gamma \vdash _c  \alpha x : A . M : A \sep \underline{B}$}
\end{prooftree}

\begin{prooftree}
    \AxiomC{$\Delta ; \Gamma_1 \vdash_c M : A \sep \underline{B}$}
    \AxiomC{$\Delta ; \Gamma_2 \vdash_v N : A$}
    \RightLabel{}
    \BinaryInfC{$\Delta ; \Gamma_1 * \Gamma_2 \vdash_c M @ N : \underline{B}$}
\end{prooftree}

\begin{prooftree}
    \AxiomC{$\Delta , X ; \Gamma \vdash_c M : \underline{B}$}
    \RightLabel{}
    \UnaryInfC{$\Delta ; \Gamma \vdash_c \Lambda X . M : \forall X . \underline{B}$}
\end{prooftree}

\begin{prooftree}
    \AxiomC{$\Delta ; \Gamma \vdash_c : M : \forall X . \underline{B}$}
    \AxiomC{$\Delta \vdash A$}
    \RightLabel{}
    \BinaryInfC{$\Delta ; \Gamma \vdash_c M[A] : \underline{B}[A / X]$}
\end{prooftree}

\begin{prooftree}
    \AxiomC{$\Delta ; \Gamma \vdash_v V : A$}
    \RightLabel{}
    \UnaryInfC{$\Delta ; \Gamma \vdash _c \textrm{ret } V : F A$}
\end{prooftree}

\begin{prooftree}
    \AxiomC{$\Delta ; \Gamma \vdash_v V : U \underline{B}$}
    \RightLabel{}
    \UnaryInfC{$\Delta ; \Gamma \vdash_c : \textrm{force } V : \underline{B}$}
\end{prooftree}

\begin{prooftree}
    \AxiomC{$\Delta ; \Gamma * (\sigma : \textrm{Case}A ) \vdash_c M : \underline{B}$}
    \AxiomC{$\Delta \vdash A$}
    \RightLabel{}
    \BinaryInfC{$\Delta ; \Gamma \vdash_c \textrm{newcase}_{A} x ; M : \underline{B}$}
\end{prooftree}

\begin{prooftree}
    \AxiomC{$\Delta ; \Gamma \vdash_v V : \textrm{OSum}$}
    \AxiomC{$\Delta ; \Gamma \vdash_v \sigma : \textrm{Case }A$}
    \AxiomC{$\Delta ; \Gamma , x : A \vdash M : \underline{B}$}
    \AxiomC{$\Delta ; \Gamma \vdash_c N : \underline{B}$}
    \RightLabel{}
    \QuaternaryInfC{$\Delta ; \Gamma \vdash_c \textrm{match } V \textrm{ with } \sigma \{ \textrm{ inj }x. M \| N\} : \underline{B}$}
\end{prooftree}

\begin{prooftree}
    \AxiomC{$\Delta ; \Gamma \vdash_v V : A_1 \times A_2$}
    \AxiomC{$\Delta ; \Gamma , x : A_1 , y : A_2 \vdash_c M : \underline{B}$}
    \RightLabel{}
    \BinaryInfC{$\Delta ; \Gamma \vdash_c$ let $(x,y) = V; M : \underline{B}$}
\end{prooftree}

\begin{prooftree}
    \AxiomC{$\Delta ; \Gamma \vdash_v V : A_1 * A_2$}
    \AxiomC{$\Delta ; \Gamma * x : A_1 * y : A_2 \vdash_c M : \underline{B}$}
    \RightLabel{}
    \BinaryInfC{$\Delta ; \Gamma \vdash_c$ let $(x*y) = V; M : \underline{B}$}
\end{prooftree}


\begin{prooftree}
    \AxiomC{$\Delta \vdash \underline{B}$}
    \AxiomC{$\Delta ; \Gamma \vdash_v V : \exists X . A$}
    \AxiomC{$\Delta , X ; \Gamma , x : A \vdash_c M : \underline{B}$}
    \RightLabel{}
    \TrinaryInfC{$\Delta ; \Gamma \vdash_c : \textrm {unpack} (X , x) = V ; M : \underline{B}$}
\end{prooftree}

\section{Meta Language}
\subsection{Raw Formulas}
    
%CBPV OSum%
\begin{bnf}
    $\phi,\psi$ : Formula ::= 
   % $\bot$%
     $t =_A u$ 
    | $R(t,u)$
    | $\phi \implies \psi$
    | $\phi \land \psi$
    | $\phi \lor \psi$ 
    | $\exists x \colon A .\phi$
    | $\exists X . \phi$
    | $\exists \underline{X} . \phi$
    | $\exists R \colon Rel[A,B] . \phi$
    | $\forall x \colon A . \phi$ 
    | $\forall X . \phi$
    | $\forall \underline{X} . \phi$
    | $\forall R \colon Rel[A,B] . \phi$
    ;;
\end{bnf}
\subsection{Typed Formulas}
Propositions, or well-formed formulas, use a term environment $\Gamma$, type environment $\Delta$
and relation environment $\Theta$. The typing judgement for Propositions is $\Delta ; \Gamma ; \Theta \vdash_p P$.
There are value relations and computation relations.

\begin{prooftree}
    \AxiomC{$\Delta ; \Gamma \vdash_\_ t : A$}
    \AxiomC{$\Delta ; \Gamma \vdash_\_ u : A$}
    \RightLabel{for $v,c$}
    \BinaryInfC{$\Delta ; \Gamma ; \Theta \vdash_p t =_A u $}
\end{prooftree}

\begin{prooftree}
    \AxiomC{$\Delta ; \Gamma \vdash_\_ t : A$}
    \AxiomC{$\Delta ; \Gamma \vdash_\_ u : B$}
    \AxiomC{$R : Rel_\_[A,B]\in \Theta$}
    \RightLabel{for $v,c$}
    \TrinaryInfC{$\Delta ; \Gamma ; \Theta \vdash_p R(t,u) $}
\end{prooftree}

\begin{prooftree}
    \AxiomC{$\Delta ; \Gamma ; \Theta \vdash_p \phi$}
    \AxiomC{$\Delta ; \Gamma ; \Theta \vdash_p \psi$}
    \RightLabel{$\square \in \{\land , \lor , \implies\}$}
    \BinaryInfC{$\Delta ; \Gamma ; \Theta \vdash_p \phi \square \psi$}
\end{prooftree}

\blue{what about *? Something like exists fresh }
\begin{prooftree}
    \AxiomC{$\Delta ; \Gamma , x : A | \Theta \vdash_p \phi$}
    \RightLabel{}
    \UnaryInfC{$\Delta ; \Gamma ; \Theta \vdash_p \exists x : A . \phi$}
\end{prooftree}

\begin{prooftree}
    \AxiomC{$\Delta ; \Gamma ; \Theta \vdash_p \phi$}
    \RightLabel{$X \notin FV(\Delta,\Gamma,\Theta)$}
    \UnaryInfC{$\Delta ; \Gamma ; \Theta \vdash_p \exists X . \phi$}
\end{prooftree}

\begin{prooftree}
    \AxiomC{$\Delta ; \Gamma ; \Theta \vdash_p \phi$}
    \RightLabel{$\underline{X} \notin FV(\Delta,\Gamma,\Theta)$}
    \UnaryInfC{$\Delta ; \Gamma ; \Theta \vdash_p \exists \underline{X} . \phi$}
\end{prooftree}

\begin{prooftree}
    \AxiomC{$\Delta ; \Gamma ; \Theta , R : Rel_\_[A,B] \vdash_p \phi$}
    \RightLabel{for $\{v,c\}$}
    \UnaryInfC{$\Delta ; \Gamma ; \Theta \vdash_p \exists R:Rel_\_[A,B].\phi$}
\end{prooftree}

\begin{prooftree}
    \AxiomC{$\Delta ; \Gamma , x : A | \Theta \vdash_p \phi$}
    \RightLabel{}
    \UnaryInfC{$\Delta ; \Gamma ; \Theta \vdash_p \forall x : A . \phi$}
\end{prooftree}

\begin{prooftree}
    \AxiomC{$\Delta ; \Gamma ; \Theta \vdash_p \phi$}
    \RightLabel{$X \notin FV(\Delta,\Gamma,\Theta)$}
    \UnaryInfC{$\Delta ; \Gamma ; \Theta \vdash_p \forall X . \phi$}
\end{prooftree}

\begin{prooftree}
    \AxiomC{$\Delta ; \Gamma ; \Theta \vdash_p \phi$}
    \RightLabel{$\underline{X} \notin FV(\Delta,\Gamma,\Theta)$}
    \UnaryInfC{$\Delta ; \Gamma ; \Theta \vdash_p \forall \underline{X} . \phi$}
\end{prooftree}

\begin{prooftree}
    \AxiomC{$\Delta ; \Gamma ; \Theta , R : Rel_\_[A,B] \vdash_p \phi$}
    \RightLabel{for $\{v,c\}$}
    \UnaryInfC{$\Delta ; \Gamma ; \Theta \vdash_p \forall R:Rel_\_[A,B].\phi$}
\end{prooftree}


\subsection{Typed Relations}
Relations are of the form $(x : A , y : B). \phi$ where $\phi$ is a proposition that can use $x,y$.
The typing judgement for relations is $\Delta ; \Gamma ; \Theta \vdash_r (x : A , y : B). \phi : Rel_\_[A,B]$.
The body of the relation is a proposition. 
Here we pay attention to the difference between value and computation relations.

\blue{again, what about *?}
\begin{prooftree}
    \AxiomC{$\Delta ; \Gamma , x : A \vdash_v t : C$}
    \AxiomC{$\Delta ; \Gamma , y : B \vdash_v u : C$}
    \RightLabel{}
    \BinaryInfC{$\Delta ; \Gamma ; \Theta \vdash_r (x : A , y : B). t =_C u : Rel_v[A,B]$}
\end{prooftree}
\blue{secretly inserting stoup}
\begin{prooftree}
    \AxiomC{$\Delta ; \Gamma | x : \underline{A} \vdash_c t : \underline{C}$}
    \AxiomC{$\Delta ; \Gamma | y : \underline{B} \vdash_c u : \underline{C}$}
    \RightLabel{}
    \BinaryInfC{$\Delta ; \Gamma ; \Theta \vdash_r (x : \underline{A}, y : \underline{B}). t =_{\underline{C}} u : Rel_c[\underline{A},\underline{B}]$}
\end{prooftree}
Given some $x : A$ and $y : B$ the terms $t,u$ are related by $R$, thus we have a relation on $A,B$.
Think of these like a lambda abstraction over two parameters. If the body is related, we can 
\begin{prooftree}
    \AxiomC{$\Delta ; \Gamma , x : A \vdash_v t : C$}
    \AxiomC{$\Delta ; \Gamma , y : B \vdash_v u : D$}
    \RightLabel{}
    \BinaryInfC{$\Delta ; \Gamma ; \Theta, R : Rel_v[C,D] \vdash_r (x : A , y : B). R(t,u) : Rel_v[A,B]$}
\end{prooftree}

\begin{prooftree}
    \AxiomC{$\Delta ; \Gamma | x : \underline{A} \vdash_c t : \underline{C}$}
    \AxiomC{$\Delta ; \Gamma | y : \underline{B} \vdash_c u : \underline{D}$}
    \RightLabel{}
    \BinaryInfC{$\Delta ; \Gamma ; \Theta ,\underline{R} : Rel_c[\underline{C},\underline{D}]\vdash_r (x : \underline{A}, y : \underline{B}). \underline{R}(t , u) : Rel_c[\underline{A},\underline{B}] $}
\end{prooftree}

\blue{What is the intuition here? This rule is in figure 5 of the PE logic paper.}
\begin{prooftree}
    \AxiomC{$\Delta ; \Gamma ;\Theta \vdash_p \phi$}
    \AxiomC{$\Delta ; \Gamma ;\Theta \vdash_r (x : A, y : B). \psi : Rel_v{A,B}$}
    \RightLabel{also $c$ version}
    \BinaryInfC{$\Delta ; \Gamma ; \Theta \vdash_r (x : A , y : B). \phi \implies \psi : Rel_v[A,B]$}
\end{prooftree}

\blue{?}
\begin{prooftree}
    \AxiomC{$\Delta ; \Gamma ;\Theta \vdash ?$}
    \AxiomC{$\Delta ; \Gamma ;\Theta \vdash ?$}
    \RightLabel{also $c$ version}
    \BinaryInfC{$\Delta ; \Gamma ; \Theta \vdash_r (x : A , y : B). \phi \land \psi : Rel_v[A,B]$}
\end{prooftree}

\begin{prooftree}
    \AxiomC{$\Delta ; \Gamma , z : C ;\Theta \vdash_r (x : A , y : B). \phi : Rel_v[A,B] ?$}
    \RightLabel{also $c$ version}
    \UnaryInfC{$\Delta ; \Gamma ; \Theta \vdash_r (x : A , y : B). \forall (z : C) . \phi : Rel_v[A,B]$}
\end{prooftree}

\begin{prooftree}
    \AxiomC{$\Delta , X ; \Gamma ;\Theta \vdash_r (x : A , y : B). \phi : Rel_v[A,B] ?$}
    \RightLabel{also $c$ version}
    \UnaryInfC{$\Delta ; \Gamma ; \Theta \vdash_r (x : A , y : B). \forall X . \phi : Rel_v[A,B]$}
\end{prooftree}

\begin{prooftree}
    \AxiomC{$\Delta , X ; \Gamma ;\Theta , R : Rel_{\mathbf{n}}[C,D] \vdash_r (x : A , y : B). \phi : Rel_{\mathbf{m}}[A,B]$}
    \RightLabel{$\mathbf{n},\mathbf{m} \in \{v,c\}$}
    \UnaryInfC{$\Delta ; \Gamma ; \Theta \vdash_r (x : A , y : B). \forall (R : Rel_{\mathbf{n}}[C , D]). \phi : Rel_{\mathbf{m}}[A,B]$}
\end{prooftree}
Analogous versions for $\exists$ connectives.

\subsection{Deduction Rules}
The judgement for deduction sequence are of the form $\Delta ; \Gamma ; \Theta ; \Phi \vdash_d \psi$ 
where $\Delta$ is a type environment, $\Gamma$ is a term environment, $\Theta$ is a relation environment, 
$\Theta$ is a proposition environment, and $\psi$ is a proposition.
\blue{like term intro and elim, but without proof terms}
\blue{also for computations?}
\begin{prooftree}
    \AxiomC{$\Delta ; \Gamma \vdash_v t : A$}
    \RightLabel{refl}
    \UnaryInfC{$\Delta ; \Gamma ; \Theta ; \Phi \vdash_d t =_A t$}
\end{prooftree}

\begin{prooftree}
    \AxiomC{$\Delta ; \Gamma ; \Theta ; \Phi \vdash_d t =_A u $}
    \AxiomC{$\Delta ; \Gamma ; \Theta ; \Phi \vdash_d \phi [t/x]$}
    \RightLabel{subst}
    \BinaryInfC{$\Delta ; \Gamma ; \Theta ; \Phi \vdash_d \phi[u/x]$}
\end{prooftree}

\begin{prooftree}
    \AxiomC{$\Delta ; \Gamma ; \Theta ; \Phi , \phi \vdash_d \psi $}
    \RightLabel{$\implies$ Intro}
    \UnaryInfC{$\Delta ; \Gamma ; \Theta ; \Phi \vdash_d \phi \implies \psi $}
\end{prooftree}

\begin{prooftree}
    \AxiomC{$\Delta ; \Gamma ; \Theta ; \Phi \vdash_d \phi \implies \psi $}
    \AxiomC{$\Delta ; \Gamma ; \Theta ; \Phi \vdash_d \phi $}
    \RightLabel{$\implies$ Elim}
    \BinaryInfC{$\Delta ; \Gamma ; \Theta ; \Phi \vdash_d \psi $}
\end{prooftree}
and familiar rules for logical and ($\land$)

\begin{prooftree}
    \AxiomC{$\Delta ; \Gamma ,(x : A) ; \Theta ; \Phi \vdash_d \phi $}
    \RightLabel{$\forall$ Term Intro}
    \UnaryInfC{$\Delta ; \Gamma ; \Theta ; \Phi \vdash_d \forall (x : A) . \phi$}
\end{prooftree}

\begin{prooftree}
    \AxiomC{$\Delta ; \Gamma ; \Theta ; \Phi \vdash_d \forall (x : A) . \phi$}
    \AxiomC{$\Delta ; \Gamma \vdash_v t : A$}
    \RightLabel{$\forall$ Term Elim}
    \BinaryInfC{$\Delta ; \Gamma ; \Theta ; \Phi \vdash_d \phi[t/x] $}
\end{prooftree}

\begin{prooftree}
    \AxiomC{$\Delta ,X ; \Gamma  ; \Theta ; \Phi \vdash_d \phi $}
    \LeftLabel{$X \notin FV(..)$, also for $c$}
    \RightLabel{$\forall$ Type Intro}
    \UnaryInfC{$\Delta ; \Gamma ; \Theta ; \Phi \vdash_d \forall X . \phi$}
\end{prooftree}

\begin{prooftree}
    \AxiomC{$\Delta ; \Gamma  ; \Theta ; \Phi \vdash_d \forall X . \phi$}
    \AxiomC{$\Delta \vdash A$}
    \LeftLabel{also for $c$}
    \RightLabel{$\forall$ Type Elim}
    \BinaryInfC{$\Delta ; \Gamma ; \Theta ; \Phi \vdash_d \phi[A/X]$}
\end{prooftree}

\begin{prooftree}
    \AxiomC{$\Delta ; \Gamma ; \Theta , R: Rel_v[A , B]; \Phi \vdash_d  \phi$}
    \LeftLabel{also for $c$}
    \RightLabel{$\forall$ Rel Intro}
    \UnaryInfC{$\Delta ; \Gamma ; \Theta ; \Phi \vdash_d \forall (R : Rel_v[A , B]) . \phi$}
\end{prooftree}

\begin{prooftree}
    \AxiomC{$\Delta ; \Gamma ; \Theta ; \Phi \vdash_d \forall(R : Rel_v[A,B]).\phi $}
    \AxiomC{$\Delta ; \Gamma ; \Theta \vdash_r (x : A, y : B).\psi : Rel_v[A,B]$}
    \LeftLabel{also for $c$}
    \RightLabel{$\forall$ Rel Elim}
    \BinaryInfC{$\Delta ; \Gamma ; \Theta ; \Phi \vdash_d \phi[\psi[t/x,u/y]/R(t,u)]$}
\end{prooftree}

\subsection{Axioms \& Axiom Schemas}
\subsubsection{Congruences}

\begin{prooftree}
    \AxiomC{$\Delta ; \Gamma , (x : A) \vdash_c t , u : \underline{B} $}
    \AxiomC{$\Delta ; \Gamma , (x : A); \Theta ; \Phi \vdash_d t = u$}
    \RightLabel{$\lambda$ cong, $x \notin FV(\Phi)$}
    \BinaryInfC{$\Delta ; \Gamma ; \Theta ; \Phi \vdash_d (\lambda (x : A). t) =_{A \rightarrow \underline{B}} (\lambda (x : A). u) $}
\end{prooftree}

\subsubsection{Beta / Eta Laws}

\begin{prooftree}
    \AxiomC{}
    \RightLabel{$\lambda \beta$}
    \UnaryInfC{$\Delta ; \Gamma ; \Theta ; \Phi \vdash_d \forall (u : A). ((\lambda x : A . t)u =_B t[u/x])$}
\end{prooftree}


\begin{prooftree}
    \AxiomC{}
    \RightLabel{$\lambda \eta$}
    \UnaryInfC{$\Delta ; \Gamma ; \Theta ; \Phi \vdash_d \forall X. \forall Y. \forall (f : X \rightarrow Y). ((\lambda x: X . f x) =_{X\rightarrow Y}f)$}
\end{prooftree}

\subsubsection{Parametricity}

\subsection{Relational interpretation of Types}
Let $\bm{X}$ and $\bm{\underline{X}}$ be vectors of value type and computation type variables of length $n$.
Let $\bm{\rho}$ be a vector of value relations $\Delta ; \Gamma ; \Theta \vdash_r \rho_i : Rel_v[C_i , C'_i]$ for all $i \in {1..n}$.
Let $\bm{\underline{\rho}}$ be a vector of computation relations $\Delta ; \Gamma ; \Theta \vdash_r \underline{\rho_i} : Rel_c[\underline{C_i} , \underline{C'_i}]$ for all $i \in {1..n}$.
Let $A$ be a \textbf{value type} with $FTV(A) \in \{\bm{X},\bm{\underline{X}}\}$. Define:
\[
  A[\bm{\rho}/\bm{X},\bm{\underline{\rho}}/\bm{\underline{X}}] : Rel_v[A[\bm{C}/\bm{X},\bm{\underline{C}}/\bm{\underline{X}}], A[\bm{C'}/\bm{X},\bm{\underline{C'}/\bm{\underline{X}}}]] 
\] 
by induction on $A$. 
\blue{Note: PE Logic defines the relational interpretation of value types, 
but not computation types? }. 
It seems like this relation needs to be indexed by worlds to describe how Case A and OSum are related.

\begin{align*}
    X_i[\bm{\rho},\bm{\underline{\rho}}] &= \rho_i\\
    \textrm{Unit} [\bm{\rho},\bm{\underline{\rho}}]&= (x : Unit , y : Unit). x =_{Unit} y\\
    \textrm{Case} A [\bm{\rho},\bm{\underline{\rho}}]&= (x : \textrm{Case } (A[\bm{C}/\bm{X},\bm{\underline{C}}/\bm{\underline{X}}]),
    y : \textrm{Case } (A[\bm{C'}/\bm{X},\bm{\underline{C'}}/\bm{\underline{X}}])). 
    \\
    & foo\\
    \textrm{OSum} [\bm{\rho},\bm{\underline{\rho}}]&= \textrm{same injection, related values}\\
    A \times A' [\bm{\rho},\bm{\underline{\rho}}]&= ((x,y) : A\times A'[\bm{C}/\bm{X},\bm{\underline{C}}/\bm{\underline{X}}], (x',y') :  A \times A'[\bm{C'}/\bm{X},\bm{\underline{C'}/\bm{\underline{X}}}]).\\
    & A[\bm{\rho},\bm{\underline{\rho}}](x,x')\land A'[\bm{\rho},\bm{\underline{\rho}}](y,y') \\
    A * A' [\bm{\rho},\bm{\underline{\rho}}]&= \textrm{same as product.. but fiddle with world index?}\\
    \exists X . A [\bm{\rho},\bm{\underline{\rho}}]&= standard \\
    U \underline{B} [\bm{\rho},\bm{\underline{\rho}}]&= \textrm{related thunks?}\\
    (A \sep \underline{B})[\bm{\rho},\bm{\underline{\rho}}]&= \textrm{related inputs to related outputs?}\\
\end{align*}


\subsection{Example Derivations}

\subsubsection{Equality Reasoning}
\paragraph{Transitivity}
\begin{prooftree}
    \AxiomC{}
    \UnaryInfC{$...;(x =_X y) \land (y =_X z ) \vdash_d (x =_X y) \land (y =_X z ) $}
    \UnaryInfC{$...;(x =_X y) \land (y =_X z ) \vdash_d y =_X z$}

    \AxiomC{}
    \UnaryInfC{$...; (x =_X y) \land (y =_X z ) \vdash_d (x =_X y) \land (y =_X z ) $}
    \UnaryInfC{$...;(x =_X y) \land (y =_X z ) \vdash_d (x =_X y)$}
    \UnaryInfC{$...; (x =_X y) \land (y =_X z ) \vdash_d (x =_X y)[y/y]$}
    
    \RightLabel{subst}
    \BinaryInfC{$X ; x,y,z ; \cdot ; (x =_X y) \land (y =_X z ) \vdash_d (x =_X y)[z/y]$}
    \UnaryInfC{$X ; x,y,z ; \cdot ; (x =_X y) \land (y =_X z ) \vdash_d (x =_X z)$}
    \UnaryInfC{$\vdash_d \forall X . \forall (x y z : X). (x =_X y) \land (y =_X z) \implies (x =_X z)$}
\end{prooftree}

\paragraph{Symmetry}
Use IdExt and opRel?

\subsubsection{Extensionality}

\begin{prooftree}
    \AxiomC{}
    \RightLabel{in $\Phi$}
    \UnaryInfC{$X ,\underline{Y} ; f,g,x ; \cdot ; (\forall(x : X). f x =_{\underline{Y}} g x)  \vdash_d \forall (x : X).(f x =_{\underline{Y}} g x) $}
    
    \AxiomC{}
    \RightLabel{Var}
    \UnaryInfC{$...\vdash_v x : X$}
    
    \RightLabel{$\forall$ Elim}
    \BinaryInfC{$X ,\underline{Y} ; f,g,x ; \cdot ; (\forall(x : X). f x =_{\underline{Y}} g x)  \vdash_d (f x =_{\underline{Y}} g x) $}
    \RightLabel{$\lambda$ cong}
    \UnaryInfC{$X ,\underline{Y}; f,g ; \cdot ; (\forall(x : X). f x =_{\underline{Y}} g x)  \vdash_d ((\lambda x : X. f x) =_{X \rightarrow \underline{Y}} (\lambda x : X .g x)) $}
    \RightLabel{$\lambda \eta$}
    \UnaryInfC{$X ,\underline{Y} ; f,g ; \cdot ; (\forall(x : X). f x =_{\underline{Y}} g x)  \vdash_d (f =_{X \rightarrow \underline{Y}} g) $}
    \RightLabel{Intros}
    \UnaryInfC{$X,\underline{Y} ; f,g ; \cdot ; \cdot \vdash_d ((\forall(x : X). f x =_{\underline{Y}} g x) \implies f =_{X \rightarrow \underline{Y}} g) $}
    \RightLabel{Intros}
    \UnaryInfC{$...\vdash_d \forall (f , g: X \rightarrow \underline{Y}).((\forall(x : X). f x =_{\underline{Y}} g x) \implies f =_{X \rightarrow \underline{Y}} g) $}
\end{prooftree}
\subsubsection{Identity Extension Lemma}
By Induction on Types.
\paragraph{For Unit}
 
Recall 
\begin{align*}
    eq_A := (x : A, y : A). x =_A y
\end{align*}
and substitution of a relation into a base type is just the relational interpretation of the base type.
\begin{align*}
    Unit[eq_{Unit}] = \llbracket Unit \rrbracket_{Rel} = \{(tt,tt)\}
\end{align*}

\blue{What rules are missing here to make this proof go through?}
\begin{prooftree}
    \AxiomC{}
    \RightLabel{Refl}
    \UnaryInfC{$...; \Phi , \{u=tt,v=tt\} \vdash_d tt =_{Unit} tt$}
    \RightLabel{Unit $\eta$ ?}
    \UnaryInfC{$...; \Phi ,u(\llbracket Unit \rrbracket_{Rel})v \vdash_d u =_{Unit} v$}
    \RightLabel{Intro}
    \UnaryInfC{$... \vdash_d u(\llbracket Unit \rrbracket_{Rel})v \implies u =_{Unit} v$}

    \AxiomC{}
    \RightLabel{By Def}
    \UnaryInfC{$...; \Phi , tt =_{Unit} tt \vdash_d tt(\llbracket Unit \rrbracket_{Rel})tt$}
    \RightLabel{Unit $\eta$?}
    \UnaryInfC{$...; \Phi , u =_{Unit} v \vdash_d u(\llbracket Unit \rrbracket_{Rel})v$}
    \RightLabel{Intro}
    \UnaryInfC{$... \vdash_d u =_{Unit} v \implies u(\llbracket Unit \rrbracket_{Rel})v$}

    \RightLabel{Logical Equiv}
    \BinaryInfC{$\cdot ; u : Unit , v : Unit ; \cdot ; \cdot \vdash_d u(\llbracket Unit \rrbracket_{Rel})v \equiv u =_{Unit} v$}
    \RightLabel{Intros}
    
    \UnaryInfC{$\cdot ; \cdot ; \cdot ; \cdot \vdash_d \forall u : Unit , v : Unit. u(\llbracket Unit \rrbracket_{Rel})v \equiv u =_{Unit} v$}
    \RightLabel{Relational Subst}
    \UnaryInfC{$\cdot ; \cdot ; \cdot ; \cdot \vdash_d \forall u : Unit , v : Unit. u(Unit[eq_{Unit}])v \equiv u =_{Unit} v$}

\end{prooftree}

\paragraph{For $X \rightarrow \underline{Y}$}
Recall
\begin{align*}
    (X \rightarrow \underline{Y})[eq_X , eq_{\underline{Y}}] &= 
    (f : (X \rightarrow \underline{Y}) g : (X \rightarrow \underline{Y})).\\
    &\forall (x : X). \forall (x' : X). \\
    & x (eq_X) x' \implies (f x)(eq_Y)(g x')
\end{align*}
One direction. 
\begin{prooftree}
    
    \AxiomC{}
    \RightLabel{in $\Phi$} 
    \UnaryInfC{(the relation in above on $f,g$)}

    \AxiomC{}
    \RightLabel{Var}
    \UnaryInfC{$..\vdash_v x : X$}

    \RightLabel{$\forall$ Elim}
    \BinaryInfC{$..\vdash_d \forall (x' : X). x (eq_X) x' \implies fx =_{\underline{Y}} gx' $}

    \AxiomC{}
    \RightLabel{Var}
    \UnaryInfC{$..\vdash_v x : X$}

    \RightLabel{$\forall$ Elim}
    \BinaryInfC{$..;f((X \rightarrow \underline{Y})[eq_X , eq_{\underline{Y}}])g  \vdash_d x (eq_X) x \implies fx =_{\underline{Y}} gx $}

    \AxiomC{}
    \RightLabel{Refl}
    \UnaryInfC{$\vdash_d x =_X x$}
    \RightLabel{IdExt}
    \UnaryInfC{$\vdash_d x (eq_X) x$}
    \RightLabel{$\implies$ Elim}

    \BinaryInfC{$..;f,g,x;..;f((X \rightarrow \underline{Y})[eq_X , eq_{\underline{Y}}])g  \vdash_d  fx (eq_{\underline{Y}}) gx $}
    \RightLabel{IdExt}
    \UnaryInfC{$..;f,g,x;..;f((X \rightarrow \underline{Y})[eq_X , eq_{\underline{Y}}])g  \vdash_d  fx =_{\underline{Y}} gx $}
    \RightLabel{$\lambda$ cong}
    \UnaryInfC{$..;f,g;..;f((X \rightarrow \underline{Y})[eq_X , eq_{\underline{Y}}])g  \vdash_d  (\lambda x : X .fx) =_{X \rightarrow \underline{Y}} (\lambda x: X . gx) $}
    \RightLabel{$\lambda \eta$}
    \UnaryInfC{$..;f,g;..;f((X \rightarrow \underline{Y})[eq_X , eq_{\underline{Y}}])g  \vdash_d  f =_{X \rightarrow \underline{Y}} g $}
    \RightLabel{Intros}
    \UnaryInfC{$..;f,g;.. \vdash_d f((X \rightarrow \underline{Y})[eq_X , eq_{\underline{Y}}])g \implies f =_{X \rightarrow \underline{Y}} g $}
    \RightLabel{Intros}
    \UnaryInfC{$... \vdash_d \forall (f , g : X \rightarrow \underline{Y}). f((X \rightarrow \underline{Y})[eq_X , eq_{\underline{Y}}])g \implies f =_{X \rightarrow \underline{Y}} g $}
\end{prooftree}

\subsubsection{Identity Function}

\begin{prooftree}
    \AxiomC{?}
    \UnaryInfC{$X ; f : \forall X . X \rightarrow F X, x : X ; \cdot ; \cdot \vdash_d f[X]x =_{FX} ret \;x$}
    \RightLabel{Intro}
    \UnaryInfC{$\cdot ; f : \forall X . X \rightarrow F X ; \cdot ; \cdot \vdash_d \forall X . \forall (x : X). f[X]x =_{FX} ret \;x$}
\end{prooftree}


\subsubsection{Church Encodings}
We can only church encode the F type.
\begin{comment}
    
\subsubsection{Church Encodings}
\paragraph{Unit}
Forget the built in Unit type for now. Note that we only have the compuational function type in this language.
\begin{align*}
    &Unit := \forall X . X \rightarrow F X\\
    &\mathbf{1} : Unit\\
    &\mathbf{1} = \Lambda X . \lambda (x : X) . \textrm{ ret } x\\
\end{align*}
Given 
\begin{align*}
    &f : \forall X . X \rightarrow F X\\
\end{align*}
by $\eta$
\begin{align*}
    &f = \Lambda X . \lambda (x : X). f[X] x\\
\end{align*}
Pick a relation on $FX$,
\begin{align*}
    &R := X ; x : X ; \cdot \vdash_r (a : F X , b : F X). \textrm{ ret } x =_{FX} b\\
\end{align*}
What does parametricity say for the type $\forall X . X \rightarrow F X$?
\begin{align*}
    &\forall (t : (\forall X, X \rightarrow FX)).\forall Y, Z, R : Rel_v[Y,Z].t[Y]((X \rightarrow F X)[R])t[Z]\\
\end{align*}
$((X \rightarrow F X)[R])$ Substitute the relation in for type variables.
\begin{align*}
    &(X \rightarrow F X)[R] =\\
    &= X[R] \rightarrow (FX)[R]\\
    &= X[R] \rightarrow \red{?}\\
\end{align*}
\begin{prooftree}
    \AxiomC{?}
    \UnaryInfC{$Unit,X ;\mathbf{1},f,x;\cdot;\cdot \vdash_d \forall X .\forall (x : X). \textrm{ ret } x =_{F X} f [X]x$}
    
    \AxiomC{}
    \UnaryInfC{$Unit , X \vdash X$}

    \RightLabel{$\forall$ Elim}
    \BinaryInfC{$Unit,X ;\mathbf{1},f,x;\cdot;\cdot \vdash_d \forall (x : X). \textrm{ ret } x =_{F X} f [X]x$}

    \AxiomC{}
    \RightLabel{Var}
    \UnaryInfC{$..\vdash_v x : X$}
    \RightLabel{$\forall$ Elim}

    \BinaryInfC{$Unit,X ;\mathbf{1},f,x;\cdot;\cdot \vdash_d \textrm{ ret } x =_{F X} f [X]x$}
    \RightLabel{cong}
    \UnaryInfC{$Unit ;\mathbf{1},f;\cdot;\cdot \vdash_d \Lambda X . \lambda (x : X) . \textrm{ ret } x =_{Unit} \Lambda X . \lambda (x : X). f [X]x$}
    \RightLabel{Def, $\eta$}
    \UnaryInfC{$Unit ;\mathbf{1},f;\cdot;\cdot \vdash_d \mathbf{1} =_{Unit} f$}
    \RightLabel{Intro}
    \UnaryInfC{$Unit ;\mathbf{1};\cdot;\cdot \vdash_d \forall (f : Unit). \mathbf{1} =_{Unit} f$}
\end{prooftree}
\end{comment}

\subsubsection{Practice}
\paragraph{Unit}
Practice parametricity proofs for System F.
Let 
\[
    Unit := \forall X . X \rightarrow X    
\]
and 
\[
    tt := \Lambda X . \lambda (x : X) . x  
\]
Prove 
\[
    \vdash_d : \forall (u : Unit). tt = u
\]

By the parametricity schema for type Unit, we have 
\begin{align*}
    & \forall (p : (\forall X. X \rightarrow X)). \forall Y. \forall Z. \forall (R : Rel[Y,Z]). ((X \rightarrow X)[R])(p[Y],p[Z]) \\
\end{align*}
Specialized to $u, X, X$, we have 
\[
    \forall (R : Rel[X,X]). ((X \rightarrow X)[R])(u[X],u[X])
\]
which \textit{evaluates} to 
\[
    \forall (R : Rel[X,X]).\forall (y : X). \forall (z : X). R(y,z) \implies R(u[X]y,u[X]z)
\]
choosing $R$ to be $X ; x : X ; \cdot \vdash_r (y : X , z : X). z = x$, we use $\forall$Rel Elim to get
\begin{prooftree}
    \AxiomC{}
    \UnaryInfC{$\forall (R : Rel[X,X]).\forall (y : X). \forall (z : X). R(y,z) \implies R(u[X]y,u[X]z)$}

    \AxiomC{}
    \UnaryInfC{$X ; x : X ; \cdot \vdash_r (y : X , z : X). z = x$}

    \BinaryInfC{$\vdash_d \forall (y : X). \forall (z : X). z = x \implies u[X]z = x$}
\end{prooftree}
specializing $y,z$ to $x,x$
\[
    x = x \implies u[X]x = x  
\]
which we can apply to refl to yield 
\[
    u[X]x = x 
\]


\begin{prooftree}
    \AxiomC{$\rotatebox{90}{\ldots}$}
    \UnaryInfC{$X ; u : Unit, x : X \vdash_d x = u[X](x)$}
    \RightLabel{$\lambda$ cong}
    \UnaryInfC{$X ; u : Unit \vdash_d \lambda (x : X).x =  \lambda (x : X). u[X](x)$}
    \RightLabel{$\Lambda$ cong}
    \UnaryInfC{$u : Unit \vdash_d \Lambda X . \lambda (x : X).x = \Lambda X . \lambda (x : X). u[X](x)$}
    \RightLabel{Eta}
    \UnaryInfC{$u : Unit \vdash_d tt =_{\forall X . X \rightarrow X} u$}
    \RightLabel{Intros}
    \UnaryInfC{$\vdash_d : \forall (u : Unit). tt =_{\forall X . X \rightarrow X}  u$}
\end{prooftree}


\subsubsection{OSum Free Theorems}
Try to prove 
\begin{align*}
    \vdash_d & \forall A \;B. \\
    & \forall (y : \textrm{OSum}). \\
    & \forall (f : U(\forall X . \textrm{Case }X \sep \textrm{OSum} \rightarrow F X)). \\
    & \forall (k \; k' : U(A \rightarrow B)). \\
    & \textrm{newcase}_A \sigma ; x \leftarrow (!f)[A]\sigma y ; (!k) x  \\
    & = \\
    & \textrm{newcase}_A \sigma ; x \leftarrow (!f)[A]\sigma y ; (!k') x  
\end{align*}   

The intuition is?: Any term $f$ can't use the fresh case to extract a value from OSum. 
So $f$ is effectively a term of type $U (\forall X . F X)$? 
After you force $f$ and apply it to $A$, $(!f)[A]$, you get an effectful value $x$ of $A$. 
Then for any two possible continuations of type $A \rightarrow B$, applying the continuations to $x$ yields related results?

\begin{prooftree}
    \AxiomC{}
    \UnaryInfC{$...\vdash_d \textrm{newcase}_A \sigma ; x \leftarrow (!f)[A]\sigma y ; (!k) x = \textrm{newcase}_A \sigma ; x \leftarrow (!f)[A]\sigma y ; (!k') x$}
    \RightLabel{Intros}
    \UnaryInfC{$\vdash_d \forall ... \textrm{newcase}_A \sigma ; x \leftarrow (!f)[A]\sigma y ; (!k) x = \textrm{newcase}_A \sigma ; x \leftarrow (!f)[A]\sigma y ; (!k') x$}
\end{prooftree}

\subsubsection{cruft}
Looking for free theorems for types containing our separating connectives, OSum, and Case.
From the gradual parametricity paper: given
\[
    \vdash M : \forall ^{\nu} X . ? \rightarrow X 
\]
and $\vdash V :\; ?$ then
\[
  \textrm{unseal}_X (M \{X \cong A \} V) \textrm{true}  
\]
either diverges or errors. 
\\
In CBPV OSum:
\[
  \vdash_c M : \forall X . \textrm{Case} X \sep (\textrm{OSum} \rightarrow F X)  
\]
should be uninhabited. If error $\mho$ was added to the language,
then 
\[
  A ; \sigma : \textrm{Case} A * d : \textrm{OSum} \vdash_c (M[A]@ \sigma)d : F X 
\]
should always error.

\[
  A ; \sigma : \textrm{Case} A * d : \textrm{OSum} ; \cdot ; \cdot \vdash_d 
  \forall (M : X . \textrm{Case} X \sep (\textrm{OSum} \rightarrow F X)).  (M[A]@ \sigma)d =_{F X}  \mho 
\]
Seems like we have to state this property with the context loaded since we dont have a freshenss quantifier.
Would we want something like?:
\[
    \vdash_d \forall A . \fresh (\sigma :\textrm{Case A}).\forall (d : \textrm{OSum})
    \forall (M : X . \textrm{Case} X \sep (\textrm{OSum} \rightarrow F X)). 
    (M[A]@ \sigma)d =_{F X}  \mho 
\]
in Fresh Logic [CITE], Gabbay decomposes $\fresh$ into 
\[
    \fresh a . \phi(a) := \exists S \in \textrm{Fin}\mathbf{A}. \forall a \notin S. \phi(a)    
\]

\begin{comment}
What can we say about:  
\[
    \forall X. \textrm{OSum} \rightarrow (\textrm{Case}X \sep F \;\textrm{OSum})
\]
This should only be able to return the original OSum?
It would need a value of type X in addition to a Case X to construct a new value.
What about getting a value from OSum?
\[
  \textrm{OSum} \rightarrow (\forall X . \textrm{Case} X \sep X)
\]
or
\[
  \forall X . \textrm{OSum} \rightarrow ( \textrm{Case} X \sep X)
\]
This type should be uninhabited since a fresh case cannot be used to get a value of X 
since OSum could not have used that case in an injection.
\blue{Both of the above are not well-formed types since $\forall$ quantifies over value types, missing an $F$ }
What about 

\[
  \forall X . \textrm{Case X} \rightarrow (\textrm{OSum}  \sep F X)  
\]
This can be inhabited.. but there are values of type $X$ which are "inaccessible".
\end{comment}


\end{document}