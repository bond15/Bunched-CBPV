\documentclass{article}
\usepackage{graphicx} % Required for inserting images
\usepackage{simplebnf}
\usepackage{bussproofs}
\usepackage{amsmath}
\usepackage{amssymb}
\usepackage[dvipsnames]{xcolor}
\usepackage{stmaryrd}
\usepackage{comment}
\usepackage{bm}

\begin{document}
%\usepackage{xcolor}

\newcommand{\den}[1]{\llbracket #1 \rrbracket}
\newcommand{\blue}[1]{\textcolor{blue}{#1}}
\newcommand{\red}[1]{\textcolor{red}{#1}}
\newcommand{\sep}{\mathrel{-\mkern-6mu*}}
\newcommand{\thunk}[1]{\textrm{thunk }#1}
\newcommand{\injj}[2]{\textrm{inj}_{#1}#2}
\newcommand{\err}{\mho}
\newcommand{\print}[1]{\textrm{print }#1}
\newcommand{\force}[1]{\textrm{force }#1}
\newcommand{\ret}[1]{\textrm{ret }#1}
\newcommand{\bind}[3]{#1 \leftarrow #2 ; #3}
\newcommand{\newcase}[3]{\textrm{newcase}_{#1} \; #2 ; #3}
\newcommand{\match}[5]{\textrm{match }#1 \textrm{ with }#2 \;\{#3 . #4 | #5\}}
\newcommand{\unpack}[4]{\textrm{unpack }(#1,#2) = #3 ; #4}
\newcommand{\lett}[4]{\textrm{let }(#1,#2) = #3 ; #4}
\newcommand{\lets}[4]{\textrm{let }(#1*#2) = #3 ; #4}
\newcommand{\ite}[3]{\textrm{if }#1 \textrm{ then }#2 \textrm{ else }#3}
\newcommand{\at}{\textrm{@}}
\newcommand{\ttt}{\textrm{tt}}
\newcommand{\tru}{\textrm{true}}
\newcommand{\fsl}{\textrm{false}}
\newcommand{\pworld}{\widehat{\mathbf{World}}}
\newcommand{\world}{{\mathbf{World}}}
\newcommand{\calculus}{\operatorname{-calculus}}

\section{Object Language}
\subsection{Raw Terms}
The untyped syntax for CBPV OSum, add separating connectives, remove error. 
Add $\forall \underline{X} . \underline{B}$ ?

%CBPV OSum%
\begin{bnf}
    $A$ : Value Types ::= $X$
    | Unit
    | Case $A$
    | OSum
    | $A \times A$
    | $A * A$
    | $\exists X . A$
    | $U \underline{B}$
    ;;
    $\underline{B}$ : Computation Types ::= $A \rightarrow \underline{B}$
    | $A \sep \underline{B}$
    | $\forall X . \underline{B}$
    | $F A$
    ;;
    $V$ : Values ::= $x$
    | $tt$
    | $\sigma$
    | $\textrm{inj}_V V$
    | $(V , V)$
    | $(V * V)$
    | pack $(A , V)$ as $\exists X .A$
    | thunk $M$
    ;;
    $M$ : Computations ::= 
     $\lambda x \colon A . M$
    | $M V$
    | $\alpha x \colon A . M$ 
    | $M @ V$
    | $\Lambda X . M$
    | $M[A]$
    | ret $V$
    | $x \leftarrow M ; N$
    | force $V$
    | newcase$_A x ; M$
    | match V with V \{ inj $x . M \| N$ \}
    | let ($x , x$) = $V ; M$
    | let ($x * x$) = $V ; M$
    | unpack $(X , x) = V ; M$
    ;;
    $\Gamma$ : Value Context ::= $\cdot$
    | $\Gamma , x \colon A$ 
    | $\Gamma * x \colon A$
    ;;
    $\Xi$ : Type Context ::= $\cdot$
    | $\Xi , X$
    ;;
\end{bnf}
\subsection{Typed Terms}

\begin{prooftree}
    \AxiomC{}
    \RightLabel{}
    \UnaryInfC{$\Xi ; \Gamma, x \colon A \vdash_v x : A$}
\end{prooftree}

\begin{prooftree}
    \AxiomC{}
    \RightLabel{}
    \UnaryInfC{$\Xi ; \Gamma * x \colon A  \vdash_v x : A$}
\end{prooftree}

\begin{prooftree}
    \AxiomC{}
    \RightLabel{}
    \UnaryInfC{$\Xi ; \Gamma \vdash_v \ttt : \textrm{Unit}$}
\end{prooftree}


\begin{prooftree}
    \AxiomC{$\Xi ; \Gamma \vdash_v \sigma : \textrm{Case} A$}
    \AxiomC{$\Xi ; \Gamma \vdash_v V : A$}
    \RightLabel{}
    \BinaryInfC{$\Xi ; \Gamma \vdash_v \injj{\sigma}{V}: \textrm{OSum}$}
\end{prooftree}

\begin{prooftree}
    \AxiomC{$\Xi ; \Gamma \vdash_v V_1 : A_1$}
    \AxiomC{$\Xi ; \Gamma \vdash_v V_2 : A_2$}
    \RightLabel{}
    \BinaryInfC{$\Xi ; \Gamma \vdash_v (V_1,V_2) : A_1 \times A_2$}
\end{prooftree}

\begin{prooftree}
    \AxiomC{$\Xi ; \Gamma_1 \vdash_v V_1 : A_1$}
    \AxiomC{$\Xi ; \Gamma_2 \vdash_v V_2 : A_2$}
    \RightLabel{}
    \BinaryInfC{$\Xi ; \Gamma_1 * \Gamma_2 \vdash_v (V_1 * V_2) : A_1 * A_2$}
\end{prooftree}

\begin{prooftree}
    \AxiomC{$\Xi ; \Gamma \vdash_v V : A[A'/X]$}
    \RightLabel{}
    \UnaryInfC{$\Xi ; \Gamma \vdash_v \textrm{pack}(A',V) \textrm{ as } \exists X .A : \exists X .A$}
\end{prooftree}

\begin{prooftree}
    \AxiomC{$\Xi ; \Gamma \vdash_c M : \underline{B} $}
    \RightLabel{}
    \UnaryInfC{$\Xi ; \Gamma \vdash_v \textrm{thunk } M : U \underline{B}$}
\end{prooftree}

\begin{prooftree}
    \AxiomC{$\Xi ; \Gamma , x : A \vdash_c M : \underline{B}$}
    \RightLabel{}
    \UnaryInfC{$\Xi ; \Gamma \vdash _c  \lambda x : A . M : A \rightarrow \underline{B}$}
\end{prooftree}

\begin{prooftree}
    \AxiomC{$\Xi ; \Gamma \vdash_c M : A \rightarrow \underline{B}$}
    \AxiomC{$\Xi ; \Gamma \vdash_v N : A$}
    \RightLabel{}
    \BinaryInfC{$\Xi ; \Gamma \vdash_c M N : \underline{B}$}
\end{prooftree}

\begin{prooftree}
    \AxiomC{$\Xi ; \Gamma * x : A \vdash_c M : \underline{B}$}
    \RightLabel{}
    \UnaryInfC{$\Xi ; \Gamma \vdash _c  \alpha x : A . M : A \sep \underline{B}$}
\end{prooftree}

\begin{prooftree}
    \AxiomC{$\Xi ; \Gamma_1 \vdash_c M : A \sep \underline{B}$}
    \AxiomC{$\Xi ; \Gamma_2 \vdash_v N : A$}
    \RightLabel{}
    \BinaryInfC{$\Xi ; \Gamma_1 * \Gamma_2 \vdash_c M @ N : \underline{B}$}
\end{prooftree}

\begin{prooftree}
    \AxiomC{$\Xi , X ; \Gamma \vdash_c M : \underline{B}$}
    \RightLabel{}
    \UnaryInfC{$\Xi ; \Gamma \vdash_c \Lambda X . M : \forall X . \underline{B}$}
\end{prooftree}

\begin{prooftree}
    \AxiomC{$\Xi ; \Gamma \vdash_c : M : \forall X . \underline{B}$}
    \AxiomC{$\Xi \vdash A$}
    \RightLabel{}
    \BinaryInfC{$\Xi ; \Gamma \vdash_c M[A] : \underline{B}[A / X]$}
\end{prooftree}

\begin{prooftree}
    \AxiomC{$\Xi ; \Gamma \vdash_v V : A$}
    \RightLabel{}
    \UnaryInfC{$\Xi ; \Gamma \vdash _c \textrm{ret } V : F A$}
\end{prooftree}

\begin{prooftree}
    \AxiomC{$\Xi ; \Gamma \vdash_c M : F A $}
    \AxiomC{$\Xi ; \Gamma , x : A \vdash_c N : \underline{B}$}
    \BinaryInfC{$\Xi ; \Gamma \vdash_c x \leftarrow M ; N : \underline{B}$}
\end{prooftree}

\begin{prooftree}
    \AxiomC{$\Xi ; \Gamma \vdash_v V : U \underline{B}$}
    \RightLabel{}
    \UnaryInfC{$\Xi ; \Gamma \vdash_c : \textrm{force } V : \underline{B}$}
\end{prooftree}

\begin{prooftree}
    \AxiomC{$\Xi ; \Gamma * (\sigma : \textrm{Case}A ) \vdash_c M : \underline{B}$}
    \AxiomC{$\Xi \vdash A$}
    \RightLabel{}
    \BinaryInfC{$\Xi ; \Gamma \vdash_c \textrm{newcase}_{A} x ; M : \underline{B}$}
\end{prooftree}

\begin{prooftree}
    \AxiomC{$\Xi ; \Gamma \vdash_v V : \textrm{OSum}$}
    \AxiomC{$\Xi ; \Gamma \vdash_v \sigma : \textrm{Case }A$}
    \AxiomC{$\Xi ; \Gamma , x : A \vdash M : \underline{B}$}
    \AxiomC{$\Xi ; \Gamma \vdash_c N : \underline{B}$}
    \RightLabel{}
    \QuaternaryInfC{$\Xi ; \Gamma \vdash_c \textrm{match } V \textrm{ with } \sigma \{ \textrm{ inj }x. M \| N\} : \underline{B}$}
\end{prooftree}

\begin{prooftree}
    \AxiomC{$\Xi ; \Gamma \vdash_v V : A_1 \times A_2$}
    \AxiomC{$\Xi ; \Gamma , x : A_1 , y : A_2 \vdash_c M : \underline{B}$}
    \RightLabel{}
    \BinaryInfC{$\Xi ; \Gamma \vdash_c$ let $(x,y) = V; M : \underline{B}$}
\end{prooftree}

\begin{prooftree}
    \AxiomC{$\Xi ; \Gamma \vdash_v V : A_1 * A_2$}
    \AxiomC{$\Xi ; \Gamma * x : A_1 * y : A_2 \vdash_c M : \underline{B}$}
    \RightLabel{}
    \BinaryInfC{$\Xi ; \Gamma \vdash_c$ let $(x*y) = V; M : \underline{B}$}
\end{prooftree}


\begin{prooftree}
    \AxiomC{$\Xi \vdash \underline{B}$}
    \AxiomC{$\Xi ; \Gamma \vdash_v V : \exists X . A$}
    \AxiomC{$\Xi , X ; \Gamma , x : A \vdash_c M : \underline{B}$}
    \RightLabel{}
    \TrinaryInfC{$\Xi ; \Gamma \vdash_c : \textrm {unpack} (X , x) = V ; M : \underline{B}$}
\end{prooftree}

\section{Logic}
\subsection{Judgments}
\begin{comment}
    
    Q:
    relation contexts contain value and computation contexts?
    how then are value vs computation propositions indexed?
    \\
    value formation rules involve 
    
    \begin{itemize}
        \item type context
        \item term context
        \item value relation context?
    \end{itemize}

    computation formation rules involve 
\begin{itemize}
    \item type context
    \item term context
    \item stoup
    \item value relation context?
    \item computation relation context?
    \item 
\end{itemize}
\end{comment}
The relation environment, $\Theta$ in PE logic contains both value and computation relations. 
How does this work in the semantics when value relations are denoted as objects of $Sub_{\mathcal{V}}(A \times B)$ for $Rel_{\mathcal{V}}[A,B]$
and computation relations are denoted as objects of $Sub_{\mathcal{C}}(\underline{A} \times \underline{B})$ for $Rel_{\mathcal{V}}[\underline{A},\underline{B}]$?
Maybe we have separate relation environments?
\begin{align*}
    &\Xi ; \Gamma ; \Theta \vdash \phi \textrm{ VProp}  \\
    &\Xi ; \Gamma ; \Theta \vdash (x : A , y : B). \phi : Rel_{\mathcal{V}}[A,B]\\
    &\Xi ; \Gamma ; \Theta | \Phi \vdash \phi\\
    &\Xi ; \Gamma ; \Delta ; \Theta ; \blue{\Omega} \vdash \underline{\phi} \textrm{ CProp}  \\
    &\Xi ; \Gamma ; \Delta ; \Theta ; \blue{\Omega} \vdash (x : \underline{A}, y : \underline{B}). \underline{\phi} : Rel_{\mathcal{C}}[\underline{A},\underline{B}]  \\
    &\Xi ; \Gamma ; \Delta ; \Theta ; \blue{\Omega} | \Phi ; \underline{\Psi} \vdash \underline{\phi} \\
\end{align*}
for type environment $\Xi$, term environment $\Gamma$, stoup $\Delta$, value relation environment $\Theta$,
computation relation environment $\Omega$, value proposition environment $\Phi$, and computation proposition environment $\underline{\Psi}$.

\subsection{Formation Rules}

\subsubsection{Value Propositions}
\[
    \phi := \top 
        | \phi \land \phi 
        | t = _A u 
        |R (t , u) 
\]
\begin{prooftree}
    \AxiomC{}
    \UnaryInfC{$\Xi ; \Gamma ; \Theta \vdash \top \textrm{ VProp} $}
\end{prooftree}

\begin{prooftree}
    \AxiomC{$\Xi ; \Gamma ; \Theta \vdash \phi \textrm{ VProp} $}
    \AxiomC{$\Xi ; \Gamma ; \Theta \vdash \psi \textrm{ VProp} $}
    \BinaryInfC{$\Xi ; \Gamma ; \Theta \vdash \phi \land \psi \textrm{ VProp} $}
\end{prooftree}

\begin{prooftree}
    \AxiomC{$\Xi ; \Gamma \vdash_v t : A $}
    \AxiomC{$\Xi ; \Gamma \vdash_v u : A $}
    \BinaryInfC{$\Xi ; \Gamma ; \Theta \vdash t =_A u \textrm{ VProp} $}
\end{prooftree}

\begin{prooftree}
    \AxiomC{$\Xi ; \Gamma \vdash_v t : A $}
    \AxiomC{$\Xi ; \Gamma \vdash_v u : B $}
    \AxiomC{$R : Rel_{\mathcal{V}}[A,B] \in \Theta$}
    \TrinaryInfC{$\Xi ; \Gamma ; \Theta \vdash R(t,u) \textrm{ VProp}$}
\end{prooftree}

\subsubsection{Computation Propositions}

\[
    \underline{\psi} := 
    \underline{\top} 
    | \underline{\psi} \land \underline{\psi}
    | t =_{\underline{B}} u 
    |\phi \implies \underline{\psi}
    |\underline{R}(t , u)
    |\forall (x : A) .  \underline{\psi}
    |\forall X.  \underline{\psi}
    |\forall \underline{X}.  \underline{\psi}
    |\forall (R : Rel_{\mathcal{V}}[A , B]). \underline{\psi}
    |\forall (R : Rel_{\mathcal{C}}[\underline{A} , \underline{B}]). \underline{\psi}
\]

\begin{prooftree}
    \AxiomC{}
    \UnaryInfC{$\Xi ; \Gamma ; \Delta ; \Theta ; \Omega \vdash \underline{\top} \textrm{ CProp}$}
\end{prooftree}

\begin{prooftree}
    \AxiomC{$\Xi ; \Gamma ; \Delta ; \Theta ; \Omega \vdash \underline{\phi} \textrm{ CProp}$}
    \AxiomC{$\Xi ; \Gamma ; \Delta ; \Theta ; \Omega \vdash \underline{\psi} \textrm{ CProp}$}
    \BinaryInfC{$\Xi ; \Gamma ; \Delta ; \Theta ; \Omega \vdash \underline{\phi} \land \underline{\psi} \textrm{ CProp}$}
\end{prooftree}

\begin{prooftree}
    \AxiomC{$\Xi ; \Gamma \vdash_c t : \underline{B}$}
    \AxiomC{$\Xi ; \Gamma \vdash_c u : \underline{B}$}
    \BinaryInfC{$\Xi ; \Gamma ; \Delta ; \Theta ; \Omega \vdash t =_{\underline{B}} u \textrm{ CProp}$}
\end{prooftree}

\begin{prooftree}
    \AxiomC{$\Xi ; \Gamma; \Theta \vdash \phi $ VProp}
    \AxiomC{$\Xi ; \Gamma ; \Delta ; \Theta ; \Omega \vdash \underline{\psi}$ CProp}
    \BinaryInfC{$\Xi ; \Gamma ; \Delta ; \Theta ; \Omega \vdash \phi \implies \underline{\psi}$ CProp}
\end{prooftree}

\begin{prooftree}
    \AxiomC{$\Xi ; \Gamma \vdash_c t : \underline{A} $}
    \AxiomC{$\Xi ; \Gamma \vdash_c u : \underline{B} $}
    \AxiomC{$\underline{R} : Rel_{\mathcal{C}}[\underline{A},\underline{B}] \in \Omega$}
    \TrinaryInfC{$\Xi ; \Gamma ; \Delta ; \Theta ; \Omega \vdash \underline{R}(t,u) $ CProp}
\end{prooftree}

\begin{prooftree}
    \AxiomC{$\Xi ; \Gamma, x : A ; \Delta ; \Theta ; \Omega \vdash \underline{\psi}$ CProp}
    \UnaryInfC{$\Xi ; \Gamma ; \Delta ; \Theta ; \Omega \vdash \forall (x : A) . \underline{\psi}$ CProp}
\end{prooftree}

\begin{prooftree}
    \AxiomC{$\Xi ,X ; \Gamma ; \Delta ; \Theta ; \Omega \vdash \underline{\psi}$ CProp}
    \UnaryInfC{$\Xi ; \Gamma ; \Delta ; \Theta ; \Omega \vdash \forall X . \underline{\psi}$ CProp}
\end{prooftree}

\begin{prooftree}
    \AxiomC{$\Xi ,\underline{X} ; \Gamma ; \Delta ; \Theta ; \Omega \vdash \underline{\psi}$ CProp}
    \UnaryInfC{$\Xi ; \Gamma ; \Delta ; \Theta ; \Omega \vdash \forall \underline{X} . \underline{\psi}$ CProp}
\end{prooftree}

\begin{prooftree}
    \AxiomC{$\Xi ; \Gamma ; \Delta ; \Theta , R; \Omega \vdash \underline{\psi}$ CProp}
    \UnaryInfC{$\Xi ; \Gamma ; \Delta ; \Theta ; \Omega \vdash \forall R. \underline{\psi}$ CProp}
\end{prooftree}

\begin{prooftree}
    \AxiomC{$\Xi ; \Gamma ; \Delta ; \Theta ; \Omega, \underline{R} \vdash \underline{\psi}$ CProp}
    \UnaryInfC{$\Xi ; \Gamma ; \Delta ; \Theta ; \Omega \vdash \forall \underline{R}. \underline{\psi}$ CProp}
\end{prooftree}

\subsubsection{Value Relations}
\begin{prooftree}
    \AxiomC{$\Xi ; \Gamma , x : A \vdash_v t : C$}
    \AxiomC{$\Xi ; \Gamma , y : B \vdash_v u : C$}
    \RightLabel{}
    \BinaryInfC{$\Xi; \Gamma ; \Theta \vdash (x : A , y : B). t =_C u : Rel_{\mathcal{V}}[A,B]$}
\end{prooftree}

\begin{prooftree}
    \AxiomC{$\Xi ; \Gamma , x : A \vdash_v t : C$}
    \AxiomC{$\Xi ; \Gamma , y : B \vdash_v u : D$}
    \RightLabel{}
    \BinaryInfC{$\Xi ; \Gamma ; \Theta, R : Rel_{\mathcal{V}}[C,D] \vdash (x : A , y : B). R(t,u) : Rel_{\mathcal{V}}[A,B]$}
\end{prooftree}

\begin{comment}
    No, now we need to be careful with the difference between value and computation proposition
\begin{prooftree}
    \AxiomC{$\Xi; \Gamma , z : C ;\Theta \vdash (x : A , y : B). \phi : Rel_{\mathcal{V}}[A,B] $}
    \UnaryInfC{$\Xi ; \Gamma ; \Theta \vdash (x : A , y : B). \forall (z : C) . \phi : Rel_{\mathcal{V}}[A,B]$}
\end{prooftree}

\begin{prooftree}
    \AxiomC{$\Xi , X ; \Gamma ;\Theta \vdash (x : A , y : B). \phi : Rel_{\mathcal{V}}[A,B]$}
    \UnaryInfC{$\Xi ; \Gamma ; \Theta \vdash (x : A , y : B). \forall X . \phi : Rel_{\mathcal{V}}[A,B]$}
\end{prooftree}

\begin{prooftree}
    \AxiomC{$\Xi , \underline{X} ; \Gamma ;\Theta \vdash (x : A , y : B). \phi : Rel_{\mathcal{V}}[A,B]$}
    \UnaryInfC{$\Xi ; \Gamma ; \Theta \vdash (x : A , y : B). \forall \underline{X} . \phi : Rel_{\mathcal{V}}[A,B]$}
\end{prooftree}
\end{comment}

\subsubsection{Computation Relations}
\red{Something seems off including the stoup, $\Delta$, in the computation relation judgment..}
\begin{prooftree}
    \AxiomC{$\Xi ; \Gamma | x : \underline{A} \vdash_c t : \underline{C}$}
    \AxiomC{$\Xi; \Gamma | y : \underline{B} \vdash_c u : \underline{C}$}
    \RightLabel{}
    \BinaryInfC{$\Xi; \Gamma ; \Delta ; \Theta ; \Omega \vdash (x : \underline{A}, y : \underline{B}). t =_{\underline{C}} u : Rel_{\mathcal{C}}[\underline{A},\underline{B}]$}
\end{prooftree}
\begin{prooftree}
    \AxiomC{$\Xi ; \Gamma | x : \underline{A} \vdash_c t : \underline{C}$}
    \AxiomC{$\Xi ; \Gamma | y : \underline{B} \vdash_c u : \underline{D}$}
    \RightLabel{}
    \BinaryInfC{$\Xi ; \Gamma ; \Delta ;  \Theta ; \Omega,\underline{R} : Rel_{\mathcal{C}}[\underline{C},\underline{D}]\vdash (x : \underline{A}, y : \underline{B}). \underline{R}(t , u) : Rel_{\mathcal{C}}[\underline{A},\underline{B}] $}
\end{prooftree}

\subsection{Derivation Rules}

\subsubsection{Values}

\begin{prooftree}
    \AxiomC{}
    \UnaryInfC{$\Xi ; \Gamma  ; \Theta | \Phi  \vdash \top $}
\end{prooftree}

\begin{prooftree}
    \AxiomC{$\Xi ; \Gamma  ; \Theta | \Phi  \vdash \phi $}
    \AxiomC{$\Xi ; \Gamma  ; \Theta  | \Phi \vdash \psi $}
    \RightLabel{I-$\land$}
    \BinaryInfC{$\Xi ; \Gamma  ; \Theta  | \Phi \vdash \phi \land \psi$}
\end{prooftree}

\begin{prooftree}
    \AxiomC{$\Xi ; \Gamma  ; \Theta | \Phi \vdash \phi \land \psi $}
    \RightLabel{E1-$\land$}
    \UnaryInfC{$\Xi ; \Gamma  ; \Theta | \Phi  \vdash \phi$}
\end{prooftree}

\begin{prooftree}
    \AxiomC{$\Xi ; \Gamma  ; \Theta | \Phi \vdash \phi \land \psi $}
    \RightLabel{E2-$\land$}
    \UnaryInfC{$\Xi ; \Gamma  ; \Theta | \Phi  \vdash \psi$}
\end{prooftree}

\begin{prooftree}
    \AxiomC{$\Xi ; \Gamma \vdash_v t : A$}
    \RightLabel{}
    \UnaryInfC{$\Xi ; \Gamma  ; \Theta | \Phi  \vdash t =_A t$}
\end{prooftree}

\begin{prooftree}
    \AxiomC{$\Xi ; \Gamma  ; \Theta | \Phi  \vdash t =_A u$}
    \AxiomC{$\Xi ; \Gamma  ; \Theta  | \Phi \vdash \phi[t/x]$}
    \RightLabel{}
    \BinaryInfC{$\Xi ; \Gamma  ; \Theta | \Phi  \vdash \phi[u/x]$}
\end{prooftree}
Rel
\begin{prooftree}
    \AxiomC{$\Xi ; \Gamma  ; \Theta | \Phi  \vdash$}
    \AxiomC{$\Xi ; \Gamma  ; \Theta  | \Phi \vdash$}
    \RightLabel{}
    \BinaryInfC{$\Xi ; \Gamma  ; \Theta | \Phi  \vdash$}
\end{prooftree}


\subsubsection{Computation}
\begin{prooftree}
    \AxiomC{}
    \UnaryInfC{$\Xi ; \Gamma ; \Delta ; \Theta ; \Omega | \Phi ; \Psi \vdash \underline{\top} $}
\end{prooftree}

\begin{prooftree}
    \AxiomC{$\Xi ; \Gamma ; \Delta ; \Theta ; \Omega | \Phi ; \Psi \vdash  \underline{\phi} $}
    \AxiomC{$\Xi ; \Gamma ; \Delta ; \Theta ; \Omega | \Phi ; \Psi \vdash  \underline{\psi} $}
    \RightLabel{I-$\land$}
    \BinaryInfC{$\Xi ; \Gamma ; \Delta ; \Theta ; \Omega | \Phi ; \Psi \vdash \underline{\phi} \land \underline{\psi}$}
\end{prooftree}

\begin{prooftree}
    \AxiomC{$\Xi ; \Gamma ; \Delta ; \Theta ; \Omega | \Phi ; \Psi \vdash  \underline{\phi} \land \underline{\psi} $}
    \RightLabel{E1-$\land$}
    \UnaryInfC{$\Xi ; \Gamma ; \Delta ; \Theta ; \Omega | \Phi ; \Psi \vdash \underline{\phi}$}
\end{prooftree}

\begin{prooftree}
    \AxiomC{$\Xi ; \Gamma ; \Delta ; \Theta ; \Omega | \Phi ; \Psi \vdash  \underline{\phi} \land \underline{\psi} $}
    \RightLabel{E2-$\land$}
    \UnaryInfC{$\Xi ; \Gamma ; \Delta ; \Theta ; \Omega | \Phi ; \Psi \vdash  \underline{\psi}$}
\end{prooftree}

\begin{prooftree}
    \AxiomC{$\Xi ; \Gamma \vdash_c t : \underline{B}$}
    \RightLabel{}
    \UnaryInfC{$\Xi ; \Gamma ; \Delta ; \Theta ; \Omega | \Phi ; \Psi  \vdash t =_{\underline{B}} t$}
\end{prooftree}

\begin{prooftree}
    \AxiomC{$\Xi ; \Gamma ; \Delta ; \Theta ; \Omega | \Phi ; \Psi \vdash t =_{\underline{B}}  u$}
    \AxiomC{$\Xi ; \Gamma ; \Delta ; \Theta ; \Omega | \Phi ; \Psi \vdash \underline{\phi}[t/x]$}
    \RightLabel{}
    \BinaryInfC{$\Xi ; \Gamma ; \Delta ; \Theta ; \Omega | \Phi ; \Psi \vdash \underline{\phi}[u/x]$}
\end{prooftree}

\begin{prooftree}
    \AxiomC{$\Xi ; \Gamma ; \Delta ; \Theta ; \Omega | \Phi , \phi ; \Psi \vdash \underline{\psi}$}
    \RightLabel{I-$\implies$}
    \UnaryInfC{$\Xi ; \Gamma ; \Delta ; \Theta ; \Omega | \Phi ; \Psi \vdash \phi \implies \underline{\psi}$}
\end{prooftree}

\begin{prooftree}
    \AxiomC{$\Xi ; \Gamma ; \Delta ; \Theta ; \Omega | \Phi ; \Psi \vdash  \phi \implies \underline{\psi}$}
    \AxiomC{$\Xi ; \Gamma  ; \Theta | \Phi  \vdash \phi$}
    \RightLabel{E-$\implies$}
    \BinaryInfC{$\Xi ; \Gamma ; \Delta ; \Theta ; \Omega | \Phi ; \Psi \vdash \underline{\psi}$}
\end{prooftree}

Rel?


\begin{prooftree}
    \AxiomC{$\Xi ; \Gamma, x : A ; \Delta ; \Theta ; \Omega | \Phi ; \Psi \vdash \underline{\phi} $}
    \RightLabel{I-$\forall$ term , FV constraint?}
    \UnaryInfC{$\Xi ; \Gamma ; \Delta ; \Theta ; \Omega | \Phi ; \Psi \vdash \forall (x : A). \underline{\phi}$}
\end{prooftree}

\begin{prooftree}
    \AxiomC{$\Xi ; \Gamma ; \Delta ; \Theta ; \Omega | \Phi ; \Psi \vdash \forall (x : A). \underline{\phi}$}
    \AxiomC{$\Xi ; \Gamma \vdash_v t : A$}
    \RightLabel{E-$\forall$ term}
    \BinaryInfC{$\Xi ; \Gamma ; \Delta ; \Theta ; \Omega | \Phi ; \Psi \vdash \underline{\phi}[t/x]$}
\end{prooftree}

\begin{prooftree}
    \AxiomC{$\Xi,X ; \Gamma ; \Delta ; \Theta ; \Omega | \Phi ; \Psi \vdash \underline{\phi} $}
    \RightLabel{I-$\forall$ vtype , FV constraint?}
    \UnaryInfC{$\Xi ; \Gamma ; \Delta ; \Theta ; \Omega | \Phi ; \Psi \vdash \forall X. \underline{\phi}$}
\end{prooftree}

\begin{prooftree}
    \AxiomC{$\Xi ; \Gamma ; \Delta ; \Theta ; \Omega | \Phi ; \Psi \vdash \forall X. \underline{\phi}$}
    \AxiomC{$\Xi \vdash A$}
    \RightLabel{E-$\forall$ vtype}
    \BinaryInfC{$\Xi ; \Gamma ; \Delta ; \Theta ; \Omega | \Phi ; \Psi \vdash \underline{\phi}[A/X]$}
\end{prooftree}

\begin{prooftree}
    \AxiomC{$\Xi,\underline{X} ; \Gamma ; \Delta ; \Theta ; \Omega | \Phi ; \Psi \vdash \underline{\phi} $}
    \RightLabel{I-$\forall$ ctype , FV constraint?}
    \UnaryInfC{$\Xi ; \Gamma ; \Delta ; \Theta ; \Omega | \Phi ; \Psi \vdash \forall \underline{X} . \underline{\phi}$}
\end{prooftree}

\begin{prooftree}
    \AxiomC{$\Xi ; \Gamma ; \Delta ; \Theta ; \Omega | \Phi ; \Psi \vdash \forall \underline{X} . \underline{\phi}$}
    \AxiomC{$\Xi \vdash \underline{A} $}
    \RightLabel{E-$\forall$ ctype}
    \BinaryInfC{$\Xi ; \Gamma ; \Delta ; \Theta ; \Omega | \Phi ; \Psi \vdash \underline{\phi}[\underline{A} /\underline{X} ]$}
\end{prooftree}

\begin{prooftree}
    \AxiomC{$\Xi; \Gamma ; \Delta ; \Theta , (R : Rel_{\mathcal{V}}[A,B]); \Omega | \Phi ; \Psi \vdash \underline{\phi} $}
    \RightLabel{I-$\forall$ vrel}
    \UnaryInfC{$\Xi ; \Gamma ; \Delta ; \Theta ; \Omega | \Phi ; \Psi \vdash \forall (R : Rel_{\mathcal{V}}[A,B]) . \underline{\phi}$}
\end{prooftree}

\begin{prooftree}
    \AxiomC{$\Xi ; \Gamma ; \Delta ; \Theta ; \Omega | \Phi ; \Psi \vdash \forall (R : Rel_{\mathcal{V}}[A,B]) . \underline{\phi}$}
    \AxiomC{$\Xi; \Gamma ; \Theta , \vdash (x : A , y : B). \psi : Rel_{\mathcal{V}}[A,B]$}
    \RightLabel{E-$\forall$ vrel}
    \BinaryInfC{$\Xi ; \Gamma ; \Delta ; \Theta ; \Omega | \Phi ; \Psi \vdash \forall (R : Rel_{\mathcal{V}}[A,B]) . \underline{\phi}[\psi[t/x,u/y]/R(t,u)]$}
\end{prooftree}


\begin{prooftree}
    \AxiomC{$\Xi; \Gamma ; \Delta ; \Theta , (\underline{R} : Rel_{\mathcal{C}}[\underline{A},\underline{B}]); \Omega | \Phi ; \Psi \vdash \underline{\phi} $}
    \RightLabel{I-$\forall$ crel}
    \UnaryInfC{$\Xi ; \Gamma ; \Delta ; \Theta ; \Omega | \Phi ; \Psi \vdash \forall (\underline{R} : Rel_{\mathcal{C}}[\underline{A},\underline{B}]) . \underline{\phi}$}
\end{prooftree}

\begin{prooftree}
    \AxiomC{$\Xi ; \Gamma ; \Delta ; \Theta ; \Omega | \Phi ; \Psi \vdash \forall (R : Rel_{\mathcal{C}}[\underline{A},\underline{B}]) . \underline{\phi}$}
    \AxiomC{$\Xi; \Gamma ; \Delta ; \Theta ; \Omega , \vdash (x : \underline{A} , y : \underline{B}). \underline{\psi} : Rel_{\mathcal{C}}[\underline{A},\underline{B}]$}
    \RightLabel{E-$\forall$ crel}
    \BinaryInfC{$\Xi ; \Gamma ; \Delta ; \Theta ; \Omega | \Phi ; \Psi \vdash \forall (R : Rel_{\mathcal{C}}[\underline{A},\underline{B}]) . \underline{\phi}[\underline{\psi}[t/x,u/y]/R(t,u)]$}
\end{prooftree}

\subsubsection{Congruences}
\begin{prooftree}
    \AxiomC{$\Xi ; \Gamma \vdash_c t : \underline{B}$}
    \AxiomC{$\Xi ; \Gamma \vdash_c u : \underline{B}$}
    \AxiomC{$\Xi ; \Gamma,x : A ; \Delta ; \Theta ; \Omega | \Phi ; \Psi \vdash t = u$}
    \RightLabel{cong-$\lambda$}
    \TrinaryInfC{$\Xi ; \Gamma ; \Delta ; \Theta ; \Omega | \Phi ; \Psi \vdash \lambda(x : A).t = \lambda(x : A).u$}
\end{prooftree}
\subsection{Axioms}
Beta/Eta/(parametricity schema?)


\subsection{Logical Interpretation of Types}
Let $\bm{X}$ and $\bm{\underline{X}}$ be vectors of value type and computation type variables of length $n$.
Let $\bm{\rho}$ be a vector of value relations $\Xi ; \Gamma ; \Theta \vdash \rho_i : Rel_{\mathcal{V}}[C_i , C'_i]$ for all $i \in {1..n}$.
Let $\bm{\underline{\rho}}$ be a vector of computation relations $\Xi ; \Gamma ; \Delta ; \Theta ; \Omega \vdash \underline{\rho_i} : Rel_{\mathcal{C}}[\underline{C_i} , \underline{C'_i}]$ for all $i \in {1..n}$.
\\Let $A$ be a \textbf{value type} with $FTV(A) \in \{\bm{X},\bm{\underline{X}}\}$. Define:
\[
  A[\bm{\rho}/\bm{X},\bm{\underline{\rho}}/\bm{\underline{X}}] : Rel_{\mathcal{V}}[A[\bm{C}/\bm{X},\bm{\underline{C}}/\bm{\underline{X}}], A[\bm{C'}/\bm{X},\bm{\underline{C'}/\bm{\underline{X}}}]] 
\] 
by induction on $A$. 


\begin{align*}
    X_i[\bm{\rho},\bm{\underline{\rho}}] &= \rho_i\\
    \textrm{Unit} [\bm{\rho},\bm{\underline{\rho}}]&= (x : Unit , y : Unit). x =_{Unit} y\\
    \textrm{Case} A [\bm{\rho},\bm{\underline{\rho}}]&= (x : \textrm{Case } (A[\bm{C}/\bm{X},\bm{\underline{C}}/\bm{\underline{X}}]),
    y : \textrm{Case } (A[\bm{C'}/\bm{X},\bm{\underline{C'}}/\bm{\underline{X}}])). 
    \\
    & \blue{\textrm{think exists?}}\\
    \textrm{OSum} [\bm{\rho},\bm{\underline{\rho}}]&= \blue{\textrm{think exists?}}\\
    A \times A' [\bm{\rho},\bm{\underline{\rho}}]&= ((x,y) : A\times A'[\bm{C}/\bm{X},\bm{\underline{C}}/\bm{\underline{X}}], (x',y') :  A \times A'[\bm{C'}/\bm{X},\bm{\underline{C'}/\bm{\underline{X}}}]).\\
    & A[\bm{\rho},\bm{\underline{\rho}}](x,x')\land A'[\bm{\rho},\bm{\underline{\rho}}](y,y') \\
    A * A' [\bm{\rho},\bm{\underline{\rho}}]&= \textrm{similar to product?}\\
    \exists X . A [\bm{\rho},\bm{\underline{\rho}}]&= \textrm{standard} \\
    U \underline{B} [\bm{\rho},\bm{\underline{\rho}}]&= \textrm{related thunks?}\\
\end{align*}

Let $\underline{B}$ be a \textbf{computation type} with $FTV(\underline{B}) \in \{\bm{X},\bm{\underline{X}}\}$. Define:
\[
  \underline{B}[\bm{\rho}/\bm{X},\bm{\underline{\rho}}/\bm{\underline{X}}] : Rel_{\mathcal{C}}[\underline{B}[\bm{C}/\bm{X},\bm{\underline{C}}/\bm{\underline{X}}], \underline{B}[\bm{C'}/\bm{X},\bm{\underline{C'}/\bm{\underline{X}}}]] 
\] 
by induction on $\underline{B}$. 

\begin{align*}
    A \rightarrow \underline{B}[\bm{\rho},\bm{\underline{\rho}}] &= \textrm{Add weakest precondition to the proposition syntax to interpret this?}\\
    A \sep \underline{B}[\bm{\rho},\bm{\underline{\rho}}] &= \\
    \forall X . \underline{B} [\bm{\rho},\bm{\underline{\rho}}]&= \\
    F A[\bm{\rho},\bm{\underline{\rho}}] &= \textrm{Could be encoded if we add } \forall \underline{X} . \underline{B}?\\
\end{align*}

\section{Goal}
Try to prove 
\begin{align*}
    \vdash_d & \forall A \;\underline{B}. \\
    & \forall (y : \textrm{OSum}). \\
    & \forall (f : U(\forall X . \textrm{Case }X \sep \textrm{OSum} \rightarrow F X)). \\
    & \forall (k \; k' : U(A \rightarrow \underline{B})). \\
    & \textrm{newcase}_A \sigma ; x \leftarrow (!f)[A]\sigma y ; (!k) x  \\
    & = \\
    & \textrm{newcase}_A \sigma ; x \leftarrow (!f)[A]\sigma y ; (!k') x  
\end{align*}   

\section{TODO/Questions}
\begin{itemize}
    \item Do we add the type $\forall \underline{X}. \underline{B}$ to the object lang and then encode $F A$ ?
    \item Finalize the judgment forms, what contexts do they actually need to be displayed over? Do we need to split the relation and proposition contexts into distinct value/computation contexts?
    \item Check that the classification of logical connectives makes sense (value prop vs comp prop)
    \item Denotation of value/computation propositions
    \item Understand the operation $\oslash$ and its laws
    \item Denotation of value/computation derivations 
    \item Define the operation $\oslash^*$ and find its laws
    \item Finish the known relational interpretation of types
    \item Attempt the relational interpretation of our new types
    \item Write up the beta and eta deduction rules 
    \item Check the correctness of the relational interpretation of types. (By proving Reynold's Identity Extension Lemma?)
    \item PE Logic denotes the collection of computation relations, $Rel_{\mathcal{C}}[\underline{A},\underline{B}]$, by $Sub_{\mathcal{C}}(\underline{A} \times \underline{B})$. However,
          they never define $\underline{A} \times \underline{B}$ or state that it is a derivable type.
\end{itemize}

\end{document}