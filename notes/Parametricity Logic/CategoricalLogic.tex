\documentclass{article}
\usepackage{graphicx} % Required for inserting images
\usepackage{simplebnf}
\usepackage{bussproofs}
\usepackage{amsmath}
\usepackage{amssymb}
\usepackage[dvipsnames]{xcolor}
\usepackage{stmaryrd}
\usepackage{comment}
\usepackage{bm}

%\usepackage{xcolor}

\newcommand{\den}[1]{\llbracket #1 \rrbracket}
\newcommand{\blue}[1]{\textcolor{blue}{#1}}
\newcommand{\red}[1]{\textcolor{red}{#1}}
\newcommand{\sep}{\mathrel{-\mkern-6mu*}}
\newcommand{\thunk}[1]{\textrm{thunk }#1}
\newcommand{\injj}[2]{\textrm{inj}_{#1}#2}
\newcommand{\err}{\mho}
\newcommand{\print}[1]{\textrm{print }#1}
\newcommand{\force}[1]{\textrm{force }#1}
\newcommand{\ret}[1]{\textrm{ret }#1}
\newcommand{\bind}[3]{#1 \leftarrow #2 ; #3}
\newcommand{\newcase}[3]{\textrm{newcase}_{#1} \; #2 ; #3}
\newcommand{\match}[5]{\textrm{match }#1 \textrm{ with }#2 \;\{#3 . #4 | #5\}}
\newcommand{\unpack}[4]{\textrm{unpack }(#1,#2) = #3 ; #4}
\newcommand{\lett}[4]{\textrm{let }(#1,#2) = #3 ; #4}
\newcommand{\lets}[4]{\textrm{let }(#1*#2) = #3 ; #4}
\newcommand{\ite}[3]{\textrm{if }#1 \textrm{ then }#2 \textrm{ else }#3}
\newcommand{\at}{\textrm{@}}
\newcommand{\ttt}{\textrm{tt}}
\newcommand{\tru}{\textrm{true}}
\newcommand{\fsl}{\textrm{false}}
\newcommand{\pworld}{\widehat{\mathbf{World}}}
\newcommand{\world}{{\mathbf{World}}}
\newcommand{\calculus}{\operatorname{-calculus}}


\title{Categorical Logic}
\begin{document}
\maketitle

\section{Simply Typed Logics}
Here we present the setup for a logic of simply typed lambda calculus term equality.

\subsection{Syntax}
\begin{bnf}
    $\alpha$ : Types ::= Unit | $\alpha \times \alpha$ | $\alpha \rightarrow \alpha$ ;;
    $M$ : Terms ::= $tt$ | $(M , N)$ | $\pi_1 M$ | $\pi_2 M$ | $\lambda (x \colon \alpha). M$ | $MN$
\end{bnf}

With the usual typing rules, structural rules, and beta/eta equalities.
\subsection{Classifying Category}
This category represents the syntax of the simply typed lambda calculus. 
(\red{not just $\beta/\eta$ equivalence? conversion can have $\alpha$ renaming or congruence rules?})
\begin{align*}
    \textrm{Objects} &:= \textrm{Types, $\alpha$, constructed by the grammar}\\
    \textrm{Morphisms} &:= \alpha \rightarrow \beta \textrm{ are equivalence classes $[M]$ of terms } x : \alpha \vdash M : \beta \\
    &\textrm{ with one free variable(Equivalence relative to } \beta/\eta \textrm{ laws})\\
    \textrm{Identity } \alpha &:= \textrm{equivalence class of term } x : \alpha \vdash x : \alpha\\
    \textrm{Composition }f g  &:= \textrm{ is substitution. Say } f \textrm{ contains } x : \alpha \vdash M : \beta\\
    &\textrm{ and g contains } y : \beta \vdash N : \gamma \textrm{ then } x : \alpha \vdash N[M/y] : \gamma \textrm { in g $\circ$ f.}
    %\textrm{ where } (f : \alpha \rightarrow \beta)(g : \beta \rightarrow \gamma) \\
\end{align*}

This category, denoted here by $\mathcal{CL}(\lambda_{1,\times,\rightarrow})$, is cartesian closed. 
\begin{comment}
    My mistake here is using the definition of classifying category for a theory that does not assume built in products


\begin{align*}
    \textrm{Objects} &:= \textrm{Contexts } \Gamma = (x_1 \colon \alpha_1 , x_2 \colon \alpha_2... )\\ 
    \textrm{Morphisms} &:= \Gamma \rightarrow \Delta \textrm{with } |\Gamma| = n , |\delta| = m \\
    & \textrm{ with } \Gamma = (x_1 : \alpha_1 , x_2 : \alpha_2, .., x_n : \alpha_n) \; \delta = (y_1 : \beta_1 , y_2 : \beta_2, .., y_m : \beta_m)\\
    & \textrm{ consists of an $m$ tuple of equivalence clases of terms} (\beta/\eta \textrm { equivalence}) \\
    & ([M_1], [M_2], .. [M_m]) \textrm{ where } \Gamma \vdash M_i : \beta_i \\
    & \blue{think "multimophisms"}
\end{align*}

\blue{Or, since our calculus has products, we can just take \textbf{Types} as objects. See Crole page 177 and Jacobs page 139}
\end{comment}
\subsection{Term Model}
A \textbf{model} of the term language in a cartesian closed category $\mathcal{C}$ is a functor from the classifying category into $\mathcal{C}$\\
\[
M : \mathcal{CL}(\lambda_{1,\times,\rightarrow}) \rightarrow \mathcal{C}
\]
\subsection{Base Logic Syntax}
We now turn out attention to a logic where the only proposition is equality of simply typed lambda calculus terms.
We have two judgements:
\[
  \Gamma \vdash \phi \textrm{ Prop}  
\]
which states $\phi$ is a proposition. And 
\[
    \Gamma | \Phi \vdash \psi 
\]
which is a proof derivation judgement. There are a few structural and context rules for proof derivations 
\begin{itemize}
    \item Axiom
    \item Identity
    \item Cut 
    \item weakening (for prop)
    \item contraction (for prop)
    \item exchange (for prop)
    \item weakening (for term context)
    \item contraction (for term context)
    \item exchange (for term context)
    \item substitution
\end{itemize}
\subsection{Equational Logic Syntax}
Here are the rules for the equality proposition.
For the first judgement, $\phi$ is generated from a boring bnf grammar:

\begin{bnf}
    $\phi$ ::= $ M =_{\alpha} M'$ ;;
\end{bnf}
\\
There is one formation rule for propositions
\begin{prooftree}
    \AxiomC{$\Gamma \vdash M : \alpha $}
    \AxiomC{$\Gamma \vdash M' : \alpha $}
    \BinaryInfC{$\Gamma \vdash M =_{\alpha} M' \textrm{ Prop}$}
\end{prooftree}

You have the expected rules for $=$.
\begin{prooftree}
    \AxiomC{$\Gamma \vdash M : \alpha$}
    \RightLabel{refl}
    \UnaryInfC{$\Gamma | \Phi \vdash M =_{\alpha} M$}
\end{prooftree}

\begin{prooftree}
    \AxiomC{$\Gamma | \Phi \vdash M =_{\alpha} M' $}
    \RightLabel{sym}
    \UnaryInfC{$\Gamma | \Phi \vdash M' =_{\alpha} M$}
\end{prooftree}

\begin{prooftree}
    \AxiomC{$\Gamma | \Phi \vdash M =_{\alpha} N $}
    \AxiomC{$\Gamma | \Phi \vdash N =_{\alpha} P $}
    \RightLabel{trans}
    \BinaryInfC{$\Gamma | \Phi \vdash M =_{\alpha} P$}
\end{prooftree}

\begin{prooftree}
    \AxiomC{$\Gamma | \Phi \vdash M =_{\alpha} M' $}
    \AxiomC{$\Gamma , x : \alpha \vdash N : \beta$}
    \RightLabel{replace}
    \BinaryInfC{$\Gamma | \Phi \vdash N[M/x] =_{\beta} N[M'/x]$}
\end{prooftree}

The four previous rules are equivalent to the following two. For proof, see page 180 of Jacobs.

\begin{prooftree}
    \AxiomC{$\Gamma \vdash M : \alpha $}
    \AxiomC{$\Delta , x : \alpha , \Delta' \;|\; \overrightarrow{N} =_{\overrightarrow{\beta}} \overrightarrow{N'} \vdash L =_{\gamma} L'$}
    \RightLabel{substitution}
    \BinaryInfC{$\Delta , \Gamma , \Delta' \;| \;\overrightarrow{N[M/x]} =_{\overrightarrow{\beta}} \overrightarrow{N'[M/x]} \vdash L[M/x] =_{\gamma} L'[M/x]$}
\end{prooftree}

\begin{prooftree}
    \AxiomC{$\Gamma, x : \alpha | \Phi \vdash N[x/y]=_{\beta}N'[x/y]$}
    \doubleLine
    \RightLabel{Lawvere Equality, $=$-mate}
    \UnaryInfC{$\Gamma , x : \alpha , y : \alpha | \Phi , x =_{\alpha} y \vdash N =_{\beta} N'$}
\end{prooftree}

Additionally, we also have that definitional equality / conversion is contained in propositional equality.

\begin{prooftree}
    \AxiomC{$\Gamma \vdash M : \alpha$}
    \AxiomC{$\Gamma \vdash M' : \alpha$}
    \AxiomC{$\Gamma \vdash M = M' : \alpha$}
    \TrinaryInfC{$\Gamma \vdash M =_{\alpha} M'$}
\end{prooftree}

\subsection{Preliminaries for the Logic Semantics}
\textbf{Def. Preorder}: A set $X$ with a binary relation $R \subseteq X \times X$ that is reflexive and transitive.
\\
\textbf{Def. Partially Ordered Set (Poset)}: A set $X$ with a binary relation $R : \subseteq X \times X$ that is reflexive, transitive, and anti-symmetric.
\\
Any preorder can be regarded as a thin category where $(x,y) \in R$ is regarded as the existence of a unique morphism from $x$ to $y$.
\\
Any preorder can be turned into a partial order via a \textit{posetal reflection}. Let $(A, \leq )$ be a preorder. We construct a poset $(A/\cong,\leq')$ where 
\[
  a \cong b \iff a \leq b \land b \leq a  
\]
and 
\[
  [a] \leq' [b] \iff a \leq b 
\]
The quotient \textit{enforces} the anti-symmetry condition.
\\
For a category $\mathcal{C}$ and an object, $X : ob \;\mathcal{C}$, we can take the category 
$Mono(\mathcal{C})$ to be a full subcategory $\mathcal{C}/X$ which consists only of monomorphisms.
This category can be regarded as a preorder.
\\ 
For a category $\mathcal{C}$ and an object, $X : ob \;\mathcal{C}$, we can take the \textit{posetal reflection} of $Mono(X)$ 
and regard it as a category $Sub(X)$. This category consists of the subobjects of $X$ in $\mathcal{C}$.
\\
Miscelaneous facts:
\begin{itemize}
    \item binary products + pullbacks $\implies$ equalizers
    \item every equalizer is a monomorphism
    \item pullbacks preserve monomorphisms
    \item Every category with pullbacks of monomorphisms has a contravariant functor $Sub:C^{op}\rightarrow Pos$ to the category of posets called the subobject poset functor, making it into a hyperdoctrine.
\end{itemize}

\subsection{Basics of Set Based Logic Semantics}
Assume we have a set based model, $F : \mathcal{CL}(\lambda_{1,\times,\rightarrow}) \rightarrow Set$, for our simply typed lambda calculus.
We need to interpret our two equational logic judgments:
\begin{align*}
    &\Gamma \vdash \phi \textrm{ Prop}\\
    &\Gamma | \Phi \vdash \phi
\end{align*}
For a context $\Gamma$, the proposition $\Gamma \vdash \phi$ will be interpreted as a subset $X \subseteq \den{\Gamma}_F$. 
Propositional context, $\Phi = \phi \land \psi \land ..$, will be interpreted as a conjunction of propositions 
where a conjunction of propositions $\Gamma \vdash \phi$ and $\Gamma \vdash \psi$ is:
\[
    \den{\phi \land \psi}_F = \den{\phi}_F \cap \den{\psi}_F
\]
We can drop $F$ from the denotation subscript when the model is obvious.
For any term context $\Gamma$, we have the poset $(\mathcal{P}(\den{\Gamma}_F), \leq)$
where $\mathcal{P}(\den{\Gamma}_F)$ is the powerset of $\den{\Gamma}_F$.
The ordering is given by subset inclusion, that is:
\[
  \den{\phi} \leq \den{\psi} \iff \den{\phi} \subseteq \den{\psi}  
\]
Checking this is a poset.. we clearly have that the relation is reflexive since $X \subseteq X$
Additionally, it is also transitive since subset inclusion is transitive.
Finally, we have that $\leq$ is anti\-symmetric since 
\[
    X \subseteq Y \land Y \subseteq X \implies  X = Y 
\]

Thus we will interpret $\den{\Gamma | \Phi \vdash \phi} = \den{\Phi} \leq  \den{\phi}$ 
which means that either $\den{\Phi} \subseteq \den{\phi}$ or not. 
Our derivations are \textit{proof irrelevant} since there is at most one term representing if a derivation is inhabited.

\subsection{Set Based Semantics for Propositional Equality}
Our equational logic only has one logical connective, $M =_{\alpha} M'$ propositional equality.
To interpret this proposition, we use equalizers. Equalizers in $Set$ are rather simple.
Given $\Gamma \vdash M : \alpha$ and $\Gamma \vdash M' : \alpha$, we have the equalizer:
\[
    Eq(\den{M},\den{M'}) = \{ x : \den{\Gamma} \;|\; \den{M}(x) = \den{M'}(x)\}
\]
Remember that terms are denoted as morphisms $\den{\Gamma \vdash M : \alpha} : Set[\den{\Gamma},\den{\alpha}]$ or functions in $Set$.
The equalizer of terms $M$ and $M'$ is just the subset of $\den{\Gamma}$ for which these functions return equal values in $\den{\alpha}$.
\subsection{Demonstration of Propositional Equality Logic}
In our logic, we should be able to prove the sequent
\[
  x : \alpha, y : \alpha, z : \alpha | x =_{\alpha} y , y =_{\alpha} z \vdash x =_{\alpha} z  
\]
via the transitivity rule. Lets check that this is sound w.r.t the set based denotation.
We have that $\den{\alpha}$ is some set and $\den{\Gamma} = \den{\alpha} \times \den{\alpha}\times \den{\alpha}$.
The terms $x,y,z$ just projections out of the context.
\begin{align*}
    &\den{x} : \den{\Gamma} \rightarrow \den{\alpha} = \pi_1 \\
    &\den{y} : \den{\Gamma} \rightarrow \den{\alpha} = \pi_2\\
    &\den{z} : \den{\Gamma} \rightarrow \den{\alpha} = \pi_3 \\
\end{align*}
We have that:
\begin{align*}
    &\den{x =_{\alpha} y} = Eq(x,y) = \{\gamma : \den{\Gamma} \;|\; \pi_1(\gamma) = \pi_2(\gamma)\}\\
    &\den{y =_{\alpha} z} = Eq(y,z) = \{\gamma : \den{\Gamma} \;|\; \pi_2(\gamma) = \pi_3(\gamma)\}\\
    &\den{x =_{\alpha} z} = Eq(x,z) = \{\gamma : \den{\Gamma} \;|\; \pi_1(\gamma) = \pi_3(\gamma)\}\\
\end{align*}
and 
\[
    (x =_{\alpha} y \land y =_{\alpha} z) = \den{x =_{\alpha} y} \cap \den{y =_{\alpha} z} = \{ \gamma : \den{\Gamma} \; | \; \pi_1(\gamma) = \pi_2(\gamma) = \pi_3(\gamma)\}
\]
Our proposition considered true in the model if:
\[
    \den{x =_{\alpha} y} \cap \den{y =_{\alpha} z} \subseteq \den{x =_{\alpha} z} 
\]
The LHS consists of all tuples of the form 
\[
    (M , M, M) : \den{\alpha} \times \den{\alpha}\times \den{\alpha}
\]  
where all components of the tuple are equal. While the RHS consists of all tuples of the form
\[
    (M , N, M) : \den{\alpha} \times \den{\alpha}\times \den{\alpha}
\]
where only the first and third components are required to be equal. 
Thus, there is clearly a subset inclusion from the LHS to the RHS. 

\subsection{Changing Contexts}
We have sequent $x : \alpha \;|\; \cdot \vdash x =_{\alpha} x $ via reflexivity. 
We can apply weakening for terms to this sequent to obtain a proof of 
$x : \alpha, y : \beta \;|\; \cdot \vdash x =_{\alpha} x $. 
How is this operation justified in the set based semantics? 
Notice that the first sequent is in the poset \textit{over} $\den{\alpha}$ 
while the second sequent is in the poset \textit{over} $\den{\alpha} \times \den{\beta}$.

We need a way to \textit{change basis}. This is the role of substitution functors. 
Recall that we can regard preorders(refl,trans) as categories if we interpret $a \leq_R b$ as the existence of a unique morphism between objects $a,b$.
Furthermore, a poset(refl,trans,anti-sym) can be regarded as a category where $a \leq_R b$ and $b \leq_R b$ implies $a \cong b$. \
\footnote{Since there is at most one morphisms between objects, this isomorphism is forced}
We can then regard a poset \textit{over} $\den{\alpha}$ as a category $Sub(\den{\alpha})$.
For any morphism 
\[
 f : \den{\alpha} \rightarrow \den{\beta}   
\]

we have an induced functor
\[
 f^* : \textrm{Functor} \;Sub(\den{\beta}) \; Sub(\den{\alpha})
\]
given by 
\begin{align*}
    f^*_0 (X \subseteq \den{\beta}) &= \{a : \den{\alpha} \;|\; f(a) \in X\} \subseteq \den{\alpha}\\
    f^*_1 (X \leq_{\den{\beta}} Y) &= \{ a : \den{\alpha} \; | \; f(a) \in X\}\leq_{\den{\alpha}} \{ a : \den{\alpha} \; | f(a) \in Y\; \}\\
\end{align*}
We can check that 
\begin{align*}
    f^*_1(X \leq_{\den{\beta}} X) &= \{ a : \den{\alpha} \; | \; f(a) \in X\}\leq_{\den{\alpha}} \{ a : \den{\alpha} \; | f(a) \in X\; \}\\
    &= id_{f^*_0(X)}
\end{align*}
and 
\begin{align*}
    f^*_1(X \leq_{\den{\beta}}  Y ; Y \leq_{\den{\beta}}  Z) &=   f^*_1(X \leq_{\den{\beta}}  Z)\\
    & \textrm{b.c. at most one morphism and } X \subseteq Y \land Y \subseteq Z \implies X \subseteq Z\\
    &= \{ a : \den{\alpha} \; | \; f(a) \in X\}\leq_{\den{\alpha}} \{ a : \den{\alpha} \; | f(a) \in Z\; \}\\
    &= f^*_1(X \leq_{\den{\beta}}  Y) ; f^*_1(Y \leq_{\den{\beta}}  Z)\\
\end{align*}

Lets consider a special case 
\begin{align*}
    &\pi: \den{\alpha}\times\den{\beta}\rightarrow\den{\alpha}\\
    &\pi (a,b) = a \\
\end{align*}
This induces a substitution functor $\pi^* :$ Functor $Sub(\den{\alpha})$ $Sub(\den{\alpha} \times \den{\beta})$ 
which we can use to transport our proof over $\den{\alpha}$ to be over $\den{\alpha}\times \den{\beta}$.

\subsection{Eq as left adjoint to $\delta^*$}
\subsection{TODO: explain the total category $Sub(\mathcal{C})$ and how Functor $Sub(\mathcal{C})$  $\mathcal{C}$ is a fibration}
 
\subsection{Regular Logic Syntax}    
Equational logic only has one type of proposition, propositional equality. 
We will now focus on a logic with four connectives: 

\begin{bnf}
    $\phi$ ::= $M =_{\alpha} M'$ | $\top$ | $\phi \land \phi'$ | $\exists(x \colon \alpha). \phi$
\end{bnf}

This is a subset of first order logic called \textit{regular logic}. In addition to the base logic structural rules, we have the following:
\subsubsection{Proposition Formation Rules}

\begin{prooftree}
    \AxiomC{$\Gamma \vdash M : \alpha$}
    \AxiomC{$\Gamma \vdash M' : \alpha$}
    \BinaryInfC{$\Gamma \vdash M =_{\alpha} M'$ Prop}
\end{prooftree}

\begin{prooftree}
    \AxiomC{}
    \UnaryInfC{$\Gamma \vdash \top $ Prop}
\end{prooftree}

\begin{prooftree}
    \AxiomC{$\Gamma \vdash \phi$ Prop }
    \AxiomC{$\Gamma \vdash \psi$ Prop}
    \BinaryInfC{$\Gamma \vdash \phi \land \psi$ Prop}
\end{prooftree}

\begin{prooftree}
    \AxiomC{$\Gamma, x : \alpha \vdash \phi$ Prop}
    \UnaryInfC{$\Gamma \vdash \exists (x : \alpha). \phi$ Prop}
\end{prooftree}

\subsubsection{Derivation Rules}
\begin{prooftree}
    \AxiomC{}
    \RightLabel{$\top$-Intro}
    \UnaryInfC{$\Gamma | \Phi \vdash \top$}    
\end{prooftree}

\begin{prooftree}
    \AxiomC{$\Gamma | \Phi \vdash \phi$}
    \AxiomC{$\Gamma | \Phi \vdash \psi$}
    \RightLabel{$\land$-Intro}
    \BinaryInfC{$\Gamma | \Phi \vdash \phi \land \psi$}
\end{prooftree}

\begin{prooftree}
    \AxiomC{$\Gamma | \Phi \vdash \phi \land \psi$}
    \RightLabel{$\land\textrm{-Elim}_1$}
    \UnaryInfC{$\Gamma | \Phi \vdash \phi$}
\end{prooftree}

\begin{prooftree}
    \AxiomC{$\Gamma | \Phi \vdash \phi \land \psi$}
    \RightLabel{$\land\textrm{-Elim}_2$}
    \UnaryInfC{$\Gamma | \Phi \vdash \psi$}
\end{prooftree}

Equality and existential propositions have special, bidirectional, \textit{mate} rules.
These combined introduction/elimination rules are a reflection of adjunctions used in the semantics.

\footnote{See page 225 Jacobs for the 4 rules $=$-mate replaces and the 2 rules $\exists$-mate replaces}
\begin{prooftree}
    \AxiomC{$\Gamma, x : \alpha | \Phi \vdash \phi[x/y]$}
    \doubleLine
    \RightLabel{$=$-mate}
    \UnaryInfC{$\Gamma, x : \alpha, y : \alpha | \Phi, x =_{\alpha} y \vdash \phi$}
\end{prooftree}

\begin{prooftree}
    \AxiomC{$\Gamma | \exists (x : \alpha). \phi \vdash \psi$}
    \doubleLine
    \RightLabel{$\exists$-mate}
    \UnaryInfC{$\Gamma , x : \alpha | \phi \vdash \psi $}
\end{prooftree}

\subsection{Semantics of Regular Logic}
A sequent, $\Gamma |\Phi \vdash \phi$, is \textit{valid} if 
\[
    \den{\Gamma \vdash \Phi} \leq_{\den{\Gamma}} \den{\Gamma \vdash \phi}  
\]
and a proposition, $\Gamma \vdash \phi$ is \textit{valid} if the sequent $\Gamma | \top \vdash \phi$ is valid.
\[
  \den{\Gamma} \leq_{\Gamma} \den{\Gamma \vdash \phi}
\]

We need a semantic interpretation of our new logical connectives.
\begin{align*}
     \den{\Gamma \vdash M =_{\alpha}M'} &= Eq_{\den{\Gamma}}(\den{\Gamma \vdash M : \alpha},\den{\Gamma \vdash M' : \alpha})\\
     \den{\Gamma \vdash \top} &= \den{\Gamma}\\
     \den{\Gamma \vdash \phi \land \psi} &= \den{\Gamma \vdash \phi} \land \den{\Gamma \vdash \psi} \\
     \den{\Gamma \vdash \exists (x : \alpha). \phi} &= \exists_{\den{\Gamma},\den{\alpha}}(\den{\Gamma, x : \alpha \vdash \phi})\\
\end{align*}

The terms $Eq_{\den{\Gamma},\den{\alpha}}$ and $\exists_{\den{\Gamma},\den{\alpha}}$ need a bit more explaination.

\subsubsection{Adjoints to Substitution Functors}
As mentioned in section 1.10, any morphism $f : \den{\alpha} \rightarrow \den{\beta}$
induces a substitution functor $f^* : \textrm{Functor} (Sub(\den{\beta})) (Sub(\den{\alpha}))$.
We've already seen the weakening substitution functor $\pi^*$ induced from 
\begin{align*}
    & \pi : \den{\alpha} \times \den{\beta} \rightarrow \den{\alpha} \\
    & \pi (a , b) = a\\
\end{align*}
Another important functor is the contraction functor, $\delta^*$, induced by
\begin{align*}
    & \delta : \den{\alpha} \rightarrow \den{\alpha} \times \den{\alpha}\\
    & \delta (a) = (a , a)
\end{align*}
The operation $Eq$ is defined to be left adjoint to $\delta^*$.
\[
    Eq \dashv \delta^*
\]
and 
\[
    \exists \dashv \pi^*
\]
Lets expand these out to see what they mean. 
We'll start with equality.
\blue{move this to section 1.11}
We have $\delta_{\Gamma} : \den{\Gamma} \rightarrow \den{\Gamma }\times \den{\Gamma}$. 
The statement $Eq \dashv \delta^*$ elaborates to the following isomorphism of hom sets 
for any $X : \; Sub(\den{\Gamma})$ and $Y : ob \; Sub(\den{\Gamma }\times \den{\Gamma})$.

\[
  Sub(\den{\Gamma} \times \den{\Gamma})[Eq(X),Y] \cong Sub(\den{\Gamma})[X , \delta^*(Y)]  
\]
The definition of $Eq(X)$ for which this holds is 
\[
 Eq(X) = \{ (\gamma , \gamma') : \den{\Gamma} \times \den{\Gamma} \; | \; \gamma = \gamma' \land \gamma \in X\}    
\]
In our set based poset models, the hom set adjunction elaborates to 
\[
    Eq(X) \subseteq_{\den{\Gamma}\times \den{\Gamma}} Y \iff X \subseteq_{\den{\Gamma}} \delta^*(Y)
\]
This corresponds to the $=$-mate rule above. (renaming $X$ as $\Phi$ and $Y$ as $\phi$)

\begin{prooftree}
    \AxiomC{$\Gamma, x : \alpha | \Phi \vdash \phi[x/y]$}
    \doubleLine
    \RightLabel{$=$-mate}
    \UnaryInfC{$\Gamma, x : \alpha, y : \alpha | \Phi, x =_{\alpha} y \vdash \phi$}
\end{prooftree}

Now for $\exists$. We have $\pi_{\Gamma,\alpha} : \den{\Gamma} \times \den{\alpha} \rightarrow \den{\Gamma }$.
The statement $\exists \dashv \pi^*$ elaborates to the following isomorphism of hom sets 
for any $X : ob \;Sub(\den{\Gamma} \times \den{\alpha})$ and $ Y : ob \;Sub(\den{\Gamma})$
\[
    Sub(\den{\Gamma})[\exists(X),Y] \cong Sub(\den{\Gamma} \times \den{\alpha})[X, \pi^*(Y)]
\]
The definition of $\exists$ for which this holds is 
\[
    \exists(X) = \{ \gamma  : \den{\Gamma}  \; | \; \exists (a : \alpha). (\gamma , a) \in X \} 
\]
In out set based poset models, the homset adjunction elaborates to 
\[
    \exists(X) \subseteq_{\den{\Gamma}} Y \iff X \subseteq_{\den{\Gamma }\times \den{\alpha}} \pi^*(Y)
\]
This corresponds to the $\exists$-mate rule above. (renaming $X$ as $\phi$ and $Y$ as $\psi$)
\begin{prooftree}
    \AxiomC{$\Gamma | \exists (x : \alpha). \phi \vdash \psi$}
    \doubleLine
    \RightLabel{$\exists$-mate}
    \UnaryInfC{$\Gamma , x : \alpha | \phi \vdash \psi $}
\end{prooftree}


\subsection{Demonstration of Regular Logic}

\begin{comment}
category has to be well powered for Sub(X) to be a category and not just a poset?

Assume we have a model, $F : \mathcal{CL}(\lambda_{1,\times,\rightarrow}) \rightarrow \mathcal{C}$, of the object language.
For our simply typed object language, this is just a functor from the \textit{syntactic} category into a cartesian closed category.
We will need $\mathcal{C}$ to also have pullbacks (pullbacks and binary products yield equalizers) in order to interpret derivations.
Propositions in context $\Gamma$ will be interpreted as subobjects of $F(\Gamma)$. 

For any term context, $\Gamma$, there is a logic in the context $\Gamma$. The structure of this logic is posetal. 
\begin{itemize}
    \item binary products + pullbacks $\implies$ equalizers
    \item every equalizer is a monomorphism
    \item pullbacks preserve monomorphisms
    \item Every category with pullbacks of monomorphisms has a contravariant functor $Sub:C^{op}\rightarrow Pos$ to the category of posets called the subobject poset functor, making it into a hyperdoctrine.
\end{itemize}



    
We say that the model satisfies an algebraic equation $\Gamma \vdash M =_{\alpha} M'$ (equation with no propositional context) if 
\[
    \den{\Gamma \vdash M : \alpha}_F = \den{\Gamma \vdash M' : \alpha}_F    
\]
the denotations are equal morphisms in category $\mathcal{C}$. 
Proof derivations with a non empty propositional context can be interpreted in $\mathcal{C}$ assuming $\mathcal{C}$ has equalizers.
\blue{$\mathcal{C}$ has equalizers if it has binary products and pullbacks. Consider the pullback of the two morphism along the diagonal.}
Lets consider how to denote 
\[
\Gamma | \Phi \vdash \psi    
\]
Note that $\den{\Gamma} : ob \;\mathcal{C}$. We take 
\end{comment}

\end{document}